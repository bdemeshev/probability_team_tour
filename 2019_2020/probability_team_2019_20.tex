\documentclass[11pt, a4paper]{article}


\usepackage{gensymb}

\usepackage{fontspec}
\usepackage{polyglossia}
\usepackage{pstricks-add}

\setmainlanguage{russian}
\setotherlanguages{english}

% download «Linux Libertine» OTF-fonts:
% http://www.linuxlibertine.org/index.php?id=91&L=1
\setmainfont{Linux Libertine O} % or Helvetica, Arial, Cambria
% why do we need \newfontfamily:
% http://tex.stackexchange.com/questions/91507/
\newfontfamily{\cyrillicfonttt}{Linux Libertine O}
\newfontfamily{\cyrillicfont}{Linux Libertine O}
\newfontfamily{\cyrillicfontsf}{Linux Libertine O}
 
\usepackage{etoolbox} % to use ifdef, must be after babel

 

\usepackage[paper=a4paper,
top=15mm,
bottom=13.5mm,
left=13mm, right=13mm, includefoot]{geometry}

\usepackage{etex} % расширение классического tex
% в частности позволяет подгружать гораздо больше пакетов, чем мы и займёмся далее




\usepackage{makeidx} % для создания предметных указателей
\usepackage{verbatim} % для многострочных комментариев
%\usepackage[pdftex]{graphicx} % для вставки графики
% omit pdftex option if not using pdflatex

\usepackage{comment} % для команды excludecomment


%\usepackage{dsfont} % шрифт для единички с двойной палочкой (для индикатора события)


% \usepackage[usenames, dvipsnames, svgnames, table, rgb]{xcolor}

\usepackage{colortbl}



\usepackage[colorlinks, hyperindex, unicode, breaklinks]{hyperref} % гиперссылки в pdf

\usepackage{amssymb}
\usepackage{amsmath}
\usepackage{amsthm}

\usepackage{bbm} % шрифт - двойные буквы
\usepackage{bm}
\usepackage{color}

\usepackage{multicol}
\usepackage{multirow} % Слияние строк в таблице

\usepackage{textcomp}  % Чтобы в формулах можно было русские буквы писать через \text{}


\usepackage{subfigure} % для создания нескольких рисунков внутри одного

\usepackage{tikz, pgfplots} % язык для рисования графики из latex'a
\usetikzlibrary{trees} % прибамбас в нем для рисовки деревьев
\usetikzlibrary{arrows} % прибамбас в нем для рисовки стрелочек подлиннее
\usepackage{tikz-qtree} % прибамбас в нем для рисовки деревьев

\usepackage{enumitem} % развернутые списки

% свешиваем пунктуацию (т.е. знаки пунктуации могут вылезать за правую границу текста, при этом текст выглядит ровнее)
\usepackage{microtype}

% более красивые таблицы
\usepackage{booktabs}
% заповеди из докупентации:
% 1. Не используйте вертикальные линни
% 2. Не используйте двойные линии
% 3. Единицы измерения - в шапку таблицы
% 4. Не сокращайте .1 вместо 0.1
% 5. Повторяющееся значение повторяйте, а не говорите «то же»

\usepackage{epigraph}

% по поводу заголовков разделов в колонтитулах
% https://tex.stackexchange.com/questions/236705
% поэтому выбрали titleps вместо fancyhdr
\usepackage{titleps} % заголовки страниц


\newpagestyle{mypage}{%
  \headrule
  \sethead{\sectiontitle}{}{\subsectiontitle}
  \setfoot{}{}{\thepage}
}
\settitlemarks{section,subsection,subsubsection} % !!!!!!no space after comma!!!!!!
\pagestyle{mypage}





\DeclareMathOperator{\Lin}{\mathrm{Lin}}
\DeclareMathOperator{\Linp}{\Lin^{\perp}}
\DeclareMathOperator*\plim{plim}
\DeclareMathOperator{\grad}{grad}
\DeclareMathOperator{\card}{card}
\DeclareMathOperator{\sgn}{sign}
\DeclareMathOperator{\sign}{sign}

\DeclareMathOperator*{\argmin}{arg\,min}
\DeclareMathOperator*{\argmax}{arg\,max}
\DeclareMathOperator*{\amn}{arg\,min}
\DeclareMathOperator*{\amx}{arg\,max}
\DeclareMathOperator{\Var}{Var}
\DeclareMathOperator{\Cov}{Cov}
\DeclareMathOperator{\Corr}{Corr}
\DeclareMathOperator{\pCorr}{pCorr}
\DeclareMathOperator{\E}{\mathbb{E}}
\let\P\relax
\DeclareMathOperator{\P}{\mathbb{P}}


\newcommand{\cN}{\mathcal{N}}
\newcommand{\cU}{\mathcal{U}}
\newcommand{\cBinom}{\mathcal{Binom}}
\newcommand{\cExp}{\mathcal{Exp}}
\newcommand{\cPois}{\mathcal{Pois}}
\newcommand{\cBeta}{\mathcal{Beta}}
\newcommand{\cGamma}{\mathcal{Gamma}}

\def \R{\mathbb{R}}
\def \N{\mathbb{N}}
\def \Z{\mathbb{Z}}





\newcommand{\dx}[1]{\,\mathrm{d}#1} % для интеграла: маленький отступ и прямая d
\newcommand{\ind}[1]{\mathbbm{1}_{\{#1\}}} % Индикатор события
%\renewcommand{\to}{\rightarrow}
\newcommand{\eqdef}{\mathrel{\stackrel{\rm def}=}}
\newcommand{\iid}{\mathrel{\stackrel{\rm i.\,i.\,d.}\sim}}
\newcommand{\const}{\mathrm{const}}


% вместо горизонтальной делаем косую черточку в нестрогих неравенствах
\renewcommand{\le}{\leqslant}
\renewcommand{\ge}{\geqslant}
\renewcommand{\leq}{\leqslant}
\renewcommand{\geq}{\geqslant}



\AddEnumerateCounter{\asbuk}{\russian@alph}{щ} % для списков с русскими буквами
\setlist[enumerate, 2]{label=\asbuk*),ref=\asbuk*}




% делаем короче интервал в списках
\setlength{\itemsep}{0pt}
\setlength{\parskip}{0pt}
\setlength{\parsep}{0pt}

% \newenvironment{problem}{}{}
% тут перещёлкиваем комментарий, чтобы убрать или показать решения:
% \newenvironment{sol}{}{} % with solutions
% \excludecomment{sol} % without solutions



\unitlength=0.6mm


%%%%%%%%%%%%%%%%%% вставки
%%%%%%%%%%%%%%%%%%%%%%%%%%%%%%%%%%%%%%% Списки без уродских отступов
\newenvironment{enumerate*}{
\begin{enumerate}
  \setlength{\itemsep}{0pt}
  \setlength{\parskip}{0pt}
  \setlength{\parsep}{0pt}
}{\end{enumerate}}

\newenvironment{itemize*}{
\begin{itemize}
  \setlength{\itemsep}{0pt}
  \setlength{\parskip}{0pt}
  \setlength{\parsep}{0pt}
}{\end{itemize}}

\abovedisplayskip=0mm
\abovedisplayshortskip=0mm
\belowdisplayskip=0mm
\belowdisplayshortskip=0mm


% https://tex.stackexchange.com/questions/241343
% https://tex.stackexchange.com/questions/168480
\emergencystretch 5em % разрешаем дополнительные пробелы для упаковки параграфа до правой границы



\theoremstyle{definition}

\renewcommand{\arraystretch}{1.6}


\begin{document}




\section*{Условия задач праздника}
\subsection*{Тур 1}
\begin{enumerate}
    \item Комбинаторика
%Бабкен
\begin{enumerate}
    \item «Шаловливый Циклоп» (Дуэль)
    
    Бандит с кликухой Циклоп вышел из своей, как он её назвает, берлоги порезвиться. Он захватил с собою рогатку и 3 камня. Видит — на соседнем ранчо гуляют 20 коров (мало кто знает, но всех их зовут Мартами, потому что - ну а как ещё назвать корову?). Сколькими способами Циклоп может
    подбить 3 Март, учитывая то, что для него все коровы одинаковые.
    
    \item «Вилкой в глаз или в кости раз?» (Дуэль)
    
    Заключенные в тюрьме ковбои Матвей и Ваня решили сыграть в игру: Каждый из них подкидывает кость и, если в сумме выпадает 6, то выигрывает Ваня. А если выпадет 10, то выигрывает Матвей. Найдите вероятность того, что победит Матвей.
    
    \item «Ограбление поезда» (15 баксов)
    
    Ковбой «Лютый» Джо грабит со своей бандой пассажирский поезд с 9 вагонами. В его команде самые заядлые разбойники: «Мерзкий» Том, «Грязный» Майк, «Меткий» Дик и «Дерзкий» Билли.  Сколькими способами «Лютый» Джо может рассадить своих ковбоев по вагонам, при условии, что все они должны ехать в различных вагонах.
    
    \item «Домино» (Дуэль)
    
    Индейцы Певучий Петух и  Белый Филин раскладывают костяшки домино размером 1x2 по квадрату размера 5х5. У каждой из них по одной костяшке. Какова вероятность того, что их костяшки сойдутся в центре квадрата? Домино налагется друг на друга.
    
    \item «Пушка Моти» (20 баксов)
    
    В треугольнике написаны числа. На вершине 1, затем идут 2 двойки, потом 3 тройки и так доходим до n энок.(n - фиксированное число). Потом появился злобный ковбой Мотя и размешал своей пушкой все числа в треугольнике. Найдите веротяность того, что вытащится i-ое число.
    
    \item «Золотой куб ковбоя» (20 баксов)
    
    Ходят слухи, что весь Дикий Запад гнался за драгоценным кубом. Этот куб был найден «Тварью» Биллом (его так прозвали, потому что он тесно дружил с шерифом Иваном). Но нашел он его не целым. Дело в том, что куб состоит из 8 маленьких кубиков. Куб сделан из слоновой кости, а снаружи покрыт позолотой. 

    К тому времени совсем стемнело, и ковбою пришлось собирать куб наощупь. Причем нужно было действовать быстро, потому что за эти кубом гнались самые заядлые убийцы. Какова вероятность, что ему удастся правильно собрать куб (то есть позолотой наружу)? 
    
\end{enumerate}    


    \item Условные вероятности 
%Ваня
\begin{enumerate}
    \item История одного шерифа
    \newline
    Вероятность застать шерифа в городе зависит от двух факторов: первый — намечается ли в городе стычка, и второй — приедут ли молоденькие девушки из соседнего поселения продавать свой урожай на местную ярмарку. Данная вероятность равна 0.18, если девушки не приедут и стычка не намечается , 0.9 — если стычка намечается и девушки приедут. 0.54 — если намечается только стычка и 0.36 — если приедут девушки, но стычка не намечается. Стычки происходят с вероятностью 0.4, а девушки приезжают с вероятностью 0.6

    Найдите вероятность того, что девушки приехали в город на ярмарку, если шериф в городе?
    \newpage
    \item   ЯндексПогода от дикого запада
      \newline
      В одной из альтернативных вселенных каким-то случайным образом армянский и уральский путешественники оказались на Диком Западе. И по странной цепочке случайных событий они оба попали в плен банде Гадких Койотов. И им уже было грозила неминуемая смерть, как вдруг по совершенной случайности им выставили следующее условие: если они предскажут завтрашнюю погоду правильно, то их отпустят , иначе им крышка. Ясная погода на диком западе бывает с вероятностью 0.7, а пасмурная — 0.3. Армянский путешественник, конечно, совсем не разбирался в погоде, однако он был знатоком случайных процессов и поэтому угадывал погоду правильно с вероятностью 0.7. А уральский путешественник ни в чем не разбирался, однако хорошо умел ругаться и копировать решения других, в общем был конъюнктурщиком, поэтому он с вероятностью 0.9 просто копировал ответ второго путешественника, а с вероятностью 0.1 выбирал противоположный ответ.

    Какова вероятность того, что завтра будет ясный день, если уральский путешественник спрогнозировал ясную погоду?
    \item Обыкновенная пятница на диком западе 

    Каждую пятницу в местном салуне маленького городка Фемвиль в глубинке Мексики регулярно собираются 20 свободных и независимых женщин  , которые расслабляются после тяжелой рабочей недели за испитием алкогольных напитков. Когда девушка переберет, то она может полезть в драку. Из-за того, что это происходит регулярно, в этот салун при возникновении драки приезжает местный шериф и всех арестовывает, чтобы найти зачинщицу, проводя допрос. Если девушка действительно виновна , то шериф поймёт это с вероятностью 0.9 и арестует. Если если девушка не лжёт, то он может ошибиться с вероятностью 0.05. Найдите условную вероятность того, что арестована невиновная , если шериф решил, что она лжёт .
    \item Умельцы кидать палку[для дуэлей]

    Индейцы из племени гуронов попадают попадают копьем в оленя со 100 метров с вероятностью 0.7, из апачи — 0.8. , а могикане — 0.9. Причем индейцев из племени гуронов столько, сколько апачи и могикан вместе взятых. А апачи и могикан одинаковое количество. Какая вероятность того, что случайно выбранный индеец попадёт в оленя со 100 метров? 
    \item Кто главный в семье? [для дуэлей]
    \newline
    Существовало одно племя индейцев, чья культура сильно отличалась от европейской — они были свободны от общественных стереотипов и комплексов, а поэтому строили свою жизнь согласно зову сердца и души. В том числе они были свободны от предрассудков относительно того, людям каких полов позволительно быть в любовных отношениях, а каким нет . Кроме этого мужчины, равновероятно как и женщины, могли внутри своей пары выполнять как роль охотника, так и роль  домохозяина. Возьмём случайную пару из этого племени: известно, что в этой паре роль охотника выполняет девушка. Определите вероятность того, что роль домохозяйки в этой паре выполняет тоже девушка, но уже другая. 
    \item Кто боец? [для дуэлэй]
    \newline
    Главарь одной банды понимал, что скоро будет серьезное вооружённое столкновение с вражеской бандой за сферы влияния и репутацию. Поэтому ему надо было понять, сколько из его бойцов умеют обращаться с огнестрельным оружием. Он собрал письменные ответы своих бандитов и оказалось, что $\frac{2}{3}$ участников банды написали, что умеют обращаться с оружием. Однако до главаря дошла информация, что не все ответы истинные . Так как во время того, как решить, что писать, люди кидали кость, и если выпадала 1, то писали, что умеют, если 6 - не умеют, а если от 2 до 5, то писали честно. Каково истинное количество бойцов, которые умеют обращаться с оружием? 
\end{enumerate}    
    
    
    
    \item Пределы
%Дима
\begin{enumerate}
    \item Некоторые дельцы на Диком Западе задумались о том, чтобы расширить барабан револьвера для того чтобы он вмещал $n$ пуль. Допустим, что они изготовили $n$ таких барабанов, причём в каждом из них был оставлен один холостой заряд. Во время тестирования бравый ковбой Бабкен Джонс заряжает револьвер, делает один выстрел и меняет барабан на следующий. Найдите вероятность того, что ему никогда не попадётся холостой заряд, если $n \to \infty$.
    \item Где-то в \textbf{Те}рвер\textbf{хасе} есть стрельбище, на которм ковбои каждый отттачивают свои навыки обращения с револьвером. В точности некоторых ковбоев сомневаться не приходится, поэтому каждую мишень независимо от других нужно менять с вероятностью 0.8. Всего на стрельбище 20 мишеней. Пусть $X$ - количество мишеней, которое нужно заменить в n-ный день. Найдите $\plim_{n \to \infty}{\bar X}$ 
    \item В одном из многочисленных салунов на Диком Западе работает недобросовестный бармен, разбавляющий виски в каждом стакане водой, причем доля долива равномерно распределена на отрезке $[0; 0.2]$. Найдите $\plim_{n \to \infty}{\frac{11}{1+\bar X}}$
    \item Жители \textbf{Матc}с\textbf{тат}чуссетса организовали экспедицию по захвату соседних территорий. Перед каждым передвижением команда ковбоев выбирает равновероятно одну из $n$ точек. Расстояния в штате немаленькие, поэтому после того как были исследованы 4 точки, экспедиция возвращается в родной город для поплнения запасов провизии (до этого момента в город попасть нельзя). Пусть $X^{(k)}$ - количество посещений точки с номером $k$. Чему равен $\plim_{n \to \infty}{\frac{{X^{(k)}}}{n}}$? %не очень поняла условие - мы шагаем бесконечное число раз? и можно ли попасть на начальную точку до  5 шага?%
    \item Бравый ковбой Джонс стал самым молодым шерифом в истории города. В его обязанности по традиции стал входить поиск бандитов, угрожающих безопасности города и, разумеется, репутации шерифа. Поэтому шериф каждый день вызывает одного жителя (или гостя, об этом история умалчивает) на допрос. Чутье шерифа улучшается со временем, поэтому вероятность того, что шериф выбрал персону, которая имеет основания быть подозреваемой равна $\frac{3n}{4(n+1)}$, где $n$ - это порядковый номер проверяемого человека. Найдите вероятность того, что бесконечно набравшийся опыта шериф три раза подряд вызовет на допрос законопослушных граждан?
    \item В этой задаче вам снова нужно помочь Джонсу на позиции шерифа. Допустим, что на одном из допросов он решил проверить собеседника на знание теории вероятностей. Для этого он предложил ему найти $\plim_{n \to \infty}{X_{n}}$ и $\plim_{n \to \infty}{\E (X_{n})} $ для случайной величины с 
\[
p_{X_n}(x) = \begin{cases}
\frac{1}{n},  & x=n^2 \\
1 - \frac{1}{n}, & x=0 \\
0, & \text{иначе}
\end{cases}
\]
или ваши отношения с шерифом будут испрочены раз и навсегда.
\end{enumerate}    

\item Нормальное распределение
\begin{enumerate}
    \item Шериф вышел на охоту. Для поимки преступника он делает следующее: \begin{itemize}
        \item[1] Наугад выбирает одну случайную величину, а всего их $n$ штук;
        \item[2] Известно, что случайная величина $\xi_i\sim \cN(\mu_i, \sigma_i^2)$, $\mu_i<\infty$, $\sigma^2_i <\infty$;
        \item[3] Значение этой случайной величины реализуется;
        \item[4] Шериф идет в место, координата которой соответствует реализации этой случайной величины, в бесконечно длинном одномерном городе. 
    \end{itemize}
    
    Каково математическое ожидание координаты точки, в которую идет шериф?
    
    \item У ковбоя, которого пытается поймать шериф, другая схема. Он выбирает, где ему спрятаться в том же бесконечном одномерном городе с помощью некоторой случайной величины $\eta\sim \cN(\mu, \sigma^2)$, причем $\mu\sim \cN(0; 1)$. Известно, что сначала реализуется значение математического ожидания, затем — случайной величины. Найдите математическое ожидание случайной величины $\eta$.%разве мы так не получаем целый набор случайных величин с разными математическими ожиданиями?%
    
    \item Изрядно выпившая местная ведьма со своим молодым человеком (по совместительству — ковбоем) выбирают точки на плоскости. Ведьма выбирает абсциссы, её молодой (или уже не молодой\ldots) человек — ординаты. Они решили выбрать две точки, чтобы провести через них прямую, олицетворяющую их. Известно, что значения абсцисс точек $\xi_1$ и $\xi_2$ взяты из нормального распределения: $\xi_1, \xi_2\sim \cN(\mu_1, \sigma_1^2)$. Значения ординат $\eta_1$ и $\eta_2$ также распределены нормально, однако с другими параметрами: $\eta_1, \eta_2\sim \cN(\mu_2, \sigma_2^2)$. Все случайные величины независимые. Какова вероятность того, что угол между получившейся прямой и горизонтальной осью будет меньше 45$\degree$? Ответ выразите через функцию нормального стандартного распределения.
    
    \item Следующая величина называется дифференциальной энтропией: \[\int\limits_{\R} p(x)\ln p(x) dx \]
    где $p(x)$ — функция плотности.
    
    Чему равна дифференциальная энтропия нормального распределения с параметрами $\mu$ и $\sigma^2$?
    
    \newpage
    
    \item Некоторые хитрые ковбои и благородные плантаторки для поиска начальных моментов высших порядков используют \textit{производящую функцию моментов} $M(t)$:\[M_{\xi}(t)=\E [e^{tx}]=\int\limits_{\R} e^{tx}p(x)dx, \]
    где $p(x)$ — функция плотности случайной величины $\xi$. Так, чтобы найти начальный момент случайной величины $\xi$ порядка $n$ нужно сделать следующее: \[\left. \cfrac{\partial^n}{\partial t^n}M_{\xi}(t)\right|_{t=0}=\E \xi^n \] 
    
    Найдите производящую функцию моментов для нормального распределения. Чему равен начальный момент порядка $n$?
    
    \item Теодор Рузвельт (единственный президент США, который был ковбоем) любит составлять любовные треугольники (фантазия автора). Длины он выбирает следующим образом: реализуются значения двух случайных величин $\xi, \eta\sim \cN(0,1)$, затем Теодор берет модуль независимых получившихся чисел и рисует прямоугольный треугольник с катетами, длины которых равны полученным числам. Ключевым неразрешенным вопросом для него было то, какого же все-таки распределение длины гипотенузы? Помогите Теодору Рузвельту найти функцию плотности этой случайной величины.
\end{enumerate}
    
    \item Геометрия
%Костя
\begin{enumerate}
    \item «Пропавший ковбой»

    Между городами СтатМат и ВерТер 50 километров. Храброго ковбоя Билли отправили из одного города в другой, чтобы перевести ценный груз. Между городами также находится населенный пункт МенЭкз, который растягивается аж на величину 10 километров. Билли не прибыл в город. Найдите вероятность того, что он застрял где-то в городе МенЭкз. 
    \item «Экзамен для шерифа»
    
    Для того чтобы получить сертификат шерифа и следить за порядком в городе, необходимо сдать экзамен по стрельбе. Для этого в академии существуют специальные мишени круглой формы. Мишень состоит из 3 отделов, радиус каждого отличается от предыдущего в 2 раза. За попадание в самый центр ковобой получает 8 очков, за попадание во второй по величине отдел 4 очка, а за попадание в крайний 1 очко. Оцените сверху вероятность стать шерифом, если для получения сертификата требуется набрать не менее 18 очков за 3 выстрела. (Выстрелы независимы) %а как очки распределяются?%
    \item «Вероятностное лассо»
    
    Не многие знают, что ковбои это не только крутые парни, которые участвовали в перестрелках и носили специальные костюмы, но также и настоящие работяги, которые вели свой нелегкий быт, ухаживая за домашними животными. Поэтому атрибутом истинного ковбоя всегда являлось лассо (Длинная веревка с петлей на конце). Любой уважающий себя ковбой должен был уметь самостоятельно изготовить для себя этот инструмент. Бралась веревка длинной 2 метра, и от нее отрезалась часть для лассо. Причем, веревка надрезалась в абсолютно случайном месте, что усложняло жизнь ковбоев. Однако древние легенды говорили о том, что если длинна отрезанной части составляла не менее 1.6 метра, то этого было достаточно для изготовления качественного лассо. Найдите вероятность того, что ковбой отрежет от изначальной веревки достаточный кусок для своего лассо.
    \item «Неспокойный город»
    
    В городе ВерТер находится бар «У Пуссона», где регулярно собираются ковбои, чтобы перекусить и пропустить пару стаканчиов текилы. Также в городе ВерТер происходят перестрелки между двумя местными отпетыми ковбоями с вероятностью 3/4. Известно, что перестрелка состоится, если в баре «У Пуассона» в одно время пересекутся оба отпетых ковбоя. Также мы знаем, что ковбои приходят в бар с 16.00 до 18.00. Найдите время трапезы одного отпетого ковбоя.
    \item «Задача Бюффона»
    
    Древние индейцы племени Токку не знают о теории вероятностей, но у них есть поверье, тесно с ней связанное. Поверье гласит: «Достойный выкурить трубку мира в нашем племени должен победить в испытании, которое заключается в том, чтобы случайно бросить иголку на разлинованный лист. Испытание считается успешным. если иголка пересечет одну из линий листа.» Найдите вероятность победы в испытании, если расстояние между линиями 0,5 см, а длинна иголки 0,3 см. (Иголка не может воткнуться лист, а просто падает)
    \item «Финальное испытание»
    
    Теперь вы удостоились чести сидеть с индейцами в одном кругу и даже возможно курить трубку мира вместе с ними. Однако вождь имеет сомнения по поводу вашей кандидатуры. Он предлагает вам игру, в которой он и вы загадываете числа от 1 до 5 один раз. Если произведение двух ваших чисел будет больше или равно 3, то вы получаете почет и признание в племени. В другом случае, вас съедают. Найдите вероятность выжить в такой непростой ситуации.
    
\end{enumerate}    
    
    \item Марков и Чебышев
%Даниил 
\begin{enumerate}
    \item Датч Ван Дер Линде — главарь группы свободоборцев — стреляет из револьвера по законникам в Блэкуотере. В среднем каждый пятый патрон достигает цели. Оцените вероятность того, что свободоборец не подстрелит законника парой  шестизарядных револьверов Шофилда.
    \item Артур Морган поставил свою арабскую лошадь на то, что Двуглазый Джо не сможет попасть в 5-ти дюймовую бутылку с расстояния в 10 ярдов. Обычно Джо попадал в 4-дюймовую бутылку, поэтому он без страха пошел на спор. Приблизительно оцените вероятность того, что Артур Морган пойдет домой пешком.%не поняла задачу без решения%
    \item По наблюдением опытного налётчика Джона Марстона, количество денег, перевозимое законниками отклоняется от среднего не более, чем на 1000 рикардианских милосов в 95 процентах случаев. Какие значения дисперсии количества перевозимых денег может рассчитать Джон Марстон?
    \item Вы — типичный недобропорядочный чиновник, которого постоянно грабят по пути к куртизанкам. Пусть в среднем вас грабят в 3-х из 4-х случаев. Пользуясь неравенством Чебышева, оцените вероятность того, что из 100 поездок у вас не получится отдохнуть от 70 до 80 раз.  
    \item Сэди Адлер потеряла мужа при ограблении поезда. Чтобы утешить свой гнев, Сэди каждый день стреляет по уткам и посещает психолога.
    Вы — недипломированный психолог, перед которым стоит задача понять, как Сэди Адлер реагирует на Ваше лечение. Пусть
    в среднем Сэди Адлер убивает по 50 уток в день. Оцените вероятность того, что ваше лечение не подействует на мисс Адлер и на следующий день она убьет 
    не меньше 60 уток.
    \item Мика Белл любил оружие и ненавидел закономерности. Однажды в магазине старьевщика он приобрел оружие, которое с вероятностью 0.3 даёт осечку.
    С помощью неравенства Чебышева оцените вероятность того, что из 300 выстрелов, разница между числом осечек и средним числом осечек окажется меньше 50.
\end{enumerate}    
    \newpage
    \item $O$-малые
%Камилла
\begin{enumerate}
    \item «Дебош в салуне» (дуэль)
    
    Уже слепой Джо, несмотря на явные проблемы со зрением, смог заложить 3 динамита по периметру местного салуна «Дикое перекати-поле». Время до срабатывания каждого динамита имеет равномерное на отрезке $[2, 8]$ распределение. Известно, что стены салуна не очень крепкие и могут выдержать лишь один взрыв, но после второго они немедленно разваливаются, а тайник с хозяйским золотом  становится доступным к похищению. Сегодня салун будет работать ещё 
    5 часов. Какова вероятность того, что салун не развалится до закрытия, и хозяин Скупой Хуарес не потеряет свои пожитки?
    
    \item «Резервация в поисках агавы» (дуэль)
    
    В местной резервации пришло время заготовки текилы из выращиваемой индейцами голубой агавы. Текилу могут готовить лишь трое истинных мудрецов — агаводельцев, которым открыт тайный рецепт этого напитка. Количество текилы (в литрах), приготовленной каждым мудрецом распределено равномерно на отрезке $[0, 6]$. Пусть после того, как все мудрецы сделали текилу, их бутылки расположили по возрастанию количества текилы в бутылке — получили $Y_1 \leq Y_2 \leq Y_3$. Найдите функцию плотности $Y_1$ по определению. 
    \item «Перегон скота» (дуэль)
    
     В маленьком мексиканском штате на закате ковбой Джанго идет по побережью в сторону резервации, а ковбой Дэнди идет по побережью из резервации. Оба ковбоя гонят коров, при этом в стаде у Джанго 150 коров, а у Дэнди — 100. На середине реки в своей скромной лодке сидит гаучо, справа он видит, что идёт Джанго, а слева - Дэнди. Гаучо наблюдает за стадами. Вероятность за малый интервал времени для гаучо увидеть, что одна корова идет в сторону  резервации (стадо Джанго) и со стороны резервации (стадо Дэнди)  прямо пропорциональны этому интервалу, для коров из стада Джанго с коэффициентом пропорциональности 5, а из стада Дэнди — 3. Опишите динамику количества коров по разные стороны от гаучо за малый интервал времени.    
    \item «Кантри - наше всё» %10
    
    Уже много лет назад гаучо Фернандо украл глаз известного во всей Мексике Слепого Джо. С этих пор Джо ходит по разным салунам, надеясь найти Фернандо и вернуть свой драгоценный глаз, чей редкий окрас всегда забавлял Джо. Единственное, что могло стать для Джо препятствием, — это неприязнь к кантри, которую слишком часто играли в салунах. Вероятность того, что конец кантри-представления произойдет в малый интервал времени пропорционально этому  интервалу. То же самое выполняется для вероятности того, что неприятие Джо кантри сподвигнет его уйти в закат за малый интервал времени, причем окончание кантри-представления становится для Джо хорошей новостью, и после этого он готов надолго задержаться в салуне. При каком соотношении коэффициентов пропорциональности для вероятности окончания кантри-представления и вероятности ухода Джо в закат, вероятность того, что Джо не уходит в закат в силу окончания кантри-представления равняется $\frac{1}{2}$.
    
    
    \item «Агава найдена» %15
    
    Потомки древних ацтеков всегда любили текилу, но делать её могли лишь самые умелые члены рода. Количество текилы (в литрах), приготовленной таким умельцем распределено равномерно на отрезке $[0, 5]$. Пять умельцев сделали текилу в преддверии приближающегося праздника (ух ты, а вы как раз сейчас на нём находитесь). После того, как все умельцы (странно, но всех их звали Кронк - хотя это имя известного народного любимца инков в годы правления императора Кузко) сделали текилу, они пошли помогать Изме - текущей правительнице рода (да-да, она-таки добилась своего). Найдите вероятность того, что умелец-текилоделец-знаток беличьего языка, который приготовил наибольшее количество текилы, сделал более 4 литров. 
    
    P.S. Жми на рычаг, Кронк!
    \newpage
    \item «Мир Дикого Запада» %20
    
    Приключения Слепого Джо по Мексике продолжаются. В этот раз судьба закинула его в Чиуауа. Сейчас Слепой Джо со словами «I'm gonna take my horse to the old town road» случайно забрел на ранчо к Иниго Монтойя (их совместное прошлое отмечено известной трагедией). У Иниго Монтойя восемь соток грядки, на которой посажена агава. Вероятность того, что за малый интервал времени Слепой Джо пойдет вперёд, к следующей по номеру сотке, и назад, к предыдущей по номеру сотке, прямо пропорциональны этому интервалу времени.
    \[
    \P(\text{Слепой Джо идёт вперёд за интервал} \  [t, t + \Delta]) = \lambda_f \Delta + o(\Delta)
    \]
    \[  
    \P(\text{Слепой Джо идёт назад за интервал} \  [t, t + \Delta]) = \lambda_b \Delta + o(\Delta)
    \]
    Если Слепой Джо находится на восьмой сотке, то вперед он идти не может, если на первой — назад он идти не может. Почему не может? А потому, что много лет назад он лишил жизни отца Монтойя, и выйди он с грядки агавы, Монтойя непременно заметит Слепого Джо, после чего неминуемо последует фраза: «Привет. Меня зовут Иниго Монтойя. Ты убил моего отца. Приготовься умереть.» 
    
    Итак, вам предстоит найти вероятность пребывания на третьей сотке, выраженную через $\lambda_f$  и  $\lambda_b$, если вероятность пребывания на какой-либо сотке постоянна во времени.
\end{enumerate}    
    
    
    
    \item Метод первого шага
%Ксю    
\begin{enumerate}
    \item Ковбой Джонни немного перебрал в таверне и идёт по улице следующим образом: сначала делает шаг вперед, после чего поворачивается на 90 градусов в одну из сторон с равной вероятностью, после этого снова делает шаг, и так далее. Все шаги имеют длину 1. Найти матожидания квадрата расстояния от начальной точки до положения пьяницы через 100 шагов.
    \item Ближе к утру Джонни протрезвел и обнаружил, что он попал в лабиринт, состоящий из 5 комнат, соединённых системой двустороннних порталов. Каждый портал соединяет только одну пару комнат лабиринта, все комнаты соединены между собой. Порталом можно пользоваться неограниченное чило раз. Джонни появился в комнате $v_1$ и прыжками перемещается между комнатами. В комнате $v_5$ находится выход из лабиринта. Пусть каждый следующий портал для прыжка Джонни выбирает случайно и равновероятно среди всех порталов, включая портал до предыдущей комнаты. Найдите математическое ожидание количества прыжков бравого ковбоя до попадания в $v_5$.
    \item Храбрые ковбои Бабкен и Алекс решили поучаствовать в родео. После успешного выступления  вероятность того, что каждый из них упадёт с ведущей стороны от быка (у Бабкена это справа, у Алекса слева), равна $\frac{1}{4}$, а вероятность того, что каждый из них упадёт с неведущей стороны, равна $\frac{1}{3}$. Вероятность того, что они могут сорваться сзади или спереди от быка, ничтожно мала, а вероятность успешного выступления у Бабкена и Алекса одинаковы между собой и равны $\frac{5}{12}$. Однако если ковбой упал с быка, на следующем раунде пострадавшая рука у него держится хуже, поэтому если раненый Бабкен падает справа, то он опять падает справа с вероятностью $\frac{1}{3}$, слева с вероятностью $\frac{5}{12}$ и не сорвётся с вероятностью $\frac{1}{4}$. Аналогично если раненый Алекс падает слева, то он опять падает слева с вероятностью $\frac{1}{3}$, справа с вероятностью $\frac{5}{12}$ и не сорвётся с вероятностью $\frac{1}{4}$. Найти математическое ожидание числа падений у Алекса и Бабкена в 10 раундах.
    \item Хитрый японский шпион Кедо решил решил оценить численность американской армии и по этому заданию поехал в Техас. Сержант Карп заметил шпиона, поэтому когда Кедо подходит к базе, начинает стрелять. Но стреляет он очень плохо и попадает в японйа с вероятностью $\frac{1}{3}$. Пусть $X$ - это количество выстрелов, a $Y$ - количество промахов до первого попадания. Найдите $\E(X)$, $\E(X^2)$.
    \item После тяжелой работы на ферме Лиза и Давид играют в карты. Если Лиза выигрывает партию, то следующую партию она выигрывает с вероятностью 0,9 и проигрывает с веротностью 0,1. Если же Лиза выигрывает партию, то следующую партию он выигрывает с вероятностью 0,8 и проигрывает с вероятностью 0,2. В предположении, что результат партии зависит только от предыдущей игры, найдите вероятность того, что Лиза выиграет четвёртую партию при условии, что она проиграет первую.
    \item Илья - первый парень на деревне и мечта всех барышень - обладатель шикарной бороды. Периодически для поддержания красоты он заходит в барбершоп, где его за его бороду отвечает сам Борис Д. Несмотря на своё мастерство, Борис Д. с вероятностью 0,05 ошибается и отрезает Илье лишнюю прядь. Илья мгновенно это замечает, приходит в ярость, ломая ножницы барбера, и уходит, хлопнув дверью. В следующий период он ещё дуется на Бориса Д., но в последующий период ему становится совсем невмоготу и он снова бежит стричься. Найдите вероятность того, что в десятом периоде он будет в обиде на Бориса Д., если стрижка в первом периоде прошла без происшествий.
\end{enumerate}    
    
    
\end{enumerate}

\newpage

\subsection{Тур 2}
\begin{enumerate}
    \item Байес 
%Лиза
\begin{enumerate}
    \item «Название» (15 баксов)
    
    Вас выследил и теперь преследует шериф Ксю. Вы убегаете со всех ног, но перед вашей бандой возникла новая преграда – каньон. После долгих переговоров вы единогласно решили, что лучший способ преодолеть его – прыгать на ослах.
    
    Договорились, что все стартуют в одно и то же время, но ослы у всех разные и независимые: время, за которое $i$-ый осел преодолеет каньон имеет экспоненциальное распределение с параметром $i$ (в минутах).
    
    Найдите вероятность того, что 2 осел перепрыгнет каньон не раньше 1 минуты после старта, если известно, что за первые 30 секунд ни один из ослов не преодолел препятствие.
% Если будут спрашивать, то i=1,…, кол-во человек в команде, но вообще это неважно. Можно оставить просто i=1,\ldots,n, чтобы запутать

    \item «Название» (20 баксов)
    
    На подмогу шерифу Ксю пришел комиссар местной полиции Данила. Он почти настиг вас, когда вы решили скрыться в кукурузном поле. Длина и ширина кукурузных полей на Диком Западе имеют совместное равномерное распределение на $[20,40] \times [30,50]$.
    
    На поле какой длины ваша банда ожидает попасть, скрываясь от комиссара Данилы, если ваш гаучо уверяет вас, что в данном регионе ширина поля всегда больше длины?
    
    \item «Название» (10 баксов)
    
    Оторвавшись от комиссара вы врываетесь в ближайший салун, чтобы утолить жажду. Там в это время буянит завсегдатай местных баров Алекс. Он стреляет по бутылкам на стойке бармена, причем довольно метко. После каждого попадания он безумно смеется и стреляет еще более метко — в следующий раз попадает с вероятностью 0.9. Если же он промахнулся, то он ругается и на следующий выстрел попадает в бутылку уже с вероятностью 0.6. Если Алекс промахивается три раза подряд, то он с горя забирает ту бутылку, в которую не попал и уходит из салуна.
    
    Какова вероятность того, что вы застанете лишь 6 выстрелов Алекс, если он ругался, когда вы вошли в салун?
    \newpage
    \item «Название» (15 баксов)
    
    Выйдя из салуна, вы заметили на углу улицы две коробки с надписью «Осторожно! Внутри динамитные шашки!» Порадовавшись пополнению боеприпасов, вы начали вскрывать коробки, но вовремя остановились: если в коробке слишком много взрывчатки, то вы рискуете своими жизнями — кто знает, сколько здесь стоят эти коробки и что с ними делали до этого.
    
    Тогда один из ковбоев сказал: «Вакеро, ты родом из этих краев, ты знаешь, сколько обычно шашек кладут в такие коробки?». На что получил ответ: «Лучше, я знаю таблицу совместного распределения динамита в коробках»
    \begin{center}
	\begin{tabular}{c|c|c|c}
		 & $X_1$ = 1 & $X_1$ = 2 & $X_1$ = 8 \\
		\hline
		$X_2$ = 1 & 0 & $\frac{1}{16}$ & $\frac{1}{8}$ \\
		\hline
		$X_2$ = 2 & $\frac{1}{16}$ & $\frac{1}{16}$ & $\frac{1}{8}$ \\
		\hline
		$X_2$ = 8 & $\frac{1}{8}$ & $\frac{1}{8}$ & $\frac{5}{16}$ \\
	\end{tabular}
	\end{center}
	
	Найдите ковариацию между количеством динамитных шашек в первой и второй коробках, если, пока вы разговаривали, краснокожий из вашей команды вскрыл первую коробку, и там оказалось 2 шашки.
	
    \item «Название» (10 баксов)
    
    Началась перестрелка! Вероятность того, что вы уйдете победителем, определяется случайной величиной $\eta = 2\xi - \mathbb{I}\{\xi~>~\frac{1}{2}\},$ где $\xi \sim U[0,1] \text{— ваша удача}, \, \mathbb{I}\{\xi > \frac{1}{2}\} =
    \begin{cases}
    0, \, \xi \le \frac{1}{2} \\
    1, \, \xi > \frac{1}{2}
    \end{cases}$
    
    Определите вероятность того, что ваша вероятность выжить больше 0.5, если вы точно знаете, что вам обычно не везет: $\xi \le \frac{1}{3}$
    
    \item «Название» (10 баксов)
    
    На ферму уже много повидавшего ковбоя Шамиля приехала банда во главе с Дарьей, которая украла Глаз Уже Слепого Джо у гаучо Фернандо. Ковбои оставили своих лошадей пастись на поле, а сами начали стрелять в воздух, привлекая внимание хозяина. Лошади испугались и перемешались между собой.
    
    Но кто же знал, что именно в это время к Шамилю заехали его старые друзья, Камилла и Илья, которые известны своей меткостью. Выйдя из дома, они одним своим видом повергли банду в бегство.
    
    С какой вероятностью хотя бы один ковбой поедет на своей лошади, если известно, что верный помощник Дарьи, Одек, выбрала не свою лошадь?
\end{enumerate}    
    

    \item Пуассон 
%Давид
\begin{enumerate}
    \item «Поляна Гонзалеса» (10 баксов)
    
    На Диком Западе есть широко известный разбойник, и имя ему - Быстрый Гоназлес. По легендам, Гонзалес обитает на одной поляне, которую люди стараются обходить стороной.
    
    Быстрый Гонзалес наблюдает за парами индейцев, которые забредают на его поляну. Случайная величина $X$ - количество пар, приходящх на его поляну в течение часа, имеет распределение Пуассона с показателем $\lambda = 2$. Какое наиболее вероятное число пар Гонзалес успеет увидеть на поляне?
    
    \item «Нападения Гонзалеса» (15 баксов)
    
    Но быстрый Гонзалес не только следит за парами, но и нападает на них; это всегда неожиданно и невероятно быстро.
    
    Если пара, зашедшая в лес, разлучилась хоть на секунду, то Гонзалес нападает на эту пару. Вероятность того, что пара разлучится равна 0.5.
    
    Посещение парами поляны Гонзалеса представляет собой простейший Пуассоновский поток. В среднем за календарные сутки в лес заходит 12 пар. Найдите вероятность того, что за 12 часов Гонзалес нападёт хотя бы на одну пару.

    \item «Противостояние Гонзалесу» (20 баксов)
    
    Существует лишь один способ, чтобы противостоять невероятной скорости Быстрого Гонзалеса, но для этого вам придёдтся «запачкать свои пальцы».
    
    Будем считать, что Гонзалес в среднем делает одно нападение за 1 секунду. Это очень быстро! Поэтому успех зависит от вашей меткости и скорости. Вероятность, что вы попадёте в такую быструю цель равна 0.1.
    Наносить удары вы можете в среднем лишь один раз в 2 секунды.
    
    Считая, что нападения Гонзалеса и ваши атаки представляют собой простой Пуассоновский поток, найдите вероятность того, что за 6 секунд вы сможете 1 раз успешно атаковать Гонзалеса за совершёные две атаки.
    
    \item «Пролетая над макушкой грабителя» (дуэль)
    
    Грабитель убегал от индейского племени и смог укрыться на вершина дерева так, что его не было видно и невозможно достать оттуда. Но индейцы пытаются попасть в него стрелами издалека, несмотря на это, все индейцы промахиваются, и стрелы летят прямо над головой грабителя.
    
    Грабитель наблюдал за стрелами и увидел необычную закономерность. Полёт стрел через голову грабителя представляет собой простейший пуассоновский поток  со средней интенсивностью 30 стрел в час. Найдите вероятность того, что за 5 минут над головой грабителя пролетит хотя бы одна стрела.
    
    \item «Дуэли в городе» (дуэль)
    
    Каждый день жители маленького городка под названием «Блиф» проводят дуэли насмерть. За один день жители в среднем проводят 3 дуэли: одну «на завтрак», «на обед» и «на ужин». Число дуэлей имеет распределение Пуассона. Найдите вероятность того, что за 4 дня будет проведено 10 дуэлей.
    
    \item «Острый рост» (дуэль)
    
    Иголки пустынного кактуса растут крайне медленно. Если вести пристальное наблюдение за их ростом, то можно увидеть, что кактус, в срднем, отращивает 12 иголок за один месяц. Число иголок, растущих на кактусе, подчиняется распределению Пуассона. Найдите вероятность того, что за неделю вырастет хотя бы 1 иголка. (Считайте, что в месяце всего 4 недели)
    
\end{enumerate}    
    
    
    
    \item Ковариации и ожидание
%Данила
\begin{enumerate}
    \item «От хорошего дерева - хорошее каноэ» (15 баксов)
    
    Куда только не приведет дорога прославленную банду с дикого запада, бегущую от преследований шерифа Лизы и его верного помощника Ивана. 
    На вашем пути открывается простор реки Колорадо, самой большой реки в здешних краях. Чтобы замести следы вы решаете её переплыть. Для этого вам необходимо построить каноэ. На берегу растёт 6 деревьев: Секвоя, Сагуаро, Агава, Полынь, Клен и Кедр. Вы вроде бы где-то читали ботанический справочник, но на диком западе это явление редкое, кажется кто-то недавно разводил с его помощью костер, поэтому вы помните только это: каноэ из одного из этих деревьев точно не выдержет вашего веса и вы начнете тонуть (событие $A$), а как известно мокрый ковбой это мертвый ковбой. Другие два дерева точно вас выдержат и вы сможете успешно сбежать от шерифа (событие $B$). Об остальных деревьях ничего не известно. Материал для каноэ вы выбираете случайно и с одинаковой вероятностью. Найдите коэффициент ковариации между событиями $A$ и $B$.
    % 15 баксов
    \item «Своё каноэ не похвалишь - никто не похвалит» (10 баксов)
    
    Будем думать, что вы выбрали верное дерево. Пробегавший мимо индеец Викэнинниш Большой Томогавк оказался большим мастером по постройке каноэ. Он шепнул вам, что для того чтобы каноэ служило вам верно, на нем необходимо написать священный индейский текст. Викэнинниш (его так же называют грозой отрубленных пальцев) поможет вам, если вы поможете ему. Найдите матожидание величины $X$, и тогда сможете усыпать свое каноэ индейскими рунами. 
\[
   f(x, y)=
   \begin{cases} 
   2(\frac{x^2}{2} + y^2), 0 \leq x \leq 1, 0 \leq y \leq 1, \\
    0, \text{иначе}
    \end{cases}
\]
\newpage
  %10 баксов  
    \item «Дырявая шляпа» (20 баксов)
    
    Уже на середине Колорадо вы замечаете, что прямо за вами на своем каноэ плывет шериф Лиза со своим верным помощником Иваном. 
    У вас есть револьвер с 6 патронами. Вы знаете, что приследование прекратится, как только вы собьёте шляпу с шерифа. Кстати, попасть вы в неё можете с вероятностью 0,6. Найдите среднеожидаемое количество выстрелов. 
%20 баксов
    \item «Лимонадный Джо» (10 баксов)
    
    Ковбой Джо обожает пить лимонад со вкусом лимона - $X$ и груши - $Y$. Совместное распределение выпитых в день бутылок задается следующет таблицей:
    
    \begin{tabular}{c|c|c|c|}
			\hline $Y \backslash X$ & $X=1$ & $X=3$  & $X=5$  \\
			\hline  $Y=0$ & 0,1 & 0,2 & 0,25 \\
			\hline  $Y=2$&  0,3& 0,1 & 0,05 \\
			\hline
		\end{tabular} \\
		
		Найдите коэффициент корреляции между событиями $X$ и $Y$.
%10 баксов	дуэль	
    \item «Уже слепой Джо» (10 баксов)
    
    На диком диком западе, среди необузданных жарких прерий живет легендарный ковбой - Одноглазый Джо. Совсем недавно, группа каких-то молоденьких шерифов с бешенным конём уничтожила его прославленную банду и отбрала у Джо его самую дорогую вещь - его единственный глаз. Но Джо не стал падать духом: «С одним как-то жил, без них тоже как-нибудь проживу». Он стал искать новую банду разбойников для борьбы с молодыми шерифами и бешенным конем. Однажды слепой Джо решил пострелять по 24 бутылкам. Так как раньше он был хорошим ковбоем (даже получил прозвище «одноглазый сокол»), он до сих пор попадает в каждую восьмую. Сколько в среднем раз он попадёт в бутылку? Какова дисперсия промахов в бутылку? 
    % 10 баксов дуэль
    
    
    
    \item « Патроны слепого Джо» (10 баксов)
    
    Стреляя по бутылкам слепой Джо израсходовал все свои патроны. Чтобы получить дополнительные и защищаться ковбоев с бешенным конем ему надо открыть сейф с кодовым замком. Для этого нужно знать $\alpha$ из ковариационной матрицы, если $\Corr(X, Y) = 0.55$
\[
\begin{pmatrix}
4 & \alpha \\
\alpha & 21 
\end{pmatrix}
\]
Помогите Джо узнать код.   
\end{enumerate}    
 % 10 баксов    дуэль
    
    \item Дифференциальные формы
%Матвей
\begin{enumerate}
    \item «Практическая орнитология» (10 баксов)
    
   
Разбойник Янус Двуликий хитростью поймал ковбоя Бесстрашного Джо и закопал его по шею в раскалённый песок. Его ждёт расклевание парящими в небе стервятниками. Но у парня ещё есть шанс спастись. Негодяй неравнодушен к орнитологии, так что, если Джо вспомнит некоторые характерные свойства «птичек», парящих над ним, он будет отпущен. Помогите Джо спастись. 

Ответьте на следующие вопросы.
\begin{enumerate}
	\item Какая основная практическая польза от «птичек», как их используют?
	\item Назовите две любые характеристики «птичек»
\end{enumerate} 
\newpage
    \item «Пуассоновские трупы» (20 баксов)
    
    Стервятник Джонни сидит на ветке высохшего дерева посреди пустыни и ожидает умирающих от жажды путников. Время, прошедшее между двумя путниками, умершими в поле зрения Джонни, распределено экспоненциально с параметром $ \lambda = 3 $ (величина Y). Также стервятник на глаз прикидывает, сколько весит каждый новый труп, и как долго можно им кормиться.  Вес трупа в килограммах - равномерно распределённая случайная величина от 40 до 80 (величина X). Стервятник убеждён, что эти величины независимы.
    
    \begin{enumerate}
	\item Помогите стервятнику вычислить функцию плотности совместного распределения частоты появления трупов в поле зрения Джонни, и их веса.
	
	\item Джонни решил, что также важными показателями является отношение массы ступа к времени между трупами (грубо говоря, масса пищи на единицу времени), а также их сумма. Он решил перейти к этим новым единицам: $ R = X + Y, S = \frac{X}{Y} $. Помогите Джонни получить совместную функцию плотности двух новых величин.
    \end{enumerate}
    
    \item «Золото дураков» (10 баксов)
    
    В разгар золотой лихорадки группа старателей ищет место для новой разработки. Их предводитель Вилли Хромая Лошадь предполагает, что в пойме ближайшей реки есть золото. они начинают промывать песок. Время между попаданиями золотых самородков в часах распределено экспоненциально с параметром $\lambda = 3$. Если время до нахождения первого самородка превысит один час, то Вилли придётся искать новое место. Посчитайте примерно с помощью дифференциальной формы вероятность того, что за первые полчаса не попадётся ни одного самородка, но Вилли останется добывать золото на этом месте.
    
    
    \item  «Длительный приход» (10 баксов)
    
    Сутулая Собака, вождь индейского племени, курит трубку мира. В зависимости от табака длительность общения с духами предков может быть различной. Это случайная величина, имеющая следующее распределение:
    
    \[  
    f_{\xi}(x) =
    \begin{cases}
     \frac{x^3}{4}, x \in [0,2]\\
     0 , \text{иначе}
    \end{cases}
    \]
    
    Помогите вождю выявить плотность распределения логарифма по основанию 2 времени общения с духами. Найдите приблизительно с помощью дифференциальных форм вероятность того, что логарифм общения с духами будет лежать в промежутке $[0.5, 1]$
    
    
    \item «Таинственный ребус» (15 баксов)
    
    Шаман Мудрая Сова набрёл на таинственную пещеру в каньоне. Внутри неё стоит каменная дверь, за которой, безусловно, таится древнее и могущественное знание. Для того, чтобы открыть дверь, Мудрой Сове необходимо доказать, что он достоин, и начертить на песке максимально упрощённое выражение следующего вида:
    
    \[ d\alpha \wedge d\beta \] 
    
    \[ \alpha = \frac{xy}{x+y}, \beta = \frac{x^2}{\ln y} \]
    
    
    \item «Аллея висельников» (15 баксов)
    Количество трупов разбойников (X), развешенных на деревьях вдоль дороги, а также количество стервятников(Y), на эти трупы позарившихся — случайные величины, имеющее совместное распределение следующего вида:
    
    \[
f(y)=\left\{\begin{array}{l}{(x-1)(y-1), x \in[1 ; 2], y \in[1 ; 3]} \\ {0, \text { иначе }}\end{array}\right.
\]

Уровень устрашения разбойников задаётся величиной $S = 2X + Y$. Уровень устрашения индейцев - величиной $R = \frac{Y}{X}$. Найдите функцию плотности совместного устрашения. 
    
\end{enumerate}    
    
    
    \item Комбинаторика
%Коля
\begin{enumerate}
    \item «Плохой салун» (10 баксов)
    
    Как-то раз группа из $8$ ковбоев залетает в салун чтобы поесть. Но хозяин салуна совсем обленился и обслуживает очень долго, от чего образуются гигантские очереди. Сколькими способами ковбои могут занять очередь друг за другом, если Ли и Чарльз хотят стоять рядом, а бедный Джон настолько хочет есть, что ни при каких обстоятельствах не будет последним (иначе он всех перестреляет)? 
    % 10 баксов дуэль
    \item «Ночное дежурство» (10 баксов)
    
    Ковбои часто посылались на дежурство в отдаленные военные базы (которые представляли из себя часто просто дом с небольшим двориком), на защиту истинно американских рубежей от индейцев. Сколькими способами в течение $5$ дней можно выбирать на дежурство по $4$ ковбоя из армейской группы в $20$ человек так, чтобы каждый день состав дежурных был разным?
    % 10 баксов
    \item «Ковбойская игра» (20 баксов)
    
    Ковбой Арнольд и его $9$ друзей принялись играть в следующую игру: каждый бросает игральную кость. Игрок получает пинту виски, если он выбросил число очков, которое не удалось выбросить никому больше.
    
    1) Какова вероятность того, что Арнольд дорвётся до виски?
    
    2) Какова вероятность того, что хоть кто-то сегодня напьётся? 
    %20 баксов
    \item «Справедливые шерифы» (10 баксов)
    
    Команде шерифов округа Линкольн из $2$-х человек дали задание уничтожить банду грабителей из $10$ человек. Шерифы - люди честные и независимые, поэтому хотят распределить жертв поровну. Так, каждый из шерифов независимо от другого выбирает для себя $5$ грабителей, которых он прихлопнет. Какова вероятность, что их выбор не пересечется?
    %10 баксов дуэль
    \item «Типичная перестрелка» (15 баксов)
    
    Сидят в салуне $10$ мексиканцев, $10$ шерифов и $10$ индейцев. К ним вламывается «Малыш» Билли с острым желанием убить каких-нибудь $10$ человек. Какова вероятность, что среди выбранных им $10$ человек не будет как минимум одной категории посетителей салуна?
    %15 баксов
    \item «Охотник за головами выходит на охоту» (10 баксов)
    
    Охотнику за головами, Тому, дали большой заказ - пристрелить $20$ участников банды Wild Bunch и $10$ шерифов округа Техас. Чтобы не вызвать переполох в городе, за сегодня он должен убить $7$ человек, но среди них должно быть не менее $1$-ого шерифа. Сколько у Тома способов выполнить свою работу?
    %10 баксов дуэль
\end{enumerate}    
    
    \item Энтропия
%Алексай    
\begin{enumerate}
    \item  «Чем длиннее имя, тем короче прозвище» (10 баксов)
    
    Эстебан Хулио Рикардо Монтойя де ля Росса Рамирес не является ковбоем, но очень хочет им стать. Он очень гордится своим полным именем, но в среде ковбоев длина имен не приветствуется, так что Эстебан решил придумать псевдоним. Псевдонимы у ковбоев должны состоять из прозвища и имени (короткого). Вероятность того, что Эстебан выберет прозвище \textbf{«Непобедимый»} - 0.4, \textbf{«Неотразимый»} - 0.2, \textbf{«Непревзойденный»} - 0.4. Вероятность того, что он оставит имя Эстебан - 0.3, Хулио - 0.3, Рикардо - 0.4. Найлите энтропию его прозвища.
    %10 баксов дуэль
    \item «К.К. Б.Б.» (10 баксов)
    
    Косорукий Боб и Косоглазый Билли стреляют по мишеням. Они поспорили на 42 фунта честно сворованного золота, кто точнее сделает 1 000 000 выстрелов. Спор честный, т.к. оба имеют одинаковые вероятности поражения различных частей мишени: «край» - 0.1, «середина» - 0.1, «яблочко» - 0.1. Они решили, что тратить силы на выстрелы куда важнее, чем на запись результатов, поэтому предварительно изобрели шифровку, с которой результаты должны записываться наиболее коротким способом. Известно, что шифровка состоит из первых символов их имен и фамилий - «К» и «Б». Сколько в среднем символов в шифровке уйдет запись результата одного выстрела?
    %10 баксов дуэль
    
    \newpage
    
    \item «Всего 5 пальцев» (10 баксов)
    Приезжая Бездельница Наташа сидит в баре считает бутылки виски на соседних столиках и вырезает результаты по порядку ножиком прямо на деревянном столе. У себя в уме она делит виски на 10 типов, начиная с дешевого пойла и заканчивая элитными нектарами. Наташа насчитала по 11 бутылок с первого по девятый типы и всего одну десятого типа. Она предполагает, что нахождение бутылок любого типа не связано ни друг с другом, ни с бутылками другого типа, а истинные вероятности их нахождения она выяснила сегодня. Наташа уже уходит из бара и хочет разнообразить свою небогатую событиями жизнь завтра. Бездельница посмотрела себе на руку, увидела пять пальцев и решила, что завтра будет записывать результаты всего пятью символами. Сколько, по ее ожиданиям, символов уйдет на запись одного типа бутылки, если Наташа (как ни странно) шифрует оптимально? 
    % 10 баксов
    
    \item «Трусиха Джейн» (20 баксов)
    Трусливая Джейн попала в перестрелку. Она тихо сидит в укрытии, дрожит, ждет подкрепления, которое должно подойти в течение 10 минут. Так уж сложилось, что подкрепление не придет.
    
    Джейн прозвали Трусливой потому, что она не умеет выпускать по одной пуле из револьвера - начав стрелять, она обязательно истратит всю обойму. В ближайшие 10 минут Джейн равновероятно начнет стрелять. Когда 10 минут истекут, Джейн начнет нервничать сильнее, и за следующие 5 минут с вдвое большей вероятностью (за одинаковый промежуток времени) выпустит всю обойму. Джейн подобрала себе вероятности так, чтобы обязательно начать стрелять в течение этих 15 минут. Найдите энтропию времени расстрела обоймы Джейн.  
    %20 баксов
    
    \item «Покоряя космос» (10 баксов)
    Будущий капитан межпланетного корабля класса светлячок (корабль несет гордое название «Серенити») Малькольм «Мэл» Рейнольдс пытается купить этот самый корабль у жадного Джо. Они решили одновременно назвать круглую цену, которую считают справедливой. Мэл равновероятно назвает 5, 6 или 7 млрд. условных единиц; торговец с вероятностью $\frac{2}{3}$ говорит 5 млрд. у.е. и равновероятно называет 6 или 7 млрд. Найдите энтропию разницы цен уважаемых.
    %10 баксов дуэль
    \item «Слишком мало патронов» (15 баксов)
    Амбидекстер Сэм пометил все свои патроны крестиками и ноликами в отношении 1 к 3. Сэм случайным образом выбирает патроны и вставляет в барабан. Найдите вероятность того, что энтропия внутри револьвера будет максимально возможной, если всего у Сэма 20 патронов, а барабан рассчитан на шесть.  
    %15 баксов
\end{enumerate}    

\newpage
    
    \item Популярные непрерывные распределения
    
    


%Илья
\begin{enumerate}
    \item Местный врач Иствуд оценил, что длительность беременности на самом большом ранчо Матленда (в днях) распределена примерно норамльно с матожиданием равным 270 и дисперсией равной 100. Однако медик был вынужден уехать из деревни на некоторое время. Период его отсутствия начался за 290 дней до рождения ребенка и закончился за 240 дней также до его рождения. Какова вероятность того, что беременность была либо слишком долгой, либо слишком короткой?
    \item После успешного рейда банды свободоборцев, Датч Ван Дер Линду удалость украсть самую быструю повозку города Блэкуотера. Предположим, что количество миль, которые повозка может проехать, пока не сломается, распределено экспоненциально со средним значением 10 000 миль. Если ковбой захочет проехать 5000 миль, с какой вероятностью ему удасться вернуться домой живым?
    \item Банда изгнанников обнаружила золотые рудники, вход в которую имееют круглую форму. Однако рудники уже давным-давно никто не находил, поэтому дверь засыпало грязью и пылью. Изгнанники оказались слишком жадными, чтобы вначале очистить дверь, поэтому решили действовать сразу: зная, что где-то в двери находится отверстие для ключа, банда пытается с силой вставить отмычку в дверь. После пятого неудачного тычка отмычка ломается. Найдите вероястность того, что им удасться проникнуть в рудники.
    \item Великий вождь древних племен завещал потомкам несметные богатства, если те правльно посчитают вероятность, с которой можно найти дисперсию случайной величины, имеющей нормальное распределение с параметрами 5 и 7. Определите, от чего зависит вероятность заполучить несметные богатства. Найдите эту вероятность.
    \item Очень занятой, но очень ответственный шериф МакДаг набрал лучших стрелков со всего города в свою команду и обещал назначить им дело в течение одного ковбойского цикла (стыдно не знать, что один ковбойский цикл равен восемнадцати часам). 
    Сам шериф приходит к себе один раз за цикл в любое время и уходит через девять часов, если цикл не закончится раньше.
    Не менее занятые, но чуть менее ответственные стрелки не уделили особого внимания словам шерифа и решили приходить каждые\footnote{Обратите внимание, что стрелки впервые могут прийти как в первый, так и в восемнадцатый час цикла; вообще не прийти они не могут – ведь их уволят.} шесть часов на два часа.
    И шериф, и команда стрелков действую независимо друг от друга.
    Найдите вероятность того, что шерифу все-таки удастся пересечься со стрелками и назначить команде задание в рамках одного цикла.
    Примечание: задача решается с использованием инструментов непрерывных распределений, но предполагает также решение через классическое определение вероятности.
\end{enumerate}    
    
    \item ЦПТ
%Шамиль
\begin{enumerate}

    \item По просьбе шерифа округа Деликадос, группа из 120-ти ковбоев выдвинулась в сторону деревни, где по словам местных пастухов скрывается банда разбойника Фернандо. По данным ковбойской разведки (хе-хе) банда состоит из 60 человек. Малыш Билли, самый опытный из ковбоев и знаток военной стратегии, предположил, что битва с бандитами распадется на 60 мелких боев, в каждом из которых будут воевать два ковбоя и один ковбой. Пройдя через сотни сражений, он знает, что вероятность того, что в такой мини-битве умрет один ковбой или оба, равна 0.5 и 0.2, соответственно (мини-битвы независимы, так как наши ковбои конкурируют между собой и не помогают друг-другу). К сожалению, из-за незнания теории вероятностей Билли не может назвать вероятность, с которой потери составят меньше 50 человек. Помогите ему ))
    \newpage
    \item Капиталист Карнеги всю свою жизнь зарабатывал, продавая лучшее мясо для самых состоятельных людей страны, за что получил кличку «Король стейков». Но в этом месяце индейский кооператив «Озеро», все время доставлявший скот в разделочные цеха магната, обанкротился, создав кучу проблем нашему бизнесмену, ведь теперь ему придется подсчитать все риски перегона скота. По полученной им информации перегон скота от Техаса до Бостона проводят 50 ковбоев, стадо каждого составляет 50 голов. Количество погибших голов распределено по закону Пуассона с параметром 7. Американские предприниматели не любят риски, поэтому Карнеги решил обратиться к местным страховщикам,  которые расчитывают цену страховки A по формуле P(X < A) = 60\%. Ваша задача узнать цену страховой сделки (местные рынки в данный момент оценивали цену одного быка в 1\$) 
    
    
     \item Пусть количество бочек пива, выпитого в местном салуне за один день — случайная величина $X$, распределенная экспоненциально с параметром $\lambda=3$. Пусть $Y_1, \ldots, Y_n$ независимы, а $Y=e^X$. Требуется найти распределение $\overline{Y}$ при $n \rightarrow \infty$.
    \newpage
    
    \item Для улучшения своих оборонительных навыков индейцы племени Папайа тренируются, стреляя по орехам, растущим на дереве. Так как орехи маленькие а луки индейцев давно устарели, вероятность попадания равна 0.03. Но если сбить 5 орехов, глава племени дарит «снайперу» дополнительное перо в головной убор. Наш герой, гордый индеец Мимокосо, всегда мечтавший о развесистом шлеме из 10 перьев, но обладавший только кожаной повязкой (перьев нет), решил попытать счастье. Одним утром он сказал жене, что пустит тысячу стрел для достижения своей цели и ушел. Его жена, шаманка, говорящая с богами вероятности, решила к ним обратиться и, если вероятность того, что сбудется мечта ее мужа, больше 0.35, она приготовит ему индейские сладости, а иначе купит ему бутылку огненой воды. Скажите, какую вероятность подсказали боги жене, и что ждет нашего «снайпера», праздник сладостей или недельный индейский запой? 
    
    
    \item Джек, владелец местного салуна, разработал новый рецепт текиллы. Для того, чтобы определить, внедрять его в барный лист или нет, Джек решил давать новый напиток клиентам и узнавать, понравилось им или нет. Так как посетители салуна были в основном индейцы, которые совсем не разбирались в алкоголе, называя различные виды огненной водой, ответ они давали случайный (вероятность положительного и отрицательного ответа были равны). Какова вероятность того, что после 1000 опрошенных «Сыновей Сокола» Джек добавит новую текиллу в меню, если общественным одобрением он считает как минимум 520 положительных отзывов.
    
    
    \item Пусть в нашем идеальном ковбойском мире все кофликты решаются не через смертельные престрелки под известную всеми нами известную мелодию, а подкидыванием монеты 101 раз, победителем является тот, кому повезло не менее 51 раза, а проигравший отправляется в тюрьму. Супер Джо, нашему главному герою, по наследству (от прадеда - деду, от деда - отцу, от отца - его брату, а после храброй смерти брата - ему) досталась прекрасная монетка, которая с вероятностью 0.(6) показывает необходимую Супер Джо сторону. Каковы вероятность того, что Супер джо выберется победителем из конфликта?
    
    
    
\end{enumerate}    

\end{enumerate}
\section{Решение задач праздника}
\subsection{Тур 1}
\begin{enumerate}
    \item Комбинаторика

\begin{enumerate}
    \item «Шаловливый циклоп»
    
    $20 \cdot 19 \cdot 18$
    
    \item «Вилкой в глаз или в кости раз?»
    
    \begin{align*}
        \text{Всего сумм} = 6^{2}\\
        \text{Добротные события} = 3\\
        \text{Вероятность победы Матвея} = \frac{3}{36}
    \end{align*} 
    
    \item «Ограбление поезда»
    
    Т.к. все пассажиры должны ехать в разных вагонах, требуется
    отобрать 4 вагона из 9 с учетом порядка (вагоны отличаются №), эти выборки – размещения из $n$ различных элементов по $m$ элементов, где $n=9$, $m=4$. Число таких размещений находим по формуле: 
    \begin{align*}
        & A_{n}^{m} = n \cdot (n - 1) \cdot \ldots \cdot (n - m + 1)\\
        & A_{9}^{4} = 9 \cdot 8 \cdot 7 \cdot 6 = 3024
    \end{align*} 
    
    \item «Домино»
    
    \begin{align*}
        12^{2} - \text{Всего способов поставить 2 доминошки}\\
        4^{2} - \text{Случаев совпадения домино}\\
        \text{Таким образом мы имеем вероятность:}\\
        p = \frac{4^{2}}{12^{2}} = \frac{1}{9}
    \end{align*}
    
    
    \item «Пушка Моти»
    
    \begin{align*}
        \text{Всего чисел} = \frac{n(n + 1)}{2}\\
        \text{Кол-во i-ого числа} = i\\
        \text{Вероятность выпадания i-ого числа} = \frac{i}{\frac{n(n + 1)}{2}}
    \end{align*} 
    
    
    \item «Плотские забавы ковбоев»
    
    Позиция — место маленького кубика в большом кубе, положение - ориентация граней кубика. У каждого кубика 8 возможных позиций и 24 возможных положения (фиксируем одну из 6 граней - остается еще 4 возможных положения, т.е всего $6\cdot4$). В правильном кубе у каждого кубика 8 возможных позиций и 3 возможных положения (фиксируем одну из трех позолоченных граней, остается только одно правильное положение, всего $3\cdot1$). Искомая вероятность равна $\frac{8!\cdot3^8}{8!\cdot24^8}$ 
    \end{enumerate}    
    
    \item Условные вероятности 

\begin{enumerate}
    \item 0.(72)
    \item 0.84
    \item 0.(027)
    \item 0.775
    \item 0.5
    \item 0.75
\end{enumerate}    
    
    \item Пределы

\begin{enumerate}
    \item Вероятость не выстрелить холостым для $n$-ого барабана равна $(1-\frac{1}{n})$. Для $n \to \infty$ находим \[\lim_{n \to \inf}{(1-\frac{1}{n})^{n}} = e^{-1}\]
    \item Имеем дело с обычным биномиальным распределением для $X$, соответсвенно $E(X) = 16$. 
    \newline Далее по ЗБЧ \[\plim_{n \to \infty}{\bar X} = E(X) = 16\]
    \item $E(X) = 0.1.$ \[\plim_{n \to \infty}{\frac{11}{1+\bar X}} = \frac{11}{1+\plim_{n \ to \infty}{\frac{\sum X}{n}} }= \frac{11}{1+\frac{0.1}{n}} = 10\]
    \item Каждый пятый шаг данной экспедиции на базу, соответственно \[\plim_{n \to \infty}{X_{base}}=0.2\] \newline Попадание на каждый кусок территории равновероятно, соответственно \[\plim_{n \to \infty}{X^{(n)}} = \frac{(1-0.2)}{n} = \frac{4}{5n}\]
    \item Опытный шериф будет приглашать бандита с веротяностью \[\lim{n \to \infty}{\frac{3n}{4(n+1)}} = 0.75\] \newline Соответственно, вероятность три раза промазать будет равна $(1-0.75)^3 = \frac{1}{64}$
    \item $plim X_n = 0$, $plim \E(X_n) \to \infty$
\end{enumerate}    
    
    \item Нормальное распределение 
\begin{enumerate}
    \item $\E [\xi_i]=\frac{1}{n}\sum\limits_{i=1}^n \mu_i$ 
    \item $\int\limits_{\R} xdx \int\limits_{\R} \exp\left\{- \frac{(x-\mu)^2}{2\sigma^2} \right\} \exp\left\{- \frac{x^2}{2}\right\} d\mu$
    \item 1/2
    \item $\frac{1}{2} \log(2\pi e\sigma^2)$
    \item $\exp \left\{\mu t+\frac{\sigma^2t^2}{2} \right\}$
    \item $\frac{x}{\sigma^2} \exp\left\{-\frac{x^2}{2\sigma^2} \right\}$
\end{enumerate}  
    
    \item Геометрия

\begin{enumerate}
    \item Вероятность того, что ковбой застрял в городе МенЭкз составляет 0,2.
    \item $\E(X) = 2$
    
    $\E(3X) = 6$
    
    По неравенстрву Маркова:
    \[
    P(x > a) \leq \frac{E(X)}{a}
    \]
    \[
    P(x > 18) \leq \frac{1}{3}
    \]
    \item $\P(X)= \frac{0.4 + 0.4}{2} = 0.4 $
    \item Ковбой есть 60 минут.
    \item $\P(X)= \frac{2 \cdot l}{\pi \cdot a} = 0.38 $
    \item $\P(X) = 0.6$
\end{enumerate}    

    \item Марков и Чебышев

\begin{enumerate}
    \item  $E(X) = 5$, где Х - количество выстрелов до первого попадания. 
    Воспользуемся неравенством маркова:
    
    \[
    P(x > a) \leq \frac{E(X)}{a}
    \]
    \[
    P(x > 12) \leq \frac{5}{12} = 0.42
    \]

    \item 
    Пусть X - случайная величина, характеризующая размер бутылки, по которой попал Джо. Если X = 5, значит Джо попал в бутылку размером 5 дюймов. Если X = 2, то Джо мог бы попасть в бутылку размером 2 дюйма, есл
    \[
    \]
    
    \item \[ P(|X - a| < A) \geq 1 - \frac{Var(x)}{A^2} = 1 - \frac{Var(X)}{1000^2} = 0.95
    \]
    \[
    Var(X) \geq 50.000
    \]
    
    \item \[E(X) = a = 100 \cdot 0.75 = 75
    \]
    \[ |X - a| \leq 5
    \]
    \[ Var(X) = npq = 100 \cdot 0.75 \cdot 0.25 = 18.75
    \]
    Используя неравенство Чебышёва получим:
    \[ P(|X - a| < A) \geq 1 - \frac{Var(x)}{A^2} = 1 - \frac{18.75}{25} = 0.25
    \]
    
    \item
     \[
    P(x > a) \leq \frac{E(X)}{a}
    \]
    \[
    P(x > 60) \leq \frac{50}{60} = 0.83
    \]
    \item 
    \[
    a = E(X) = np = 300 \cdot 0.3 = 100
    \]
    
    \[
    Var(X) = npq = 300 \cdot 0.3 \cdot 0.7 = 70
    \]
    
    \[ 
    P(|X - a| < A) \geq 1 - \frac{Var(x)}{A^2} 
    \]
    
    \[
    P(|X - 100| < 50) \geq 1 - \frac{70}{50^2} = 0.972 
    \]
\end{enumerate}    

    \item О-малые

\begin{enumerate}
    \item «Дебош в салуне»
    
    Пусть $Y_1, Y_2, Y_3$ — время до срабатывания динамитов в порядке возрастания. Так как нам сказано, что салун развалится сразу после второго взрыва, необходимо, чтобы второй взрыв случился после закрытия салуна. Это происходит в том случае, когда $Y_2 > 5$, потому что салун закроется через 5 часов. Значит нам надо найти вероятность $\P(Y_2 > 5).$
    
    Для начала найдем функцию плотности случайной величины $Y_2$ при помощи определения через вероятность попадания случайной величины в малый интервал — $\P(Y_2 \in [t, t + \Delta])$.
    
    \[ 
    \P(Y_2 \in [t, t + \Delta]) = C_3^1 \cdot \frac{\Delta }{6}  \cdot C_2^1 \frac{t-2}{6} \cdot \frac{8-t}{6} + o(\Delta)
    \] 
    
    Почему так?
    
    1) Одна из трёх величин должна попасть в отрезок $[t, t + \Delta]$.
    
    2) Одна величина должна оказаться меньше $t$.
    
    3) Одна величина должна оказаться больше $t$.
    
    Таким образом, можем найти $ \P(Y_2 \in [t, t + \Delta])$:
    
    \[ 
    \int_5^8 \frac{(t - 2)(8-t)}{36}\,dt = \int_5^8 \frac{10t-t^2-16}{36}\,dt = \left. \frac{5t^2}{36} - \frac{t^3}{108} - \frac{16t}{36} \right|_5^8 = \left. \frac{15t^2-t^3-48t}{108} \right|_5^8 
    \]
    \[ 
    \frac{ 15 \cdot 64 - 8 \cdot 64 - 48 \cdot 8 - 15 \cdot 25 + 125 + 48 \cdot 5}{108} =  
   \frac{54}{108} = 0,5
    \]
    \item «Резервация в поисках агавы»
    
    Нам сказано, что есть три мудреца, которые делают агаву. Нам надо найти функцию плотности для $Y_1$ по определению — $ \P(Y_1 \in [t, t + \Delta])$.
    
    \[
    \P(Y_1 \in [t, t + \Delta]) =  C_3^1 \cdot \frac{\Delta}{6} \cdot \frac{(6-t)^2}{6^2} + o(\Delta) = \Delta \cdot \frac{(t-6)^2}{2 \cdot 36}
     +  o(\Delta) = \Delta \cdot \frac{(t-6)^2}{72}
     +  o(\Delta) \]
    
    \item «Перегон скота»
    
    Обозначим за $M_t$ количество коров справа (в стороне резервации), а за $L_t$ — количество коров слева (в стороне от резервации). Тогда $M_0 = 100, L_0 = 150$.
    
    За малый интервал времени возможны следующие события: прошла корова слева, прошла корова справа, прошли коровы справа и слева, не прошли ни одна корова.
    
    Вероятность того, что прошла только корова слева (в сторону резервации):
    \[
    (5\Delta + o(\Delta))(1 - 3\Delta + o(\Delta)) = 5\Delta - 15\Delta^2 + o(\Delta) = 5\Delta + o(\Delta)
    \]
    Вероятность того, что прошла только корова справа (в сторону  от резервации):
    \[
    (3\Delta + o(\Delta))(1 - 5\Delta + o(\Delta)) = 3\Delta - 15\Delta^2 + o(\Delta) = 3\Delta + o(\Delta)
    \]
    Вероятность того, что прошли две коровы (справа и слева):
    \[
    (5\Delta + o(\Delta))(3\Delta + o(\Delta)) = 15\Delta^2 + o(\Delta) =  o(\Delta) 
    \]
    Вероятность того, что не прошла ни одна корова (справа или слева):
    \[
    (1 - 5\Delta + o(\Delta))(1 - 3\Delta + o(\Delta)) = 1 - 3\Delta + o(\Delta) - 5\Delta + 15 \Delta^2 + o(\Delta) = 1 - 8\Delta + o(\Delta)
    \]
    В таком случае можем описать динамику $M$ и $L$ за малый интервал времени:
    \[
    M_{t+\Delta} = (M_t + 1) \cdot (5\Delta + o(\Delta)) + (M_t - 1) \cdot (3\Delta + o(\Delta)) + M_t \cdot (1 - 8\Delta + o(\Delta)) 
    \]
    \[
    L_{t+\Delta} = (L_t - 1) \cdot (5\Delta + o(\Delta)) + (L_t + 1) \cdot (3\Delta + o(\Delta)) + L_t \cdot (1 - 8\Delta + o(\Delta)) 
    \]
    Отсюда получим:
    \[
    M_{t+\Delta} = M_t + 2\Delta + o(\Delta) \ \ \ \ \ \ \ L_{t+\Delta} = L_t - 2\Delta + o(\Delta)
    \]
    \item «Кантри - наше всё»
    Из условия мы понимаем, что если обозначить за $Y$ момент окончания кантри-представления, а за $X$ — момент ухода Джо в закат из-за неприятия кантри, можно составить следующие условия:
    
    \[ 
    \P(Y \in [t, t + \Delta]) = \lambda_y \Delta + o(\Delta)
    \]
    \[
    \P(X \in [t, t + \Delta]) = \lambda_x \Delta + o(\Delta)
    \]
    
    Нам сказано, что вероятность того, что Джо не уйдёт в закат по причине того, что кантри-представление закончилось, равно 0.5, то есть $\P(Y < X) = 0.5$. Нам необходимо найти функции плотности для $X$ \ и \ $Y$, чтобы найти при заданном условие соотношение между $\lambda_y$ \ и \ $\lambda_x$.
    
    Обозначим за $ \P_t^x$ вероятность того, что к моменту $t$ кантри-представление не закончилось, а за $\P_t^y$ — вероятность того, что к моменту $t$ Слепой Джо не ушёл в закат.
    Тогда:
    
    \[ 
    \P_{t+\Delta}^x = \P_t^x \cdot (1 - \lambda_x  \Delta + o(\Delta)) = \P_t^x - \P_t^x \cdot \lambda_x \Delta + o(\Delta)
    \]
    \[
    \P_{t+\Delta}^y = \P_t^y \cdot (1 - \lambda_y \Delta + o(\Delta)) = \P_t^y -\P_t^y \cdot \lambda_y \Delta + o(\Delta)
    \]
    
    Тогда получим следующее:
    
    \[
    \P_{t+\Delta}^x - \P_t^x = - \P_t^x \cdot \lambda_x \Delta + o(\Delta)
    \]
    \[ \frac{\P_{t+\Delta}^x - \P_t^x}{\Delta} = - \P_t^x \cdot \lambda_x + \frac{o(\Delta)}{\Delta}\]
    
    Если мы устремим $\Delta$ \ к нулю, то получим: 
    
    \[
    (\P_t^x)' = - \P_t^x \cdot \lambda_x \longrightarrow \ \P_t^x = e^{-\lambda_x\cdot t}
    \]
    Аналогично для $Y$: $ \P_t^y = e^{-\lambda_y\cdot t}$ — вероятность того, что момент ухода Джо в закат меньше, чем $t$
    
    Значит, $\P(Y \leq t) = 1 - e^{-\lambda_y\cdot t}$ \ и \ $\P(X \leq t) = 1 - e^{-\lambda_x\cdot t}$
    
    Несложно заметить, что при таких свойствах $X$ \ и \ $Y$ — это случайные величины, имеющие экспоненциальное распределение с параметрами $\lambda_x$ \ и \ $\lambda_y$ соответственно.
    Тогда функции плотности будут равны соответственно $\lambda_x e^{-\lambda_x t}$ и  $\lambda_y e^{- \lambda_y t}$ для положительных $t$. 
    
    Если $X$ и$Y$ — независимые случайные величины, то их совместная функция плотности имеет вид: 
    \[
    f_{X, Y}(x, y) = \begin{cases}
    \lambda_x e^{-\lambda_x x} \cdot \lambda_y e^{- \lambda_y y}, \ x, y \geq 0 \\
    0, \ \text{иначе}
    \end{cases}
    \]
    
    Судя по условию задачи нам дано $\P(Y < X)$:
    \[
    \P(Y < X) = \int_0^{\infty} \int_0^x \lambda_x e^{-\lambda_x x} \cdot \lambda_y e^{- \lambda_y y}dy dx = \int_0^{\infty} \lambda_x e^{-\lambda_x x} (1 - e^{- \lambda_y x})dx =
    \left|
    -e^{-\lambda_x x} + \frac{\lambda_x}{\lambda_y} e^{-\lambda_y x} \right|_{0}^{\infty} = 
    \]
    \[
    = -0 + 1 + 0 - \frac{\lambda_x}{\lambda_y} = \frac{1}{2} \longrightarrow \ \frac{\lambda_x}{\lambda_y} = \frac{1}{2} 
    \]
    \item «Агава найдена»
    
    Нам сказано, что есть пять умельцев-текилодельцев. Если нам надо найти вероятность того, что максимальный объём текилы больше 4 литров.
    
    Это значит, что нам надо найти вероятность $\P(Y_5 > 4)$. 
    
    Для начала найдём функцию плотности $Y_5$ по определению — $ \P(Y_5 \in [t, t + \Delta])$.
    
    \[
    \P(Y_5 \in [t, t + \Delta]) = C_5^1\cdot  \frac{\Delta}{5}\cdot \left(\frac{t}{5}\right)^4
     + o(\Delta) = \Delta \cdot \frac{t^4}{5^4} + o(\Delta) = \Delta \cdot \frac{t^4}{625} + o(\Delta)
    \]
    \[ 
    \P(Y_5 > 4) = \int_4^5 \frac{t^4}{625} dt = \left. \frac{t^5}{3025} \right|_4^5 = \frac{3125 - 1024}{3125} = \frac{2101}{3125}
    \]
    \item «Мир Дикого Запада»
    
    Обозначим за $\P(r_t = k)$ вероятность того, что в момент $t$ Слепой Джо находится на сотке под номером $k$. За малый интервал времени он может пойти вперёд, на сотку с номером $k+1$, пойти назад — к сотке $k-1$, или остаться на месте — на сотке $k$. Найдём вероятность $\P(r_{t+\Delta} = k):$
    \[
    \P(r_{t+\Delta} = k) = \P(r_t = k-1)\cdot (\lambda_f \Delta + o(\Delta)) + 
    \P(r_t = k+1)\cdot (\lambda_b \Delta + o(\Delta)) + \P(r_t = k)\cdot (1-(\lambda_f + \lambda_b)\Delta + o(\Delta)) \]
    
    Мы знаем, что вероятность нахождения на какой-либо сотке не зависит от времени, поэтому сделаем переобозначение: $\P(r_t = k) = \P_k$.
    
    Тогда получаем:
    \[
    \P_k = \P_{k-1}\cdot \lambda_f \Delta + \P_{k+1} \cdot \lambda_b \Delta + \P_k \cdot (1-(\lambda_f + \lambda_b)\Delta) + o(\Delta)
    \]
    
    \[
    \P_k (\lambda_f + \lambda_b)\Delta = \P_{k-1}\cdot \lambda_f \Delta + \P_{k+1} \cdot \lambda_b \Delta  + o(\Delta)
    \]
    \[
    \P_k (\lambda_f + \lambda_b) = \P_{k-1}\cdot \lambda_f + \P_{k+1} \cdot \lambda_b + \frac{o(\Delta)}{\Delta}
    \]
    
    Если мы устремим $\Delta$ к нулю, то получим:
    
    \[
    \P_k (\lambda_f + \lambda_b) = \P_{k-1}\cdot \lambda_f + \P_{k+1} \cdot \lambda_b
    \]
    
    Однако для первой и восьмой сотки получим немного другие выражения:
    
    \[
    \P_1 = \P_2 \lambda_b \Delta + \P_1 (1-\lambda_f\Delta) + o(\Delta)
    \]
    \[
    \P_8 = \P_7 \lambda_f \Delta + \P_8(1 - \lambda_b \Delta) + o(\Delta)
    \]
    В итоге получим:
    \[
    \P_1 = \frac{\P_2 \lambda_b}{\lambda_f} \ \text{или} \ \P_2 = \frac{\P_1 \lambda_f}{\lambda_b} \ \ \text{и} \ \P_8 = \frac{\P_7 \lambda_f}{\lambda_b}
    \]
    \[
    \P_2 (\lambda_f + \lambda_b) = \P_1 \lambda_f + \P_3 \lambda_b \longrightarrow \P_2 (\lambda_f + \lambda_b) = \P_2 \lambda_b + \P_3 \lambda_b \longrightarrow \P_3 = \frac{\P_2 \lambda_f}{\lambda_b}
    \]
    
    Получаем, что $\P_{k+1} = \frac{\P_k \lambda_f}{\lambda_b}$ или $\P_k = \P_1 \cdot \left( \frac{\lambda_f}{\lambda_b}\right)^k$
    
    Так как Слепой Джо, укрывающийся в грядках агавы от Иниго Монтойя, может находиться только с первой по восьмую сотку, то сумма этих вероятностей должна давать 1:
    
    \[
    \P_1 \left(1 + \ldots + \left( \frac{\lambda_f}{\lambda_b}\right)^8 \right) = 1
    \]
    \[
    \P_1 \left( \frac{\left( \frac{\lambda_f}{\lambda_b}\right)^8 - 1}{\frac{\lambda_f}{\lambda_b} - 1}\right) = 1
    \]
    \[
    \P_1 \frac{\lambda_f^8 - \lambda_b^8}{\lambda_b^7 (\lambda_f - \lambda_b)} = 1 \longrightarrow \P_1 = \frac{\lambda_b^7 (\lambda_f - \lambda_b)}{\lambda_f^8 - \lambda_b^8}
    \] 
    \[
    \P_3 = \frac{\lambda_b^7 (\lambda_f - \lambda_b)}{\lambda_f^8 - \lambda_b^8} \cdot \left( \frac{\lambda_f}{\lambda_b}\right)^2 \longrightarrow
    \P_3 = \frac{\lambda_b^5 \lambda_f^2 (\lambda_f - \lambda_b)}{\lambda_f^8 - \lambda_b^8}
    \]
    \item Метод главных компонент
\end{enumerate}  
\begin{enumerate}
    \item $\E(x^2+y^2) = 2\E(x^2)$. Половина шагов (50) придется на ось $x$, причем каждый раз равновероятно либо вверх, либо вниз. Поэтому считаем матожидание расстояния у оси $x$, получится $50*(\frac{1}{2} + \frac{1}{2})$.
    \item $\frac{1}{5}*1 + \frac{4}{5}*\frac{1}{5}*2 + \frac{4}{5}*\frac{4}{5}*\frac{1}{5}*3 + \ldots = \frac{1}{5} \sum_{i = 1}^{\infty}i*\frac{4}{5}^{i-1}$
    \item 
    \item $\E(X) = \frac{2}{3}*\frac{1}{3}*1 + \frac{2}{3}*\frac{2}{3}*\frac{1}{3}*2 + \ldots = \frac{1}{3} \sum_{i = 1}^{\infty} i(\frac{2}{3})^i$
    $\E(X^2) = \frac{2}{3}*\frac{1}{3}*1 + \frac{2}{3}*\frac{2}{3}*\frac{1}{3}*2^2 + \ldots = \frac{1}{3} \sum_{i = 1}^{\infty} i^2(\frac{2}{3})^i$
    \item $0,9*0,9*0,9 + 0,9*0,1*0,2 + 0,1*0,2*0,9 + 0,1*0,8*0,2$
    \item 0,00676875
\end{enumerate}    

нормальное распределение
\begin{enumerate}
    \item \[ \E [\xi_i]=\frac{1}{n}\sum\limits_{i=1}^n \mu_i \]
    \item \[ \int\limits_{\R} xdx \int\limits_{\R} \exp\left\{- \cfrac{(x-\mu)^2}{2\sigma^2} \right\} \exp\left\{- \cfrac{x^2}{2}\right\} d\mu \]
    \item \[ \frac{1}{2} \]
    \item \[ \frac{1}{2} \log(2\pi e\sigma^2) \]
    \item \[ \exp \left\{\mu t+\frac{\sigma^2t^2}{2} \right\} \]
    \item \[ \frac{x}{\sigma^2} \exp\left\{-\frac{x^2}{2\sigma^2} \right\} \]
\end{enumerate}
    
\end{enumerate}
\subsection{Тур 2}
\begin{enumerate}
    \item Байес

\begin{enumerate}
    \item Пусть $T_1, \ldots, T_n$ — время, за которое $1, \ldots, n$ ослы перепрыгнут через каньон. Нужно найти\\
    $\P(T_2 \ge 1 \, | \, T_1 > 0.5, \, T_2 > 0.5, \, \ldots, \, T_n > 0.5)$. Из независимости ослов следует:
    \[
    \P(T_2 \ge 1 \, | \, T_2 > 0.5) = \frac{\P(T_2 \ge 1 \, \cap \, T_2 > 0.5)}{\P(T_2 > 0.5)} = \frac{\P(T_2 \ge 1)}{\P(T_2 > 0.5)}
    \]
    \[
    \P(T_2 \ge 1) = \int\limits_1^{+\infty} 2 e^{-2x} dx = e^{-2}
    \]
    \[
    \P(T_2 > 0.5) = \int\limits_{0.5}^{+\infty} 2 e^{-2x} dx = e^{-1}
    \]
    Получившаяся вероятность: $\P(T_2 \ge 1 \, | \, T_2 > 0.5) = e^{-1}$
    
    \item Пусть $X$ и $Y$ — длина и ширина поля. Нужно найти $\E(X \, | \, X < Y)$
    Совместная плотность $X$ и~$Y$:
    \[
    \begin{cases}
    f_{X,Y}(x,y) = \frac{1}{(40-20)\cdot(50-30)} = \frac{1}{400}, \, (x,y) \in [20,40] \times [30,50]\\
    0, \, \text{иначе}
    \end{cases}
    \]
    
    Найдем $\P(X<Y)$:
    \[
    \P(X<Y) = \int\limits_{20}^{30} \int\limits_{30}^{50} \frac{1}{400} dy dx + \int\limits_{30}^{40} \int\limits_{x}^{50} \frac{1}{400} dy dx = \int\limits_{20}^{30} \frac{20}{400} dx + \int\limits_{30}^{40} \frac{50-x}{400} dx = 
    \]
    \[
    = \frac{200}{400} + \frac{100x-x^2}{800} \Bigm|^{40}_{30} = \frac{1}{2} + \frac{300}{800} = 0.875
    \]
    Условная плотность:
    \[
    f_{XY \, | \, X < Y} =
    \begin{cases}
    \frac{f_{XY}(x,y)}{\P(X<Y)}, \, x < y \\
    0, \, x \ge y \\
    \end{cases}
    =
    \begin{cases}
    \frac{1}{400 \cdot 0.875}, \, x < y \\
    0, \, x \ge y \\
    \end{cases}
    =
    \begin{cases}
    \frac{1}{350}, \, x < y \\
    0, \, x \ge y \\
    \end{cases}
    \]
    Тогда можно и матожидание посчитать:
    \[
    \E(X \, | \, X < Y) = \int\limits_{30}^{50} \int\limits_{20}^{y} x \cdot \frac{1}{350} dx dy = \int\limits_{30}^{50} \frac{y^2 - 400}{700} dy = \frac{y^3 - 1200y}{2100} \Bigm|^{50}_{30} = \frac{740}{21} \approx 35.24
    \]
    
    \item Последние 3 выстрела Алекс будут промахами, а 3 по счету будет попаданием (т.к. иначе Алекс уйдет после 5 выстрелов). Вероятность трёх последних промахов: $0.4 \cdot 0.4 \cdot 0.1$. Осталось рассмотреть остальные случаи для 1 и 2 выстрелов (П - попал, Н - не попал): ПП, ПН, НП, НН
    
	Мы знаем, что до первого выстрела, который мы увидели, был промах (обозначим его нулевым выстрелом). Значит, первый выстрел реализуется с вероятностями $0.4, \, 0.6$. Тогда сложим вероятности всех возможных исходов, при которых Алекс покинет салун после 6 выстрелов:
	\begin{multline*}
	\P(\text{Мы увидим 6 выстрелов }|\text{ Нулевой выстрел — промах}) =\\
	= 0.4 \cdot 0.4 \cdot 0.1 \cdot (0.6 \cdot 0.9 \cdot 0.9 + 0.6 \cdot 0.1 \cdot 0.6 + 0.4 \cdot 0.6 \cdot 0.9 + 0.4 \cdot 0.4 \cdot 0.6) = 0.013344
	\end{multline*}
	
    \item Нужно найти $\Cov(X_1, X_2 \, | \, X_1 = 2) = \E(X_1 \cdot X_2 \, | \, X_1 = 2) - \E(X_1 \, | \, X_1 = 2) \E(X_2 \, | \, X_1 = 2)$
    
    Найдем каждое матожидание по отдельности:
    \begin{multline*}
        \E(X_1 \cdot X_2 \, | \, X_1 = 2) = \sum\limits_k k \cdot \P(X_1 \cdot X_2 = k \, | \, X_1 = 2) = \sum\limits_k k \cdot \frac{\P(X_1 \cdot X_2 = k \, \cap \, X_1 = 2)}{\P(X_1 = 2)} =\\
        = \sum\limits_k k \cdot \frac{\P(X_2 = \frac{k}{2} \, \cap \, X_1 = 2)}{\P(X_1 = 2)} = 1 \cdot 0 + 2 \cdot \frac{4}{16} + 4 \cdot \frac{4}{16} + 8 \cdot 0 + 16 \cdot \frac{4}{8} + 64 \cdot 0 = 9.5
    \end{multline*}
    \begin{multline*}
        \E(X_2 \, | \, X_1 = 2) = \sum\limits_k k \cdot \P(X_2 = k \, | \, X_1 = 2) = \sum\limits_k k \cdot \frac{\P(X_2 = k \, \cap \, X_1 = 2)}{\P(X_1 = 2)} =\\
        = 1 \cdot \frac{4}{16} + 2 \cdot \frac{4}{16} + 8 \cdot \frac{4}{8} = 4.75
    \end{multline*}
    
    Итого: $\Cov(X_1, X_2 \, | \, X_1 = 2) = 9.5 - 2 \cdot 4.75 = 0$
    
    \item Нужно найти $\displaystyle\P(\eta > 0.5 \, | \, \xi \le \frac{1}{3}) = \P(2\xi - \mathbb{I}\{\xi > \frac{1}{2}\} > 0.5 \, | \, \xi \le \frac{1}{3}) = \frac{\P(2\xi - \mathbb{I}\{\xi > \frac{1}{2}\} > 0.5 \, \cap \, \xi \le \frac{1}{3})}{\P(\xi \le \frac{1}{3})}$
    \[
    \P(2\xi - \mathbb{I}\{\xi > \frac{1}{2}\} > 0.5 \, \cap \, \xi \le \frac{1}{3}) = \P(2\xi > 0.5 \, \cap \, \xi \le \frac{1}{2}) + \P(2\xi > 1.5 \, \cap \, \frac{1}{2} < \xi \le \frac{1}{3}) = \P(\frac{1}{4} < \xi \le \frac{1}{2}) + 0 = \frac{1}{4}
    \]
    
    Итого: $\displaystyle \P(\eta > 0.5 \, | \, \xi \le \frac{1}{3}) = \frac{1/4}{1/3} = \frac{3}{4}$
    
    \item
    \[
    \P(\text{Хотя бы один выбрал свою лощадь} \, | \, \text{Одек выбрал не ту лошадь}) =
    \]
    \[
    = 1 - \P(\text{Все не на своих лошадях} \, | \, \text{Одек выбрал не ту лошадь}) =
    \]
    \[
    = 1 - \frac{\P(\text{Все не на своих лошадях} \, \cap \, \text{Одек выбрал не ту лошадь})}{\P(\text{Одек выбрал не ту лошадь})} =
    \]
    \[
    = 1 - \frac{\P(\text{Все не на своих лошадях})}{\P(\text{Одек выбрал не ту лошадь})} =
    \]
    \[
    = 1 - \frac{(3/4)^4}{3/4} = 1 - \frac{27}{64} = \frac{37}{64} \approx 0.578
    \]
\end{enumerate}    
    
    \item Пуассон  

\begin{enumerate}
    \item «Поляна Гонзалеса»
    
    \[ \lambda - 1 \leq k \leq \lambda \]
    
    \[ k = 1, 2 \]
    
    \item «Нападения Гонзалеса»
    
    \[ 
    \lambda_{\text{12 часов}} = \frac{12}{24} \cdot 0.5
    \]
    \[
    P(\text{разлучится}) \cdot (P1 + P2 + \ldots) = P(\text{разлучится}) \cdot (1 - P0)
    \]
    \[
    = 0.5 \cdot (1 - \exp{(-\lambda_{\text{12 часов}})}) = 0.5 * \exp{(-\frac{1}{4})} 
    \]
    
    \item «Противостояние Гонзалесу»
    
    \[ 
    \lambda_{\text{Гоназлес}} = 6
    \]
    \[
    \lambda_{\text{Удары}} = 3
    \]
    \[
    \lambda{\text{Ударов по Гонзалесу}} = 0.5
    \]
    \[
    P(\text{Успех}) = P(k=1) \cdot 0.1 + 2 \cdot P(k=2) \cdot 0.1  = 0.1 * \left( \frac{0.5^{-1}}{1!} \cdot \exp{(-0.5)} + 2 \cdot \frac{0.5^{-2}}{2!} \cdot \exp{(-0.5)}  \right)
    \]
    
    \item
    
    \[
    \lambda = \frac{30}{60} \cdot \frac{1}{12} = \frac{1}{24}
    \]
    \[
    P(k \geq 1) = 1 - P(k = 0) = 1 - \exp{(-\frac{1}{24})}
    \]
    
    
    \item
    
    \[
    \lambda = 3 \cdot 4 = 12
    \]
    \[
    P(k=10) = \frac{\lambda^{10}}{10!} \cdot \exp{(-\lambda)} = \frac{12^{10}}{10!} \cdot \exp{(-12)}
    \]
    
    \item 
    
    \[
    \lambda = \frac{12}{4} \cdot 0.25 = 0.75
    \]
    \[
    P(k \geq 1) = 1 - P(k=0) = 1 - \exp{(-0.75)}
    \]
    
\end{enumerate}    
    
    \item Ковариации и ожидание

\begin{enumerate}
    \item «От хорошего дерева - хорошее каноэ»
    
     \begin{tabular}{c|c|c}
			\hline $B \backslash A$ & $A=0$ & $A=1$  \\
			\hline  $B = 0$ & $\frac{3}{6}$ & $\frac{1}{6}$  \\
			\hline  $B = 1$ & $\frac{2}{6}$ & 0  \\
			\hline
		\end{tabular} \\
		
		$\E(A) = \frac{1}{6}$
		
		$\E(B) = \frac{2}{6}$
		
		$\E(A \cdot B) = 0$
		
		$\Cov(A,B) = 0 - \frac{1}{6} \cdot \frac{2}{6} = - \frac{1}{18}$
		
    \item «Своё каноэ не похвалишь - никто не похвалит»
    
    Найдем частную функцию плотности $X$:
    $f_x = \int_0^1 2 \cdot (\frac{x^2}{2} + y^2) dy = \frac{2y^3}{3}|_0^1 + x^2y|_0^1 = \frac{2}{3} + x^2$.
    
    Искомое матожидание равно:
    $\E(X) = \int_0^1 x \cdot f_x dx=\int_0^1 x \cdot (\frac{2}{3} + x^2) dx = \frac{2x^2}{6}|_0^1 + \frac{x^4}{4}|_0^1= \frac{1}{3} + \frac{1}{4} = \frac{7}{12}$.
    
    \item «Дырявая шляпа»
    
    Количество выстрелов подчиняется Геометрическому распределению. ($n = 6, p = \frac{3}{5}$)
    
    \begin{tabular}{c|c|c|c|c|c|c}
			\hline $x_t$ & 1 & 2 & 3 & 4 & 5 & 6 \\
			\hline  $p_t$ & $p$ & $qp$ & $qqp$ & $qqqp$ & $qqqqp$ & $qqqqqp$ \\
			\hline  $p_t$ & 0.6 & 0.24 & 0.096 & 0.0384 & 0.01536 & 0.006144 \\
			\hline  $x_t \cdot p_t$ & 0.6 & 0.48 & 0.288 & 0.1536 & 0.0768 & 0.036864 \\
			\hline
		\end{tabular} \\
		
	$\E(X) = \sum x_t \cdot p_t = 1.635264$
	
    \item «Лимонадный Джо»
    
    $\E(X) = 0.4 + 3 \cdot 0.3 + 5 \cdot 0.3 = 2.8$
    
    $\E(X^2) = 0.4 + 9 \cdot 0.3 + 25 \cdot 0.3 = 10.6$
    
    $\Var(X) = 10.6 - 7.84 = 2.76$
    
    $\E(Y) = 2 \cdot 0.45 = 0.9$
    
    $\E(Y^2) = 4 \cdot 0.45 = 1.8$
    
    $\Var(Y) = 1.8 - 0.81 = 0.99$
    
    $\E(X \cdot Y) = 2 \cdot 0.3 + 6 \cdot 0.1 + 10 \cdot 0.05 = 1.7$
    
    $\Cov(X,Y) = 1.7 - 0.9 \cdot 2.8 = -0.82$
    
    $\Corr(X,Y) = \frac{-0.82}{\sqrt{2.76 \cdot 0.99}} \approx -0.5$
    
    \item «Уже слепой Джо»
    
    Стрельба по бутылкам является Биномиальным распределением ($n=24, p =\frac{1}{8}$).
    Тогда $\E(X)= n \cdot p = \frac{24}{8}, \Var(X) = n \cdot p \cdot q = 24 \cdot \frac{1}{8} \cdot \frac{7}{8}$.
    
    \item «Патроны слепого Джо»
    
    Из ковариацинной матрицы следует, что $\Var(X)=4, \Var(Y)=21$.  Тогда $\Cov(X,Y)=\Corr(X,Y) \cdot \sqrt{Var(X)\cdot Var(Y)} = 0,55 \cdot \sqrt 84 \approx 5$
\end{enumerate}  

    \item Дифференциальные формы
    
\begin{enumerate}
    \item 
    \begin{enumerate}
        \item Для облегчения поиска совместной функции плотноси и переходов между преобразованиями случайных величин.
        
        \item Засчитывать любые два свойства из следующего списка
        
        \[ \begin{array}{l}{(d x+d y) \wedge d z=d x \wedge d z+d y \wedge d z} \\ {d r \wedge d s=-d s \wedge d r} \\ {(d r \wedge d s) \wedge d y=d r \wedge(d s \wedge d y)} \\ {(\lambda \cdot d r) \wedge d s=d r \wedge(\lambda \cdot d s)=\lambda \cdot(d r \wedge d s)}\end{array}
 \]
    \end{enumerate}
    
    \item 
    
    \begin{enumerate}
        \item 
        \[ f = \begin{cases}
        \frac{3}{40} e^{-3x}, x\in[0, +\infty], y \in [40,80]\\
        0, \text{ иначе }
        \end{cases}\]
        
        \item
         \[ f = \begin{cases}
        \frac{3}{40} e^{-3\frac{rs}{s+1}} \cdot \frac{|s-r|}{(s+1)^3} x\in[0, +\infty], y \in [40,80]\\
        0, \text{ иначе }
        \end{cases}\]
    \end{enumerate}
    мб тут без модуля
    \item 
    
   $ \P\{ X \in [0.5, 1]\} \approx 3 e^{-3 \cdot 0.5} \cdot 0.5 \approx 0.334$
    \item 
    \begin{enumerate}
        \item 
        
        $Y = \ln_2(X), X = 2^Y$
        
        Дифференциальная форма: $\frac{2^{3y}}{4} d2^y = \frac{2^{3y}}{4}\ln(2) dy$
        
        \item 
         $ \P\{ Y \in [0.5, 1]\} \approx \frac{2^{3\cdot 0.5}}{4}\ln(2) \cdot 0.5 \approx 0.245$
    \end{enumerate}
    \item
    Ответ:
    $\frac{x^3 - x^2 - 2x^3 \ln(y) }{(x + y)^2 \ln^2(y)} dx \wedge dy$
    \item 
    
    Ответ:
     \[ f = \begin{cases}
        (\frac{s}{r+2} - 1)(\frac{rs}{r+2} - 1)(\frac{s(4 + r^2)}{(r+2)^4}), x \in [1;2], y \in [1;3]\\
        0, \text{ иначе }
        \end{cases}\]
    
\end{enumerate}    
    
    \item Комбинаторика

\begin{enumerate}
    \item «Плохой салун»
    
    Ли и Чарльза превращаем в одного ковбоя, в итоге всего 7 мест. У Джона 6 способов разместиться (все кроме последнего места), у остальных 6! способов, у Ли и Чарльза есть возможность поменяться местами, поэтому умножаем ещё на 2.
    Итого $6\cdot6!\cdot2$
    \item «Ночное дежурство»
    
    В «нулевой» день у нас имеется $\binom{4}{20}$ способов выбрать 4 ковбоев из 20, в первый день у нас будет столько же способов за вычетом «нулевого» способа, то есть $\binom{4}{20} - 1$, затем во второй день за вычетом 2 способов (нулевого и первого дня) и так до $\binom{4}{20} - 4$. Так как нам нужно чтобы ковбои не повторялись, перемножим все дни, начиная от «нулевого» и заканчивая четвёртым:
    $\binom{4}{20}\cdot(\binom{4}{20} - 1)\cdot(\binom{4}{20} - 2)\cdot(\binom{4}{20} - 3)\cdot(\binom{4}{20}-4) = $
    $\frac{\binom{4}{20}!}{(\binom{4}{20}-5)!}$
    \item «Ковбойская игра»
    
    1) $(\frac{5}{6})^{9}$ 
    
    2) $10\cdot \frac{5^{9}}{6^{9}} - 45 \cdot \frac{5\cdot 4^{8}}{6^{9}} + 120 \cdot \frac{5\cdot 4\cdot 3^{7}}{6^{9}} - 210 \cdot \frac{5\cdot 4\cdot 3\cdot 2^{6}}{6^{9}} + 252 \cdot \frac{5\cdot 4 \cdot 3 \cdot 2 \cdot 1^{5}}{6^{9}}$
    % http://www.problems.ru/view_problem_details_new.php?id=65337
    \item «Справедливые шерифы»
    
    $\frac{\binom{10}{5}}{\binom{10}{5}\cdot\binom{10}{5}}=\frac{1}{\binom{10}{5}}$
    \item «Типичная перестрелка»
    
    $P=3\cdot\binom{20}{10}/\binom{30}{10}-3\cdot\binom{10}{10}/\binom{30}{10}$
    \item  «Охотник за головами выходит на охоту»
    
    $1-\frac{7!}{7^7}$
\end{enumerate}    
    
    \item Энтропия
    
\begin{enumerate}
    \item \[
    -\E(\log_2 \P(X)) = -(4 \cdot 0.12\log_2 0.12 + 2 \cdot 0.16\log_2 0.16 + 2 \cdot 0.06\log_2 0.06 + 0.08\log_2 0.08) \approx 3.09 
    \]
    \item Два символа значат, что основание логарифма два.
    \[
    -\E(\log_2 \P(X)) = -(3 \cdot 0.1\log_2 0.1 + 0.7\log_2 0.7) \approx 1.36 
    \]
    \item Пять символов значат, что основание логарифма пять.
    \[
    -\E(\log_5 \P(X)) = -(9 \cdot 0.11\log_2 0.11 + 0.01\log_5 0.01) \approx 1.39 
    \]
    \item Учитывая, что площадь под функцией плотности равна единице, ее значения от 0 до 10 равно $\frac{1}{20}$, от 10 до 15 равно $\frac{1}{10}$, тогда:
    \[
    H = -\int\limits_0^{10} \frac{1}{20}\cdot\log_2 \frac{1}{20}\,dx - \int\limits_{10}^{15} \frac{1}{10}\cdot\log_2 \frac{1}{10}\,dx = -\frac{1}{2}\cdot\log_2 \frac{1}{20} - \frac{1}{2}\cdot\log_2 \frac{1}{10} \approx 3.82
    \]
    \item Составляем таблицу с вероятностями, подставляем в формулу энтропии.
    \begin{center}
	    \begin{tabular}{c|c|c|c}
        Разница & 0 & 1 & 2\\ \hline
        $\P(X)$ & $\frac{1}{3}$ & $\frac{7}{18}$ & $\frac{5}{18}$\\
	    \end{tabular}
	    
    \[
    -\E(\log_2 \P(X)) = -(\frac{1}{3}\log_2 \frac{1}{3} + \frac{7}{18}\log_5 \frac{7}{18} + \frac{5}{18}\log_2 \frac{5}{18}) \approx 1.57 
    \]
\end{center}
    \item Энтропия будет максимальной, когда в барабане по три патрона с крестиком и с ноликом.
    \[
    \P(A) = \frac{C_{5}^{3}\cdot C_{15}^{3}}{C_{20}^{6}} \approx 0.12
    \]
\end{enumerate}    
    
    \item Популярные непрерывные распределения
    
\begin{enumerate}
    \item Пусть Х – длительность беременности. Тогда искомая вероятность: Р(Х > 290) + P(X < 240). Каждый студент ИП способен справиться со стандартизацией случайной нормальной величины и получить ответ 0.0241.
    \item Утверждается, P(тележка проедет 5000 миль) = Р(отставшаяся жизнеспособность тележки составит 5000) = 1 - F(5000), где F(5000) – это значение экспаненциальной функции распределения в точке 5000. Проделывая нехитрые манипуляции, получаем ответ 0.604.
    \item Засчитываются оба решения!!!
    Первое решение: предположим, что отвертие для ключа является не точкой на окружности, а некоторой эпсилон-окрестностью (все-таки отверстие достаточно велико в отличие от простой точки). Тогда мы можем рассматривать задачу в контексте равномерного распределения\ldots
    Второе решение: предположим, что речь идет о равномерном (то есть непрерывном!) распределении. Вероятность попасть в точку равна 0.
    Без соответсвующей предпосылки решение некорректно!
    \item Вероятность правильно ответа зависит от того, знают ли потомки, что выступает в качестве параметров нормального распределения (второй параметр, который 7) и есть дисперсия.
    Искомая вероятность равна отношению количества «знающих» студентов к общему числу студентов, решающих эту задачу.
    \item\ldots in process\ldots
\end{enumerate}    

    \item ЦПТ
    
    \begin{enumerate}
    
        \item
        $\E(X) = 0.9, \Var(X) = \E(X^2) - \E^2(X) = 0.49$\\
        $P(X<50)=1-F(\dfrac{4}{\sqrt{60 \times 0.49}})$
        
        \item $A = F^{-1}(0.6) \times \sqrt{350} + 350$
        
        \item Ответ: $\overline{Y}_n \sim \cN \left(\frac{3}{2}, \frac{3}{4n}\right) $

Решение: 
\[
\E(Y) = \int_{0}^{+\infty} e^{x} (3e^{-3x}) dx = \frac{3}{2}
\]
\[
\E(Y^2) = \int_{0}^{+\infty} e^{2x}(3e^{-3x}) dx =3
\]
Тогда $\E(Y) = \frac{3}{2}$, $\Var(Y) = \frac{3}{4}$. По ЦПТ, $\overline{Y}_n \sim \cN \left(\frac{3}{2}, \frac{3}{4n}\right) $
        
        \item $1 - F(\frac{20}{\sqrt{29.1}})$ - запой
        
        \item $1 - F(\frac{4}{\sqrt{10}})$
        
        \item Ответ: $1 - F\left(\dfrac{51-0.(6)\times 101}{\sqrt{101\times0.(6)\times 0.(3)}}\right)$
        
    \end{enumerate}



\end{enumerate}



\end{document}

