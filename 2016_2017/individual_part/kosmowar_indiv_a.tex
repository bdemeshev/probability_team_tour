\documentclass[12pt]{article}

\usepackage{tikz} % картинки в tikz
\usepackage{microtype} % свешивание пунктуации

\usepackage{array} % для столбцов фиксированной ширины

\usepackage{indentfirst} % отступ в первом параграфе

\usepackage{sectsty} % для центрирования названий частей
\allsectionsfont{\centering}

\usepackage{url}

\usepackage{amsmath, amssymb} % куча стандартных математических плюшек

\usepackage[top=2cm, left=1.5cm, right=1.5cm, bottom=2cm]{geometry} % размер текста на странице

\usepackage{lastpage} % чтобы узнать номер последней страницы

\usepackage{enumitem} % дополнительные плюшки для списков
%  например \begin{enumerate}[resume] позволяет продолжить нумерацию в новом списке
\usepackage{caption}


\usepackage{fancyhdr} % весёлые колонтитулы
\pagestyle{fancy}
\lhead{Теория вероятностей}
\chead{}
\rhead{2016-12-17, каждый сам за себя!}
\lfoot{}
\cfoot{}
\rfoot{\thepage/\pageref{LastPage}}
\renewcommand{\headrulewidth}{0.4pt}
\renewcommand{\footrulewidth}{0.4pt}



\usepackage{todonotes} % для вставки в документ заметок о том, что осталось сделать
% \todo{Здесь надо коэффициенты исправить}
% \missingfigure{Здесь будет Последний день Помпеи}
% \listoftodos — печатает все поставленные \todo'шки


% более красивые таблицы
\usepackage{booktabs}
% заповеди из докупентации:
% 1. Не используйте вертикальные линни
% 2. Не используйте двойные линии
% 3. Единицы измерения - в шапку таблицы
% 4. Не сокращайте .1 вместо 0.1
% 5. Повторяющееся значение повторяйте, а не говорите "то же"



\usepackage{fontspec}
\usepackage{polyglossia}

\setmainlanguage{russian}
\setotherlanguages{english}

% download "Linux Libertine" fonts:
% http://www.linuxlibertine.org/index.php?id=91&L=1
\setmainfont{Linux Libertine O} % or Helvetica, Arial, Cambria
% why do we need \newfontfamily:
% http://tex.stackexchange.com/questions/91507/
\newfontfamily{\cyrillicfonttt}{Linux Libertine O}



\AddEnumerateCounter{\asbuk}{\russian@alph}{щ} % для списков с русскими буквами
\setlist[enumerate, 2]{label=\asbuk*),ref=\asbuk*}


%% эконометрические сокращения
\renewcommand{\P}{\mathbb{P}}
\DeclareMathOperator{\Cov}{Cov}
\DeclareMathOperator{\Corr}{Corr}
\DeclareMathOperator{\Var}{Var}
\DeclareMathOperator{\E}{E}
\def \hb{\hat{\beta}}
\def \hs{\hat{\sigma}}
\def \htheta{\hat{\theta}}
\def \s{\sigma}
\def \hy{\hat{y}}
\def \hY{\hat{Y}}
\def \v1{\vec{1}}
\def \e{\varepsilon}
\def \he{\hat{\e}}
\def \z{z}
\def \hVar{\widehat{\Var}}
\def \hCorr{\widehat{\Corr}}
\def \hCov{\widehat{\Cov}}
\def \cN{\mathcal{N}}


\begin{document}


\textbf{Ф.\,И.\,О.}\dotfill

\bigskip


\textbf{Неравенства Берри–Эссеена:} Для любых $n \in \mathbb{N}$ и всех $x \in \mathbb{R}$ имеет место оценка:
\[
    \bigl|F_{S_n^{*}}(x) - \Phi(x)\bigr| \leq 0.48 \cdot \frac{\E(|\xi_i - \E\xi_i|^3)}{\Var^{3/2}(\xi_i)\cdot\sqrt{n}} \text{,}
\]
где $\Phi(x) = \int_{-\infty}^{x}\frac{1}{\sqrt{2\pi}}e^{-\frac{t^2}{2}}\,dt$, \; $S_n^* = \frac{S_n - \E(S_n)}{\sqrt{\Var(S_n)}}$, \; $S_n = \xi_1 + \ldots + \xi_n$



\begin{enumerate}

\item Сформулируйте:
\begin{enumerate}
  \item{} Центральную предельную теорему
  \item{} Неравенства Маркова и Чебышёва
  \item{} Дельта-метод
\end{enumerate}

\item Пусть $\E(\xi) = 1$, $\E(\eta) = 2$, $\Var(\xi) = 1$, $\E(\eta^2) = 10$, $\E(\xi \eta) = 1$. Найдите
\begin{enumerate}
  \item{} [8] $\E(2\xi-\eta+1)$, $\Cov(\xi, \,\eta)$, $\Corr(\xi, \,\eta)$,  $\Var(2\xi-\eta+1)$;
  \item{} [8] $\Cov(\xi+\eta, \,\xi+1)$, $\Corr(\xi+\eta, \,\xi+1)$, $\Corr(\xi+\eta-24, \,365 - \xi - \eta)$, $\Cov(2016\cdot\xi, \, 2017)$.
\end{enumerate}

\item
Совместное распределение доходностей акций двух компаний задано с помощью таблицы:

\begin{center}
\begin{tabular}{c|cc}
\toprule
         & $\eta=-1$ & $\eta=1$ \\
\midrule
$\xi=-1$  & $0.2$       & $0.2$   \\
$\xi=0$   & $0.2$       & $0.2$   \\
$\xi=2$   & $0.1$       & $0.1$   \\
\bottomrule
\end{tabular}
\end{center}

\begin{enumerate}
  \item{} [2] Найдите частные распределения случайных величин $\xi$ и $\eta$.
  \item{} [2] Найдите $\Cov(\xi,\,\eta)$.
  \item{} [2] Сформулируйте определение независимости дискретных случайных величин.
  \item{} [2] Являются ли случайные величины $\xi$ и $\eta$ независимыми?
  \item{} [2] Найдите условное распределение случайной величины $\xi$, если $\eta = 1$.
  \item{} [2] Найдите условное математическое ожидание случайной величины $\xi$, если $\eta = 1$.
  \item{} [2] Найдите математическое ожидание и дисперсию величины $\pi = 0.5\, \xi + 0.5\, \eta$.
  \item{} [2] Рассмотрим портфель, в котором $\alpha$ — доля акций с доходностью $\xi$ и $(1 - \alpha)$ — доля акций с доходностью $\eta$. Доходность этого портфеля есть случайная величина
  \[\pi(\alpha) = \alpha \xi + (1-\alpha)\eta.\]
  Найдите такую долю $\alpha \in [0;\,1]$, при которой доходность портфеля $\pi(\alpha)$ имеет наименьшую дисперсию.
\end{enumerate}


\newpage

\item Отведав медовухи, Винни–Пух совершает случайное блуждание на плоскости. Он стартует из начала координат и в каждую чётную минуту равновероятно совершает шаг единичной длины налево или направо, а каждую нечётную — вперёд или назад.

\begin{enumerate}
\item{} Какова вероятность того, что через два часа блужданий Винни-Пух окажется в области $X>10$, $Y>10$?
\item{} Используя неравенство Берри–Эссеена оцените погрешность вычислений предыдущего пункта.
\end{enumerate}


\item Купчиха Сосипатра Титовна очень любит чаёвничать. Её чаепитие продолжается случайное время $S$, имеющее равномерное распределение от 0 до 3 часов.  Встретив Сосипатру Титовну в пассаже на Петровке, её подруга Олимпиада Карповна узнала, сколько длилось вчерашнее чаепитие Сосипатры Титовны. Решив, что такая продолжительность чаепития является максимально возможной, Олимпиада Карповна устраивает чаепитие, продолжающееся случайное время $T$, имеющее равномерное распределение от 0 до $S$ часов.
\begin{enumerate}
\item Найдите совместную функцию плотности величин $S$ и $T$
\item Найдите вероятность $\P(S>2T)$
\item Найдите ковариацию $\Cov(S, T)$
\end{enumerate}


\item Каждую весну дед Мазай плавая на лодке спасает в среднем 10 зайцев, дисперсия количества спасенных зайцев за одну весну равна 10. Количество спасенных зайцев за разные года — независимые случайные величины. Точный закон распределения числа зайцев неизвестен.
\begin{enumerate}
\item Оцените в каких пределах лежит вероятность того, что за три года дед Мазай спасет от 20 до 34 зайцев.
\item Оцените в каких пределах лежит вероятность того, что за одну весну дед Мазай спасет более 11 зайцев.
\end{enumerate}

\item Есть три комнаты. В первой из них лежит сыр. Если мышка попадает в первую комнату, то она находит сыр через одну минуту. Если мышка попадает во вторую комнату, то она ищет сыр две минуты и покидает комнату. Если мышка попадает в третью комнату, то она ищет сыр три минуты и покидает комнату. Покинув комнату, мышка выходит в коридор и выбирает новую комнату наугад, например, может зайти в одну и ту же. Сейчас мышка в коридоре. Сколько времени ей в среднем потребуется, чтобы найти сыр?

\end{enumerate}



\end{document}
