\documentclass[a4paper,12pt]{article}


\usepackage{tikz} % картинки в tikz
\usepackage{microtype} % свешивание пунктуации

\usepackage{array} % для столбцов фиксированной ширины

\usepackage{indentfirst} % отступ в первом параграфе

\usepackage{sectsty} % для центрирования названий частей
\allsectionsfont{\centering}

\usepackage{amsmath, amsthm, amssymb} % куча стандартных математических плюшек

\usepackage{comment}
\usepackage{amsfonts}

\usepackage[top=2cm, left=1cm, right=1cm, bottom=2cm]{geometry} % размер текста на странице

\usepackage{lastpage} % чтобы узнать номер последней страницы

\usepackage{enumitem} % дополнительные плюшки для списков
%  например \begin{enumerate}[resume] позволяет продолжить нумерацию в новом списке
\usepackage{caption}

\usepackage{hyperref} % гиперссылки

\usepackage{multicol} % текст в несколько столбцов


\usepackage{fancyhdr} % весёлые колонтитулы
\pagestyle{fancy}
\lhead{}
\chead{Декабрь 2018}
\rhead{Командный тур, ИП часть кр 2}
\lfoot{}
\cfoot{}
\rfoot{Теория вероятностей}
\renewcommand{\headrulewidth}{0.4pt}
\renewcommand{\footrulewidth}{0.4pt}



\usepackage{todonotes} % для вставки в документ заметок о том, что осталось сделать
% \todo{Здесь надо коэффициенты исправить}
% \missingfigure{Здесь будет Последний день Помпеи}
% \listoftodos — печатает все поставленные \todo'шки


% более красивые таблицы
\usepackage{booktabs}
% заповеди из докупентации:
% 1. Не используйте вертикальные линни
% 2. Не используйте двойные линии
% 3. Единицы измерения - в шапку таблицы
% 4. Не сокращайте .1 вместо 0.1
% 5. Повторяющееся значение повторяйте, а не говорите "то же"


\usepackage{fontspec}
\usepackage{polyglossia}

\setmainlanguage{russian}
\setotherlanguages{english}

% download "Linux Libertine" fonts:
% http://www.linuxlibertine.org/index.php?id=91&L=1
\setmainfont{Linux Libertine O} % or Helvetica, Arial, Cambria
% why do we need \newfontfamily:
% http://tex.stackexchange.com/questions/91507/
\newfontfamily{\cyrillicfonttt}{Linux Libertine O}

\AddEnumerateCounter{\asbuk}{\russian@alph}{щ} % для списков с русскими буквами
% \setlist[enumerate, 2]{label=\asbuk*),ref=\asbuk*}

%% эконометрические сокращения
\DeclareMathOperator{\Cov}{Cov}
\DeclareMathOperator{\Corr}{Corr}
\DeclareMathOperator{\Var}{Var}
\DeclareMathOperator{\E}{E}
\def \hb{\hat{\beta}}
\def \hs{\hat{\sigma}}
\def \htheta{\hat{\theta}}
\def \s{\sigma}
\def \hy{\hat{y}}
\def \hY{\hat{Y}}
\def \v1{\vec{1}}
\def \e{\varepsilon}
\def \he{\hat{\e}}
\def \z{z}
\def \hVar{\widehat{\Var}}
\def \hCorr{\widehat{\Corr}}
\def \hCov{\widehat{\Cov}}
\def \cN{\mathcal{N}}
\def \P{\mathbb{P}}

\begin{document} % конец преамбулы, начало документа

\section*{Правила сражений}

Итак, наточите мечи и запаситесь маной. Мы отправляемся в удивительный тур по полному опасностей королевству! 
Не бойтесь трудностей и не хватайтесь за головы, 
ведь у каждого отряда будет свой проводник из числа смелых ассистентов, живущих на земле Пуассона.

Командный тур проходит в несколько этапов. Первый этап будет проходить в окрестностях замка Пуассона. 
Каждой команде будет выдана карта. Подобраться к окрестностям замка можно по трём тропинкам. 
На каждой вас ждёт несколько локаций с увлекательными задачками. Начинать можно с любой тропинки. 
Чем ближе локация к замку, тем больше ценятся решённые задачи. Задача каждой команды: 
пробиться к замку и набрать максимальное количество монет за решение задач. 

 На каждой локации, вне зависимости от уровня команде выдаётся листок с пятью задачами: 
 двумя средней сложности и тремя небольшой сложности. 
 Для прохода на следующую локацию необходимо решить либо одну среднюю задачу, либо три лёгких.  
 Темы задач варьируются от локации к локации.

По мере приближения к замку задачи возрастают по цене, но не по сложности. 
Финальные локации на обеих картах: замок Пуассона и сокровищница Пуассона. 
Оставь незнание, всяк в него входящий! 
При достижении последней локации команде выдаётся листок с тремя сложными задачами.  
У сложных задач принимаются частичные решения, и, соответственно, награда за них дробится. 

%Ниже можно увидеть таблицу с ценностью одной задачи на каждом этапе, выраженную в золотых монетах.

%\begin{center}
%	\hspace*{-2cm}
%	\begin{tabular}{ccccc}
%		\toprule
%		$ \text{Тип задачи/уровень} $ & Граница & Владения короля &Окрестности замка& Замок Пуассона \\ \midrule
%		Лёгкая задача & $1$ & $ 2 $  & $ 2 $&\\
%		Средняя задача & 3 & $ 5 $  & $ 6 $&\\
%		Сложная задача &  &  & &10\\
%		Бонус за весь листок & 2 &  3&4 &5\\ \bottomrule
%	\end{tabular}
%\end{center}
%
Если команда считает, что неспособна пройти на следующую локацию, но имеет достаточно монет, 
то может подкупить злобных, но жадных экспоненциальных гоблинов, 
которые проведут её на следующий уровень. Цена: 20 монет.


Второй этап проходит аналогично первому, но наши отважные путешественники находятся уже внутри замка Пуассона. 
Карта аналогична предыдущей, однако изменены локации и наполняющие их задачи. 
Новая цель: отыскать сокровищницу и снять магические вероятностные печати с её входа.  

В замке Пуассона гостит сумасшедший король соседних земель. Он любит проводить дуэли. 
Если он обратит на отряд взор, то команде придётся сражаться. 
Ниже официальный свод правил дуэли, принятый в королевстве Пуассона: 
Один участник команды проходит за дуэльный стол, 
где он должен один на один с соперником в течение 5 минут решить задачу. 
Выигрывает тот, кто первым решает правильно задачу. 
В случае неправильного ответа обоими участниками, магистр дуэлей выбирает победителя по более точному решению.


Король считает своим долгом, вовлечь все отряды в дуэли. 
Однако по истечению некоторого времени, 
магический куб вероятностей будет указывать ему на случайный отряд.





\end{document}