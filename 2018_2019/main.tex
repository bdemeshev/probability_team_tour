\documentclass[a4paper,12pt]{article}

% https://www.overleaf.com/project/5bfc49efb1113009d5594a73


\usepackage{cancel}
%%% Работа с русским языком
\usepackage{mathtext} 				% русские буквы в формулах
\usepackage[T2A]{fontenc}			% кодировка
\usepackage[utf8]{inputenc}			% кодировка исходного текста
\usepackage[english,russian]{babel}	% локализация и переносы
\usepackage{booktabs}

%%% Дополнительная работа с математикой
\usepackage{amsfonts,amssymb,amsthm,mathtools} % AMS
\usepackage{amsmath}
\usepackage{icomma} % «Умная» запятая: $0,2$ — число, $0, 2$ — перечисление

\usepackage[top=1.5cm, left=1.5cm, right=1.5cm, bottom=1.5cm]{geometry} % размер текста на странице

%% Номера формул
%\mathtoolsset{showonlyrefs=true} % Показывать номера только у тех формул, на которые есть \eqref{} в тексте.

%% Шрифты
\usepackage{euscript}	 % Шрифт Евклид
\usepackage{mathrsfs} % Красивый матшрифт

\usepackage{ulem}
%% Свои команды
\DeclareMathOperator{\sgn}{\mathop{sgn}}

%% Перенос знаков в формулах (по Львовскому)
\newcommand*{\hm}[1]{#1\nobreak\discretionary{}
{\hbox{$\mathsurround=0pt #1$}}{}}

%%% Работа с картинками
\usepackage{graphicx}  % Для вставки рисунков
\graphicspath{{images/}{images2/}}  % папки с картинками
\setlength\fboxsep{3pt} % Отступ рамки \fbox{} от рисунка
\setlength\fboxrule{1pt} % Толщина линий рамки \fbox{}
\usepackage{wrapfig} % Обтекание рисунков и таблиц текстом

%%% Работа с таблицами
\usepackage{array,tabularx,tabulary,booktabs} % Дополнительная работа с таблицами
\usepackage{longtable}  % Длинные таблицы
\usepackage{multirow} % Слияние строк в таблице
\usepackage{upgreek}
\usepackage{enumerate}
\usepackage{ dsfont }


\DeclareMathOperator{\cov}{Cov}
\DeclareMathOperator{\Var}{Var}
\DeclareMathOperator{\Cov}{Cov}
\DeclareMathOperator{\Corr}{Corr}
\DeclareMathOperator{\E}{\mathbb{E}}
\let\P\relax
\DeclareMathOperator{\P}{\mathbb{P}}

\def \cN{\mathcal{N}}




%%% Заголовок
\author{}
\title{}
\date{\today}

\begin{document} % конец преамбулы, начало документа

Антураж:

Положительные\\

\begin{enumerate}
\item Ходжа Насреддин
\item Принцессы
\item Драконы
\item Замки
\item Рыцари
\item Феи
\item Квесты
\item Дуэли
\end{enumerate}

Отрицательные:
2
\begin{enumerate}
\item Чума
\item Колдуны
\item Крысы
\item Инквизиция
\item Крестовые походы
\end{enumerate}

Дедлайны:

Среда вечер - названия

Пятница вечер - задачи\\

Терминология задач:

Лёгкие и средние, в замке сложные.

Три лёгкие и две средние должны решаться командой из 5 человек за 10 минут.
\\

Предположительное начало: 15.00 - 16.40

Первый тур - карта
Второй тур - карта плюс дуэли

Тайминг на дуэль в сумме - 7 минут

9 команд по 5-6 человек



Карта, которая одинаковая для двух туров

На каждой карте есть локации первого уровня, в каждой две большие по три балла и три маленькие задачи по 1 баллу. Изначально выдаём им по три листа. Для 1 уровня если решён весь листок с задачами, получают бонус плюс 2 балла.


 Чтобы отправиться в следующий уровень нужно решить 1 большую или 3 маленьких задачи. Выдаются ещё два листа.

 На втором уровне ве большие по 4.5 балла и три маленькие по 1.5. Либо 1 большую, либо 3 маленькие. За полный второй уровень - три балла.

 На третьем так же, только большая 6 баллов, маленькая 2 балла. За весь лист - 4 балла.

 На первой карте центр, на второй - сокровищница. В замке и в сокроыищнице лежит 3 задачи по 10 баллов, принимаются частичные решения, балы дробятся. За 20 баллов на любом этапе можно купить проход на уровень выше.

 В замке - разные задачи

 В замке 6 тем. \\


\section*{Зоны ответственности}

Альберт отвечает за еду!

Матвей и Даша - условные вероятности + команды + правила

Петя - Метод первого шага, 1 супербольшая, 4 больших и 6 маленьких

Сева - Комбинаторика, 1 супербольшая, 4 большие и 6 маленьких

Команда А (Ася ++ ) Непрерывные распределения (супербольшая) + энтропия + неравенства + геометрия (супербольшая)

Марина составляет 4 больших задачи и 6 маленьких по теме функции распределения(интегралы) их свойства.

Аида - бесконечно малые, дифференциальная форма, птички, распределения 1 супербольшая, 4 больших, 6 малых

Саша Андреевский Матожидание, дисперсия, ковариация, корреляция, ковариационная матрица 4 больших задачи, 6 маленьких

Юля дискретные распределения (пуассон, равновероятное, геометрическое, биномиальное), 4 больших, 6 маленьких

Петя (плюс желающие) — расставить столы





ниже идут задачи уже раскинутые по турам! Находишь себя и пишешь текст!

\contentsname
\tableofcontents.

\section*{Правила для ассистентов}
Каждый закреплён за своей командой и должен находиться рядом с ней. Возможные исключения: Добрая фея Ася отлетает на минуту, чтобы выдать плюшку.

Принимаемые задачи необходимо оценить и внести в табличку, которую сделала Даша. Лучше запомнить, что сколько стоит.

Третьим туром будет Перуда, которую мы не объявляли в анонсе правил. Во время Перуды необходимо подробно объяснить правила и сделать это в короткие сроки. Если вы никогда не видели эту игру, то лучше хотя бы один кон сыграть, иначе ничего не получится по ходу.

Если команда подкупает гоблинов чтобы проскочить локацию, необходимо это учесть в таблице.
\newpage
\part*{Тур 1}


% по сюжеше хту действие разворачивается снаружи замка, поэтому названия локаций-тем должны быть уличными
% например, Трясина Условных Вероятностей, Врата Бесконечно Малых


\section{Комбинаторное болото} % здесь нужно заменить на средневековое название
% автор: Сева
%настоящее болото в третьей задаче, все остальное норм- Аида

\begin{enumerate} % две средних задачи и три лёгких
\item % проверено, сложно, пусть будет?
В алфавите камышовых людей $55$ букв. В языке используются только такие слова, в которых ни одна буква не встречается дважды. Язык настолько развит, что в нём есть каждое возможное слово каждой возможной длины, с учетом предыдущих ограничений.

Какова вероятность того, что случайно выбранное слово из камышового языка будет составлено ровно из $55$ букв?
\item
Принц и Рапунцель весело проводили время в башне. Когда принцу пришла пора уходить, он случайно уронил игрушку Рапунцель, куб состоящий из 8 маленьких кубиков. Куб сделан из слоновой кости, а снаружи покрыт позолотой.

К тому времени совсем стемнело, и принцу пришлось собирать куб наощупь. Какова вероятность, что ему удастся правильно собрать куб (то есть позолотой наружу)?
\item (лёгкая)
% проверено!
Солдат и Черт играют в карты. Правила таковы: солдат тащит три карты с возвращением из колоды в 52 карты. Если на третьей карте масть совпадет с мастью второй карты, а достоинство — с достоинством первой карты, то Солдат выигрывает. В противном случае выигрывает Черт.

Какова вероятность, что нечистый проиграет?
\item (лёгкая)
Иван-царевич решил оценить количество лягушек-квакушек в болоте. Для этого он выловил $100$ лягушек, окольцевал их, и отпустил обратно в болото. Через некоторое время он снова выловил $100$ лягушек.

Какова вероятность того, что среди отловленных окажется $18$ окольцованных, если всего в болоте обитает $250$ лягушек?
\item (лёгкая)
% заменил
За круглый стол короля Артура в случайном порядке рассаживаются рыцари. Каждый порядок рассадки равновероятен.

Какова вероятность, что Ленселот и Галахад окажутся напротив друг друга, если всего рыцарей $12$?
\end{enumerate}
\newpage
\section{Комбинаторное болото - ответы. Автор задач: Сева}
% автор: Сева

\begin{enumerate}
\item
$\P(A) = \frac{55!}{\sum\binom{55}{k}\cdot k!} = \frac{55!}{\sum\frac{55!\cdot k!}{k!\cdot (55-k)!}}=\frac{1}{\sum \frac{1}{(55-k)!}} = \frac{1}{\frac{1}{0!}+\frac{1}{1!}+\frac{1}{2!}+\dots+\frac{1}{54!}}\approx1/e$


\item
Позиция — место маленького кубика в большом кубе, положение - ориентация граней кубика. У каждого кубика 8 возможных позиций и 24 возможных положения (фиксируем одну из 6 граней - остается еще 4 возможных положения, т.е всего $6\cdot4$). В правильном кубе у каждого кубика 8 возможных позиций и 3 возможных положения (фиксируем одну из трех позолоченных граней, остается только одно правильное положение, всего $3\cdot1$). Искомая вероятность равна $\frac{8!\cdot3^8}{8!\cdot24^8}$
\item (лёгкая)
$36/51^2 = 4/17^2$
\item (лёгкая)
$(\binom{100}{18}\cdot\binom{150}{82})/\binom{250}{100}$
\item (лёгкая)
$\frac{10!\cdot2!}{11!}$
\end{enumerate}
\newpage

\section{Пещера условного колдуна} % здесь нужно заменить на средневековое название
% автор: Матвей, Даша
%нормальные задачи- проверено Аида

\begin{enumerate} % две средних задачи и три лёгких
\item Король Артур решил убить дракона, что похитил его сокровища. Он позвал сэра Ланселота и сэра Персифаля помочь ему в этом деле. Каждый из них получил по огненному шару от сэра Мерлина. Огненный шар короля Артура попадает в цель с вероятностью $\frac{1}{3} $, шары сэра Ланселота и сэра Персифаля  с вероятностстью $\frac{1}{4}$ и $\frac{1}{6}$ соответсвенно. Чтобы убить дракона, нужно два попадания. Какова вероятность убить дракона, если Король Артур попал в цель? Если дракон убит, какова вероятность, что попал король Артур?

\item Король издал указ: в целях дальнейшего укрепления развитого феодализма и предотвращения распыления наследства (передаваемого по мужской линии) семьям, уже имеющим мальчика, запретить заводить новых детей. Главный маг королевства, подчиняясь воле короля заколдовал все семьи так, что они не могли нарушить данный указ.

Как повлияет этот указ на долю мужчин в королевстве?

\item (лёгкая)
% проверено
Одна королевская семья обратилась к волшебнику, чтобы узнать безусловную вероятность появления мальчика в их семье. Колдун лишь сказал, что условная вероятность того, что в семье с двумя детьми оба мальчика, если хотя бы один ребёнок — мальчик, равна $\frac{81}{225}$. Король был в ярости, так и не получив ответа.

Помогите колдуну избежать наказания: сообщите Королю безусловную вероятность появления мальчика!

\item (лёгкая) Маг знает, что к нему из 100 жителей ближайшей деревни на консультацию приходят только 70, при этом только 20 из них верят в предсказание мага. Какова вероятность того, что если житель ходит на консультации мага, то он верит?
\item (лёгкая)
 Гендальф пытается уговорить Бильбо Бэггинса отправиться в путешествие. Бильбо очень не хочет идти. Но Гендальф придумывает игру, если Бильбо проиграет, то он должен пойти. Игра заключается в следующем: подбрасывается правильная монетка. Если выпадает РОО, то побеждает первый игрок, если ООР, то второй. Каким игроком должен быть Гендальф, чтобы отправить домоседа Бильбо в путешествие с большей вероятностью?
\end{enumerate}
\newpage
\section{Пещера условного колдуна-ответы. Авторы задач: Даша, Матвей}
% автор: Даша, Матвей

\begin{enumerate}
\item
Нужно нарисовать дерево и оттуда получить

$\P(\text{У}|\text{Артур}) = \dfrac{3}{8}$

$\P(\text{У}|\text{Ланселот}) = \dfrac{4}{9}$

$\P(\text{У}|\text{Персифаль}) = \dfrac{1}{2}$

$\P(\text{Артур}|\text{У}) = 0.(81)$

\item
Нарисуем дерево, где за мальчиком конец игры на первом и втором шаге, а девочка на первом шаге возвращает нас в начало игры. Пусть d - количество девочек. Тогда $ d = \frac{1}{2}(d + 1)$ При $d = 1 \Rightarrow m = 1$. Следовательно, $\frac{m}{m+d} = \frac{1}{2}$
\item (лёгкая)
$\frac{9}{17} = \frac{162}{306} \approx 0.53$

\item (лёгкая)

Нарисовать дерево, и очевидно, что $\dfrac{2}{7}$

\item (лёгкая)

Нарисовать рекурсивное дерево и через систему уравнений найти, что вероятность выиграть у РОО = $\dfrac{3}{4}$

\end{enumerate}
\newpage


\section{Логово орков-беспредельщиков} % здесь нужно заменить на средневековое название
% автор: Юля

\begin{enumerate} % две средних задачи и три лёгких
\item % Проверено
Колдунья Беллатрикс имеет $n$ сундуков. В каждом из сундуков лежит $n$ белых камней и ровно 1 чёрный камень. При свершении таинства Беллатрикс не глядя достаёт по одному камню из каждого сундука. Считается, что если ей попался чёрный камень, то быть беде.

Найдите вероятность того, что колдунье никогда не попадётся предвестник несчастья, если $n$ беспредельно велико!
\item  %
Орк-беспредельщик хочет найти предел по вероятности $Y_n$:
\[
Y_n = \frac{X_1^2 + X_2^2 + \ldots + X_n^2}{X_1 + X_2 + \ldots + X_{2n}}
\]

Величины $X_i$ — независимы и имеют нормальное распределение $\cN(1;1)$.

Помогите Орку скорее!
%Алхимик Северус в среднем изготавливает 20 зелий в день. Число изготовленных зелий подчиняется закону Пуассона.

%Какова вероятность того, что он изготовит чётное количество зелий?
\item (лёгкая)%Проверил Саша Р
Благородные рыцари без вредных привычек любят играть в кости. Они радуются, когда им попадаются шестёрки, ведь тогда они могут получить бесплатные чашки чая от местной таверны.

Найдите вероятность того, что в 12 бросках выпадет не менее двух шестёрок.
\item (лёгкая)
Дикий орк Квадратур рисует квадратики со случайными сторонами, стороны квадратиков распределены равновероятно принимают значения 1 и 2 и независимы. Больше орк рисовать ничего не умеет :)

Как примерно распределена средняя площадь 100 квадратиков?

\item (лёгкая)
Орк покупает кофе каждый день. Цена кофе каждый день — случайная величина, имеющее ожидание 100 золотых и дисперсию 400 золотых.

Какова вероятность того, что за 100 дней суммарные расходы превысят 12000 золотых?
\end{enumerate}

\newpage
\section{Логово дискретных орков-ответы. Автор задач: Юля}
% автор: Юля

\begin{enumerate}
\item
Вероятность вытащить чёрный камень из сундука равна $\frac{1}{n}$, где \newline $n\to\infty$. Вероятость не вытащить чёрный камень из сундука равна $(1-\frac{1}{n+1})^n$ При $n\to\infty$ это выражение стремится к $e^{-1}$.
\item $1$, делим на $n$ в числителе и знаменателе, далее по ЗБЧ
%Вероятность изготовить $k$ зелий равна $\frac{e^{-20} \cdot 20^k}{k!}$.

%Рассмотрим сумму вероятностей распределения Пуассона:

%\[
%\sum_{k=1}^\infty \frac{e^{-20} \cdot 20^k}{k!}=1
%\]
%С другой стороны, можно представить как
%\[
%e^{-20} \cdot e^{20} = 1 = 1+\frac{20}{1!}+\frac{20^2}{2!}+ \ldots
%\]
%${e^{-20\cdot 2}}={e^{-20}} \cdot{e^{-20}}$ - результатом этого является гармонический ряд: $1-\frac{20}{1!}+\frac{20^2}{2!}- \ldots$. Складывая оба выражения, получим удвоенную вероятность чётного числа приготовленных зелий (нечётные номера взаимно уничтожатся). Следовательно, искомая вероятность равна $\frac{1+e^{-40}}{2}$.
\item (лёгкая)
$1- \sum_{x=0}^1 \ {C^x_{12}}\cdot\ {\frac{1}{6}}^x\cdot{\frac{5}{6}}^{(12-x)}$.
\item (лёгкая)
Примерно $\cN(5/2;9/400)$
%По принципу симметрии, длина левого промежутка равна 59. $59\cdot 2=118$. $118+1=119$.
\item (лёгкая)
$F(0.2/4) \approx 0.52$
% 5 кофеен делят весь отрезок на 6 равных частей. $70-5=65$, то средняя длина первых пяти отрезков равна 13. Общее число кофеен приближённо равно $70+13=83$.
\end{enumerate}

\newpage
\section{Лагерь ковариационного рыцаря и его ожидательного коня}
% автор: Саша Андреевский
% нормальные задачи - Аида + Марина
\begin{enumerate} % две средних задачи и три лёгких
\item Храбрый рыцарь Ланселот решил непременно покорить сердце прекрасной дамы Амелии. Его друг Артур поведал ему, что принцессу можно очаровать, рассказав о своих завоеваниях, но она любит четные числа, поэтому ей нравится только четное число выигранных сражений. Вероятность, с которой количество побед понравится Амелии, представлено ниже в таблице:
\begin{center}\begin{tabular}{lcccc}
\toprule
 $x$     & $2$  & $4$   & $12$ & $18$  \\ \midrule
$\P(X=x)$                 & $0.10$ & $0.20$ & $0.25$ & $0.45$\\
 \bottomrule
\end{tabular}\end{center}
Найдите ожидаемое число битв, в которых Ланселоту необходимо одержать победу.
\item Ланселот позвал своих верных товарищей, Артура и Гавейна, чтобы обсудить, как ему добиться расположения прекрасной дамы Амелии. Артур и Гавейн прибудут на встречу независимо друг от друга с вероятностями 0.7 и 0.4 соответственно. Пусть $S$ - число друзей, который навестят Ланселота. Найдите $\E(S)$.
\item (лёгкая) Верный спутник Ланселота, ожидательный конь Персеваль, задумался о том, что между тем, насколько хорошо его хозяин освоит верховую езду, и тем, сколько он одержит побед в боях, есть взаимосвязь. Он рассчитал, что ковариация равна 2, $\E(Y)$ = 6, $\E(XY) = 14$. Помогите Персевалю найти $\E(X)$.
\item (лёгкая) Чтобы развлечься, Персеваль решил рассчитать $\E(2Z-11)$, где $Z=A-B$, $\E(A)$ = 15, $\E(B)$ = 30. Помогите ему справиться с этим заданием. Можно ли утверждать, что $A$ и $B$ - независимые величины?
\item (лёгкая) Послушав советы друзей, Ланселот решил подарить Амелии букет её любимых цветов - Амариллисов. Пусть количество цветов в букете, который подарит рыцарь, определяется случайной величиной $X$, а число уличных торговцев, которые рыцарю нужно обойти, чтобы найти те самые цветы, определяется величиной $Y$. Найдите $\E(3X-5Y+12)$, $\E^2(X+4Y-2)$, если $\E(X)=15$, $\E(Y)=7$.
\end{enumerate}

\newpage
\section{Лагерь ковариационного рыцаря и его ожидательного коня-ответы. Автор задач: Саша Андреевский}
% автор: Саша Андреевский

\begin{enumerate}
\item $\E(X)=0,2+0,8+0,3+3+8,1=12,1$
\item Представим в виде таблицы возможные исходы (пришли оба друга, пришел один из друзей, никто не пришел):

\begin{center}\begin{tabular}{lccc}
\toprule
 $s$     & $0$  & $1$   & $2$ \\
$\P(S=s)$ & $0.18$ & $0.54$ & $0.28$\\
\bottomrule
\end{tabular}\end{center}

Тогда $\E(S)={1}\cdot{0,54}+{2}\cdot{0,28}=1,1$.
\item (лёгкая) $\E(X)=\frac{14-2}{6}=2$
\item (лёгкая) $\E(2Z-11)={2}\cdot{\E(A-B)}-11=-30-11=-41$. Для утверждения о независимости A и B недостаточно данных.
\item (лёгкая) $\E(3X-5Y+12)=3\E(X)-5\E(Y)+12=45-35+12=22$. $\E^2(X+4Y-2)=(15+28-2)^2=1681$.
\end{enumerate}





\newpage
\section{Чулан распределений в хижине ведьмы} % здесь нужно заменить на средневековое название
% автор: Марина
%нормальные задачи- Аида

\begin{enumerate} % две средних задачи и три лёгких
\item % средняя
Функция распределения некоторой случайной величины $X$ имеет вид:
\[
F_X(x) =
\begin{cases}
0,&\text{если } x\in(-\infty , 0) \\
x^2 + 2018! \cdot c_1,&\text{если } x\in[0, 1) \\
x + c_2^{2019}, ,&\text{если } x\in[1, 2) \\
c_3, &\text{если } x\in[2, +\infty)
\end{cases}
\]
Найдите $c_2$.
\item % средняя
Дана функция плотности случайной величины $X$.
\[
f_X(x) =
\begin{cases}
\frac{3}{16}x^2,&\text{если } x\in[-2, 2] \\
0,&\text{если } x\notin[-2, 2]
\end{cases}
\]
Найдите функцию распределения случайной величины $X$.
\item % (лёгкая)
Дана функция плотности случайной величины $X$.
\[
f_X(x) =
\begin{cases}
cx+23cx^2,&\text{если } x\in[0, 1] \\
0,&\text{если } x\notin[0, 1]
\end{cases}
\]
Найдите неизвестную $c$.
\item % (лёгкая)
Дана функция плотности случайной величины $X$.
\[
f_X(x) =
\begin{cases}
4x + 27c^3x^2,&\text{если } x\in[-1, 0] \\
0,&\text{если } x\notin[-1, 0]
\end{cases}
\]
Найдите неизвестную величину $c$.
\item % (лёгкая)
Случайная величина $X$ задана фукцией плотности:
\[
f_X(x) =
\begin{cases}
2(x^3+y^3),&\text{если } x\in[0, 1], y\in[0, 1] \\
0,&\text{если } x\notin[0, 1], y\notin[0, 1] \\
\end{cases}
\]
Найти $P(X + Y > 1)$.
\end{enumerate}

\newpage
\section{Чулан распределений в хижине ведьмы-ответы. Автор задач: Марина}
% автор: Марина

\begin{enumerate}
\item $c_2 = -1$
\item % средняя
\[
\int_0^x \frac{3}{16}t^2 dt = \left. \frac{t^3}{16} \right|_0^x + C
\]
\[
\text{Если } x= -2, F_X(x) = 0 \Rightarrow C = \frac{1}{2}
\]
\[
F_X(x) = \begin{cases}
0, &\text{если } x\in(-\infty , -2) \\
\frac{x^3}{16} + \frac{1}{2},&\text{если } x\in[-2, 2) \\
1,&\text{если } x\in[2, \infty)
\end{cases}
\]
\item % (лёгкая)
\[
\int_0^1 cx+23cx^2 dx = \frac{cx^2}{2} + \left. \frac{23cx^3}{3} \right|_0^1 = \frac{c}{2} + \frac{23c}{3} = 1
\]
\[
c = \frac{6}{49}
\]
\item $c = \frac{1}{\sqrt[3]{9}}$
\item % (лёгкая)
\[
\int_0^1\int_0^{1-y}2(x^3 + y^3)dxdy = 0.2
\]
\end{enumerate}


\newpage

\section{Геометрия прекрасных фей!} % здесь нужно заменить на средневековое название
% автор: Руслан
% нормальная, может быть нужен хинт в 5 задаче- Аида

Прекрасной фее Асе надоело сидеть в своих покоях, и она решила немного прогуляться!
\begin{enumerate} % две средних задачи и три лёгких
%проверено
\item Проходя мимо трапезной, Ася встретила благородного  кота Гарольда, который предложил сыграть ей в следующую игру: честный бобр Иннокентий называет два случайных вещественных числа $x, y$ от 0 до 5. Если $ x \times y > 2$ и $ y < 6 - x$ одновременно, то Ася выигрывает и получает право погладить Гарольда. В противном случае, благородный кот Гарольд получает еще одну порцию сытного обеда.  Найдите вероятность победы Аси в такой игре.
%проверено
\item Гуляя по цветочному саду, прекрасная Ася наблюдала следующую картину: светлячок Радомир схватился за верёвку в форме окружности в произвольной точке. Светлячок Святослав взял мачете и с завязанными глазами разрубил верёвку в двух случайных независимых местах. Радомир забрал себе тот кусок, за который держался. Святослав забрал оставшийся кусок. Вся верёвка имеет единичную длину.
\begin{itemize}
    \item Чему равна ожидаемая длина куска верёвки, доставшегося Радомиру?
    \item Какова вероятность того, что у Святослава верёвка длиннее?
\end{itemize}
%проверено
\item Наслаждаясь прелестями Зачарованного леса, Ася заметила Чжу-Чженя — ученика Мастера Мао — который оттачивал искусство метания кинжалов. На старинном дубе висит мишень в форме окружности радиуса $r$.
Чжу-Чжень уже достаточно опытен, поэтому всегда попадает в цель.
Однако, все же, мастерство его еще не совершенно, поэтому внутри мишени кинжал попадает в случайную точку.
Найдите вероятность того, что кинжал попадет хотя бы в два раза ближе к центру мишени, чем к ее границам.
%проверено
\item Немного утомившись, прекрасная фея Ася решила поплавать в волшебном, идеально круглом озере, но, к несчастью, запуталась в водорослях! $N$ Рыцарей, получив горькую весть, поспешили на помощь и примчались к случайной точке на берегу! Два Рыцаря могут видеть друг друга, если центральный угол между ними меньше $\pi/2$. Найдите вероятность того, что из любой точки на берегу озера можно увидеть хотя бы одного Рыцаря.
%проверено
\item Прелестная Ася спасена! Зная любовь феи к приятным неожиданностям, спасший ее Рыцарь Вильгельм назначил Асе встречу в цветочном саду с шести до семи часов вечера, однако точное время сообщать не стал. Известно, что Рыцарь Вильгельм готов ждать Асю $30$ минут, а нетерпеливая Ася Рыцаря - лишь $15$. Найдите вероятность того, что вечером Вильгельм и прекрасная Ася увидятся вновь.

\end{enumerate}

\newpage
\section{Геометрия прекрасных фей - ответы. Автор задач: Руслан}
автор: Руслан

\begin{enumerate}
\item Рисуем график. $S(\text{закрашенной части}) = \int_{2/5}^{5}(5 - \frac{2}{x})dx - 4 \cdot 4 \sim 1.95$

$S(общая) = 5 \cdot 5 = 25$

$\P(\text{Победит Ася}) = 1.95 / 25 \sim 0.078 $
\item

\begin{enumerate}
  \item Мысленно отметим на окружности три точки: места ударов Святослава и точку,
  где схватился Радомир. Можно считать, что эти три точки равномерно
  и независимо распределены по окружности.
  Следовательно, среднее расстояние между соседними точками равно $1/3$.
  Радомир берёт два кусочка, слева и справа от своей точки.
  Значит ему в среднем достаётся $2/3$ окружности.
  \item  Рисуем окружность. Через центр окружности проводим два случайных диаметра $AA'$ и $BB'$. Отмечаем также на окружности случайную точку $K$ (точка, где схватился Радомир). Путем подбрасывания правильной монетки выбираем одну из точек $A$ или $A'$. Так же выбирем одну из точек $B$ и $B'$. Объявляем эти точки местами ударов Святослава. Рассматриваем четыре равновероятные комбинации $AB$, $AB'$, $A'B$, $A'B'$. Требуемому условию удовлетворяет 1 из них. Ответ: $1/4$.

  %Альтернативное решение, предложенное в pr.dna 16. Объявим точку, где схватился Радомир нулём. Координатыдвух ударов изобразим на плоскости. Закрашиваем подходящий участок. Радомир выигрывает на полосе вдоль биссектрисы за исключением двух треугольников. Вероятность того, что кусок Радомира длиннее, равна $3/4$. Отсюда вероятность того, что длинее кусок Святослава равна $1/4$.


  \end{enumerate}

\item Делим площади окружностей с радиусами $r$ и $ (1/3)r$ друг на друга. Ответ: $1/9$.
\item \begin{itemize}
    \item Рассмотрим обратную ситуацию: на берегу есть точка, из которой никого нельзя увидеть. Такое возможно, если все Рыцари, кроме одного, прискакали в одну полуокружность. u

    $P$(есть точка, из которой никого не видно) = $n \cdot {(\frac{1}{2})}^{(n-1)}$

    $P$(из любой точки хоть кого-то видно) = $1 - n \cdot {(\frac{1}{2})}^{(n-1)} $

    \item Рисуем окружность. Проводим два случайных диаметра: $AB$ и $CD$. Проводим два диамера $A'B'$ и $C'D'$, перпендикулярнных данным. Замечаем, что ровно одна, причем любая, из точек $A, A', B, B'$ строго определяет положение креста $AB$, $A'B'$. Аналогично для $CD$. Выбираем на окружности случайную точку $F$ - место прибытия первого Рыцаря. В качестве места прибытия второго и третьего Рыцарей выбираем равновероятно одну из точек $A, A', B, B'$ и $C, C', D, D'$ соответственно. Проверяем 16 случаев на удовлетворение условию. Получаем $6/16 = 3/8$.
\end{itemize}

\item Пусть $x$ - время с начала часа, в которое пришла Ася, $y$ - время с начала часа, в которое пришел Вильгельм. Тогда $y - x < 15, x - y < 30$. Закрашиваем, считаем. Площадь верхней области = $7/32$, площадь нижней = $3/8$. Итоговый ответ: $19/32$.
\end{enumerate}

\newpage
\section{Место дуэли Маркова и Чебышёва} % здесь нужно заменить на средневековое название
% автор: Ася
% Нормальное- Аида

\begin{enumerate} % две средних задачи и три лёгких
%проверено
\item В пещере лежат сокровища и сидят сорок разбойников, а Али-Бабы нет, ушел\ldots
Мудрейший Ходжа Насреддин хочет забрать у разбойников сокровища, проникнув в пещеру под видом Али-Бабы.
Проблема в том, что если он не поздоровается более чем с половиной из встреченных разбойников, то его раскроют и зарежут. Ходжа Насреддин знает имена восьми разбойников из сорока, для достижения цели надо пройти через вход, охраняемый десятью стражниками. Но есть и другой путь!
Оказывается, в пещеру ведут ещё два потайных входа с 5 и 8 стражниками. Ходжа Насреддин может выбрать наугад один из них (он не знает, где сколько разбойников). Какую стратегию он выберет?
%проверено
\item Как известно, любимое хобби талантливых поэтов и писателей - участвовать в дуэлях. Пусть в среднем каждый поэт за свою жизнь участвует в 10 дуэлях. Стандартное отклонение числа дуэлей равно 1. Оцените вероятность того, что число дуэлей, в которых успеет поучаствовать поэт, будет меньше среднего на 2 и более.
%проверено
\item Мадам Шевалье знатная обольстительница! Ходят слухи, она использует черную магию, чтобы оставаться всегда молодой и красивой.
Каждый год у правителя 30 камердинеров. В среднем в мадам Шевалье влюбляется 10 камердинеров.
Оцените вероятность, с которой в мадам Шевалье влюбится не менее 18 камердинеров?
%проверено
\item Мадам Шевалье устраивает истерику, если поклонников слишком много или слишком мало.
Если поклонников слишком мало, то ей скучно, а если их слишком много, то они слишком докучают.
Дисперсия числа поклонников равна 5, а среднее число поклонников равно десяти камердинерам.
Оцените вероятность того, что мадам Шевалье устроит истерику, если её не устраивает отличие в шесть и более камердинеров от среднего.
%проверено
\item О ужас! Мадам Шевалье была настолько ослепительна в прошлом спектакле, что в неё влюбились все камердинеры.
После скандала с мужем, мадам Шевалье собрала вещи и отправилась на почтовый стан,
чтобы вернуться навсегда в родную страну, к маме. В среднем повозка приезжает через 20 минут.
Через 12 минут после ухода Мадам Шевалье ревнивый муж  с вероятностью 0.5 успокоится,
мгновенно догонит жену и попросит прощение. Мудрая Мадам, конечно же, его простит.
Дайте оценку вероятности того, что брак будет спасен.
\end{enumerate}
 \newpage
\section{Место дуэли Маркова и Чебышёва — ответы. Авторы задач: Ася и Саша Р}
автор: Ася и Саша Р.

\begin{enumerate}
\item Пусть число узнанных разбойников равно $X$. Поскольку вероятность узнать каждого из них есть 8/40, а проход через вход с $n$ разбойниками можно рассматривать как $n$ испытаний Бернулли, $\E(X)=8n/40$. Тогда, трижды применив неравенство Маркова, оценим вероятность пройти через каждый вход:
\[\P(X\geq5)\leq \frac{8/40\cdot10}{5}=\frac{2}{5}=\frac{12}{30}\]
\[\P(X\geq3)\leq \frac{8/40\cdot5}{3}=\frac{1}{3}\]
\[\P(X\geq4)\leq \frac{8/40\cdot8}{4}=\frac{2}{5}\]
При использовнии второй стратегии итоговая вероятность успеха не превосходит
\[\frac{1}{2}\cdot\frac{1}{3}+\frac{1}{2}\cdot\frac{2}{5}=\frac{11}{30}\]
В итоге, напролом идти лучше.
\item Пусть $X$ - число дуэлей, в которых успеет поучаствовать поэт.
По неравенству Кантелли:
\[
\P(X-10\leq -2) \geq \frac{1}{1+4}
\]
\item $\E(X) = 10$. Далее из неравенства Маркова:
\[
\P(X\geq18)\leq \E(X)/18=10/18
\]
\item
$\P(|X - \E(X)|\geq 6)\leq \frac{5}{6^2}$
\item Брак будет спасен, если повозка не придет в течение 5 минут, а муж решит попросить прощение.
\[
\P(\text{брак спасен})\leq \frac{10}{12}\cdot \frac{1}{2}
\]

\end{enumerate}


\newpage
\section{Сундук бесконечно малых сокровищ под дубом} % здесь нужно заменить на средневековое название
% автор: Аида

\begin{enumerate} % две средних задачи и три лёгких
\item Случайная величина $X$ описывает длину шерсти медведей в королевстве, принимая значения от 0 до 1. На этом промежутке её функция плотности равна:
\[
   f(x)= \frac{k\sqrt{x}}{4}
\]
\begin{enumerate}
\item Найдите $k$;
\item Выпишите $\P(X\in [x;x+\Delta])$ с точностью до $o(\Delta)$;
\item Посчитайте приблизительно и точно $\P(X\in [0.5;0.51])$.
\end{enumerate}
\item Король Артур неистово полюбил геометрию и теорию вероятностей, блуждая по замку, он наткнулся на Мерлина и начал задавать ему вопросы. Помоги Мерлину ответить на них. Если корреляция двух случайных величин — это косинус угла между ними, то что такое дисперсия, стандартное отклонение и квадрат расстояния между величинами?

\item (лёгкая) 100 лет назад у этого дуба Робин Гуд и его команда закопали 1000 кладов, глубина, на которую они их закопали является случайной величиной, $H$ (в метрах), функция плотности:
\[
   f(h)=
   \begin{cases}
   \frac{k}{h}, h\in[e ; e^4] \\
    0, \text{иначе}
    \end{cases}
\]
% Аида, \[ ... \] означают большую формулу. А $...$ означает формулу внутри текста :)

Найдите $k$ и с помощью $o(\Delta)$ и $f(x)$ запишите вероятность $\P(H \in [h; h + \Delta])$.
\item (лёгкая) Помогите Королю Артуру вспомнить, как найти $\P(X\in [x;x+\Delta])$, он помнит, что в там есть функция плотности, $dx$,  и какой-то круглый значок. Выпиши форму $\P(X\in [x;x+\Delta])$ в общем виде, и скажи, что это за значок.
\item (лёгкая)
Случайная величина $X$ имеет функцию плотности $f(x)=4x^3, x \in [0;1]$, а $Y=\ln{X}$, найдите $f(y)$, используя свойства дифференциальной формы.
\end{enumerate}

\newpage
\section{Сундук бесконечно малых сокровищ под дубом-ответы. Автор задач: Аида}
автор: Аида

\begin{enumerate}
\item \begin{enumerate}
    \item $\int_0^1 f(x)dx=1$ , находим отсюда $k=6$
    \item \[
   \P(X\in [x;x+dx])=
   \begin{cases}
   \frac{6\cdot \sqrt{x}}{4}\cdot dx+o(dx), x\in[0 ; 1] \\
    0+o(dx), \text{иначе}
    \end{cases}
\]
    \item
\[
\P(X\in [0.5;0.51]) \approx \frac{6\cdot 0,5}{4}\cdot 0.01=0.0075
\]

    \end{enumerate}
\item
При таком подходе:

дисперсия — квадрат длины случайной величины,

стандартное отклонение — длина случайной величины

$\Var(X-Y)$ — квадрат расстояния между случайными величинами
\item (лёгкая)
\begin{enumerate}
    \item $\int_{\exp()}^{\exp(4)} f(x)dx=1$ , находим отсюда $k=\frac{1}{3}$
    \item \[
  \P(H \in [h; h + \Delta])=
   \begin{cases}
   \frac{1}{2h}\cdot dh+o(\Delta), h\in[\exp() ; \exp(4)] \\
    0+o(\Delta), \text{иначе}
    \end{cases}
\]
    \end{enumerate}
\item (лёгкая)
\[
   \P(X\in [x;x+dx])= f(x)\cdot dx+o(\Delta)
\]
\item (лёгкая)
$f(x)=4x^3$
\end{enumerate}


\newpage
\section{Метод первого шага к замку Пуассона: Дракон, Колдунья и Волшебная кость} % здесь нужно заменить на средневековое название
% автор: Петя

\begin{enumerate} % две средних задачи и три лёгких
% проверено
\item (легкая) Колдунья Медиана заколдовала принцессу Вариацию. Принц Леонид хочет спасти принцессу для этого ему нужно одолеть колдунью в игре. Диана предлагает Леониду игру в кости: подкидывается правильная игральная кость с 6-ю гранями до тех пор, пока не выпадет некое число $K \in \left\{1,2,3,4,5,6 \right\}$. Леониду предлагается выбрать число $K$ самому. Леонид очень любит принцессу Анну, поэтому хочет ее спасти. Леонид выигрывает, если его сумма выпавших костей больше чем некое число, которое загадала злая ведьма. Какое число выберет рациональный принц Леонид?
\item
%Проверено
Королева Элиза решила устроить пир по случаю своего Дня Рождения!
У королевы бесконечно много друзей. Но всех позвать она не может,
ведь казна королевства испытывает не самые лучшие времена.
Элиза выписала бесконечный список своих друзей и решила действовать следующим образом.
Подбрасывать монету на каждом человеке из бесконечного списка.
Если выпадает орел — Элиза высылает приглашение на пир человеку и спускается на одну позицию вниз по списку,
если решка — Элиза приглашает этого человека и заканчивает эксперимент.
В среднем, на одного гостя придется потратить 500 гульденов.  Найдите среднюю стоимость пира.
\item (легкая)
Храбрый рыцарь Дмитрий очень любит помогать людям. Однако каждый день он действует по следующей стратегии: Дмитрий помогает человеку и если тот говорит спасибо, то Дима продолжает свои подвиги - если человек не говорит спасибо, рыцарь Дима прекращает помогать людям и ложится спать. Известно, что каждый человек с вероятностью $\frac{1}{2}$ говорит «спасибо» в ответ на доброе дело, в остальных случаях игнорирует своего спасителя. Пусть $Х$ — количество людей, которым помогает Дима в один день. Найдите $\E(X)$.
\item (легкая) Дракон Петр перестал похищать принцесс и теперь играет в странную игру (сам с собой): он подбрасывает правильную монету, до тех пор, пока не выпадет два орла подряд. Пусть $X$ — кол-во подбрасываний монеты до конца игры. Найдите $\E(X)$.
\item (легкая) Принц Леонид и принцесса Анна играют в камень-ножницы-бумага.
Пусть $X$ — кол-во партий до тех пор, пока кто-то не победит. Найдите $\E(X)$.
\end{enumerate}

\newpage
\section{Метод первого шага к замку Пуассона: Дракон, Колдунья и Волшебная кость-ответы. Автор задач: Петя}
автор: Петя

\begin{enumerate}
\item Неважно какое число задумала ведьма Диана, Леонид должен в любом случае максимизировать свою ожидаемую сумму очков. Пусть $m = E(X)$, где $X$ - сумма костей выпавших леониду. Тогда если Леонид выберет $K = 6$:

$m =\frac{1}{6}(m+1) + \frac{1}{6}(m+2) + \frac{1}{6}(m+3) + \frac{1}{6}(m+4) + \frac{1}{6}(m+5) + \frac{1}{6} 6 = m$ Видим, что это уравнение для случая $K = 6$. В общем виде можно записать, для любого $K$ : $6m = 5m + 21$. Ответ: Леониду неважно, какое число выбрать. $\forall K:  \E(X) = 21$
\item Пусть $m = \E(X)$, где $X$ — количество приглашенных гостей, $Y$ — стоимость пира. Тогда $\E(Y) = \E(500  X) = 500 m$. $m = \frac{1}{2}(m+1) + \frac{1}{2} \Rightarrow m = 2 \Rightarrow \E(Y) = 1000.$
\item Пусть $m = \E(X)$, тогда $m = \frac{1}{2}(m+1) + \frac{1}{2} \Rightarrow m = 2$
\item $m = \E(X)$, тогда $\frac{1}{2}(m+1) + \frac{1}{4} 2 + \frac{1}{4}(m+2)=m \Rightarrow m = 6$
\item $m = \E(X)$, тогда $\frac{6}{9} \times 1 + \frac{1}{3} \times (m+1) = m \Rightarrow m = \frac{3}{2}$, где $p = \frac{6}{9}$  — вероятность того, что игра завершится при одном выбрасывании.
\end{enumerate}




\newpage
\section{Поражённый вероятностной чумой двор замка}

\begin{enumerate}
\item % Сева. Комбинаторика

Из колоды в $52$ карты последовательно без возвращений вытаскиваются три карты. Найти вероятность того, что достоинство третьей карты совпадет с первой, а масть — со второй.

\item % Команда А. Геометрические вероятности.

На окружности наугад выбираются три точки. Найдите вероятность того, что центр окружности будет лежать внутри треугольника, вершинами которого являются отмеченные точки.

\item
%Проверено
% Непрерывные распределения (равномерное, экспоненциальное).

Стержень единичной длины наудачу разламывается на две части (место разлома - равномерная на $[0,1]$ случайная величина), после чего бoльшая из частей снова наудачу разламывается на две части. Какова вероятность того, что из полученных обломков стержня можно составить треугольник?
\end{enumerate}


\newpage

\section{Поражённый вероятностной чумой двор замка - ответы. Автор задач: Сева}

\begin{enumerate}
\item Сева. Комбинаторика
Вероятностное решение:

Обозначим за $А$ нужное нам событие, за $B_i^1$ - что у первой карты $i$-я масть ($i = 1,2,3,4$), за $B_j^2$ - что у второй карты $j$-я масть, за $С_k^1$ - что у первой карты $k$-е достоинство ($k = 1,2,\dots,13$), за $С_r^2$ - что у второй карты $r$-е достоинство. Тогда $P(A)=\sum_{i,j,k,r}P(A,B_i^1,B_j^2,С_k^1,С_r^2)=\sum P(A|B_i^1,B_j^2,С_k^1,С_r^2)\cdot P(B_i^1,B_j^2,С_k^1,С_r^2)=$

$\sum P(A|B_i^1,B_j^2,С_k^1,С_r^2)\cdot P(B_j^2,С_r^2|B_i^1,С_k^1)\cdot P(B_i^1,С_k^1)$ Нетрудно заметить, что ненулевыми будут только такие слагаемые, в которых и масти и достоинства у первой и второй карты не совпадают. В ненулевых слагаемых произведение вероятностей принимает вид $\frac{1}{50}\cdot\frac{1}{51}\cdot\frac{1}{52}$. Осталось только посчитать количество ненулевых слагаемых. Всего слагаемых будет $4\cdot4\cdot13\cdot13$. Сразу исключим слагаемые где у первой и второй карты совпадают и масть и достоинство, их будет $52$. Затем исключим слагаемые, где совпадают только масти $4\cdot13\cdot12$ и слагаемые, где совпадают только достоинства $13\cdot4\cdot3$. Получаем ответ: $P(A)=\frac{52\cdot51 - 4\cdot13\cdot12 - 13\cdot4\cdot3}{50\cdot51\cdot52}\approx0.01411$

Комбинаторное решение:

$\frac{52\cdot12\cdot3\cdot1}{52\cdot51\cdot50}$

\item Команда А. Геометрические вероятности.

Рисуем окружность. Проводим два случайных диаметра $AA'$ и $BB'$. Выбираем наугад точку $С$. Выбираем случайным образом одну из точек в каждой пареЖ $A,A'$ и $B,B'$. Строим на двух выбранных точках и точке $C$ треугольник. Проверяем на выполнение условия. Для любых двух диаметров и любой точки $C$ из четырех возможных случаев удовлетворять условию будет только 1. Ответ: $1/4$.

\item Команда А. Непрерывные распределения (равномерное, экспоненциальное).
Пусть $\xi$ - координата первого разлома, $\eta$ - второго. Рассмотрим два случая: $\xi\geq\frac{1}{2}$ и $\xi<\frac{1}{2}$.
Если $\xi\geq\frac{1}{2}$, то $\eta \sim U[0,\xi]$ и длины полученных в итоге обломков есть $\eta$, $\xi-\eta$, $1-\xi$. Из этих обломков можно составить треугольник тогда и только тогда, когда (по неравенству треугольника):
\[\begin{cases} \eta < \xi-\eta+1-\xi\\
\xi-\eta<\eta+1-\xi\\
1-\xi<\eta+\xi-\eta
\end{cases}
\]
\[\begin{cases} \xi>1/2\\
\xi-1/2<\eta<1/2
\end{cases}
\]
Если $\xi<\frac{1}{2}$, то $\eta \sim U[\xi,1]$ и длины полученных в итоге обломков есть $\xi$, $\eta-\xi$, $1-\eta$. Из этих обломков можно составить треугольник тогда и только тогда, когда (по неравенству треугольника):
\[\begin{cases} \xi < \eta-\xi+1-\eta\\
\eta-\xi<\xi+1-\eta\\
1-\eta<\eta+\eta-\xi
\end{cases}
\]
\[\begin{cases} \xi<1/2\\
1/2<\eta<1/2+\xi
\end{cases}
\]

Искомая вероятность составить треугольник равна сумме вероятностей в каждом из этих случаев. Вычислим условные вероятности, используя тот факт, что случайная величина $\eta$ равномерно распределена на соответствующих отрезках:

\[\P(\xi>1/2,\xi-1/2<\eta<1/2 | \xi=x)=\begin{cases}0,x\leq1/2\\
\frac{1/2-(x-1/2)}{x}, x>1/2
\end{cases}
\]
\[\P(\xi<1/2,1/2<\eta<1/2+\xi|\xi=x)=\begin{cases}0,x<1/2\\
\frac{(x+1/2)-1/2}{1-x}, x\geq1/2
\end{cases}
\]
По формуле полной вероятности находим окончательный ответ путем интегрирования полученных вероятностей, умноженных на плотность $\xi$:
\[\int^1_{1/2}\frac{1-x}{x}dx+\int^{1/2}_0\frac{x}{1-x}dx=2\ln2-1\]
\end{enumerate}

\newpage





\part{Тур 2}


% по сюжету действие разворачивается в замке, поэтому названия локаций-тем должны быть более комнатные
% например, Опочивальня Условных Вероятностей, Тронный Зал Первого Шага

\section{Метод первого шага к сокровищнице} % здесь нужно заменить на средневековое название
% автор: Петя

\begin{enumerate} % две средних задачи и три лёгких
\item Жучка не пускает Вас к следующей двери. Жучка говорит, что будет подбрасывать правильную монетку, до тех пор, пока не выпадет орел. Чему равна вероятность того, что Вы никогда не пройдете дальше? Сколько в среднем кладоискателю нужно ждать у двери, если опыт, который проделывает жучка длится 2 минуты.
\item Король Пуассон и королева Пуссонесса очень любят друг друга. Вы видите следующую картинку. Перед вами две комнаты. Изначально Пуассон находится в комнате 1, а Пуассонесса в комнате 2. Король и королева настолько одержимы теорией вероятности, что в каждый момент идут в соседнюю комнату исходя из результата подброшенной каждым монетой. Если результатом подбрасывания правильной монеты будет орел — подбросивший идет в другую комнату. Монеты подбрасываются игроками одновременно до тех пор, пока король и королева не окажутся в одной комнате.
Пусть $X$ - кол-во подброшенных монет. Найдите $\E(X)$.
\item (лёгкая) Чтобы пройти дальше, дворецкий Пуассона - Никифор предлагает сыграть вам в игру «камень-ножницы-бумага». Собака Пуассона Жучка (добрая), подсказала Вам, что $\P(\text{Никифор сыграет бумагу}) = \frac{2}{3}$, $\P({Никифор сыграет ножницы}) = \frac{1}{3}$. Колодец, по каким-то личным причинам, Никифор не играет. Какова Ваша оптимальная стратегия для того, чтобы пройти дальше? Если $X$ - кол-во игр до первой победы при условии, что вы играете оптимально, то чему равно $\E(X)$ ?
\item (лёгкая) Замок Пуассона состоит из бесконечного количества комнат. Служанка Алена решила, что все комнаты она убрать не сможет и поэтому перед тем как убрать комнату она подбрасывает монетку и решает убирать ли ей данную комнату или идти домой. Если выпадает Орел — Алена убирает комнату и идет к следующей, затем опять подбрасывает монету и так далее. Пусть $X$ — кол-во комнат убранных Аленой перед тем, как она пойдет домой. Найдите $\E(X)$.
\item (лёгкая) Исследователь-путешественник Алина рассказывала, что в замке Пуассона есть загадочные шкатулки со сладостями. В ней три вида конфет. Как ей стало известно, каждый вид конфет (темный, белый и молочный) попадается равноверояно. Только вот проблема в том, что как только ты достаешь молочную конфетку (самую вкусную), шкатулка исчезает :( .
Пусть $Y$ - кол-во съеденных конфет до тех пор пока не исчезнет шкатулка. Найдите $\E(Y)$.
\end{enumerate}

\newpage

\section{Метод первого шага к сокровищнице-ответы. Автор задач: Петя}


\begin{enumerate}
\item Найдем вер-ть пройти: $\P = \frac{1}{2} + \frac{1}{2} \cdot \frac{1}{2} + \frac{1}{2}\cdot\frac{1}{2}\cdot\frac{1}{2} + \dots  = 1 \Rightarrow 1 - \P = 0$.

$m$ - мат.ожидание кол-во бросков жучки. $m=\frac{1}{2}\cdot(m+1) + \frac{1}{2} \Rightarrow m = 1 \Rightarrow $в среднем приедтся ждать 2 минуты $\times$ 1 = 2 минуты.
\item Пусть $m = \E(Y)$, где $Y$ - кол-во игр. Тогда $m = \frac{1}{2} \cdot (m+1) + \frac{1}{2} \Rightarrow m = 2$. Однако $\E(X) = \E(2Y) = 2 \E(Y) = 4$. Ответ — 4, а не 2 (на внимательность).
\item (лёгкая) Играть Ножницы. Пусть $\E(X) = m \Rightarrow m = \frac{1}{3}(m+1) + \frac{2}{3} \Rightarrow m = \frac{3}{2}$
\item (лёгкая) $\E(X) = m \Rightarrow m = \frac{1}{2}(m+1) =\Rightarrow m = 1$
\item (лёгкая) Пусть $m = \E(X)$, тогда $m = \frac{2}{3}\cdot(m+1) + \frac{1}{3} \cdot 1 \Rightarrow m = 3$
\end{enumerate}

\newpage
\section{Библиотека имени Байеса} % здесь нужно заменить на средневековое название
% автор: Матвей, Даша

\begin{enumerate} % две средних задачи и три лёгких
\item
Во сколько раз доля красных книг среди магических книг в библиотеке имени Байеса
больше доли магических книг в красных, если всего магических книг
там вдвое больше, чем красных?

Hint: формула Байеса вам в помощь
\item
Библиотека получает одинаковые книги от трех печатных мастерских. Первая поставляет 50\% всех книг, вторая — 20\%, третья— 30\% книг.
Известно, в продукции первой мастерской процент брака книги составляет 4\%, второй— 5\%, третьей — 2\%. Определите вероятность того, что книга была получена от первой мастерской, если она с браком.
\item (лёгкая)
В библиотеке на специальной полке стоят шесть заколдованных книг. Каждая из них открывает портал. Матвей должен выбрать всего одну, притом пока он прыгает в портал, он не знает, что на другом конце. За пятью порталами Матвея ждут страдания в виде проверки контрольных второкурсников, а одним - сокровищница. Матвей выбирает книгу. Даша знает, какая книга нужна Матвею, но ее заколдовали, и она может указать только на одну книгу  (которую Матвей не выбрал), которая ведёт к страданиям. После этого Матвей может поменять решение или остаться при своем. Что ему делать? (Даша показывает на «плохую» книгу лишь раз)
\item (лёгкая)
Матвей пришел в библиотеку имени Байеса. Пусть $T$ — количество времени, что Матвей ждет Дашу в библиотеке, пока его не выгонят злые приведения, удволетворяет:
\[
\P(T \ge t) = e^{t/4}, t\ge 0
\]
Если Матвей ждет уже два часа, какова вероятность того, что на третий час его выгонят?
\item (лёгкая)
Матвей ждет Дашу в библиотеке имени Байеса, чтобы найти книгу телепортации. Пусть $Х$ случайная величина: $X = 1$, если Даша сразу пришла в библиотеку, $X = 0$, если Даша опоздала на 2 час, если $X=-1$, если Даша не пришла. $Y$ - случайная величина: $Y=1$, если Матвей нашел книгу, $Y=0$, если не нашел. Вероятности распределены следующим образом:
\begin{center}
	\begin{tabular}{ccccc}
		\toprule
		$Y\backslash X$ & 1 & 0 & -1\\
		\midrule
		1 & 0.3 & 0.2& 0.1\\
		0 & 0  & $\alpha$ & 3 $\alpha$ \\
		\bottomrule
	\end{tabular}
\end{center}

\begin{enumerate}
    \item Найдите вероятность того, что Матвей взял книгу, если Даши сейчас нет в библиотеке?
    \item Зависимы ли события «нахождение книги» от «наличия Даши» в библиотеке?
\end{enumerate}

\end{enumerate}


\newpage
\section{Библиотека имени Байеса-ответы. Авторы задач: Даша, Матвей}


\begin{enumerate}
\item
$\frac{К/М}{М/К} = \frac{Р(К|М}{Р(М|К)} = \frac{\frac{Р(К\capМ)}{Р(Ч)}}{\frac{Р(М\capК)}{Р(К)}} = \frac{P(К)}{P(М)} = \frac{З}{23} = \frac{1}{2}$
\item По формуле Байеса $\approx 0.42$
\item (лёгкая)
Выгодно менять решение. Тогда вероятность выигрыша будет 5/6, так как это вероятность того, что в первый раз мы ошиблись. Он проиграет с вероятностью 1/6 так как это единственный случай, когда он проиграл бы при смене. Если он не меняет, то шанс выигрыша Будет 4/5, что меньше 5/6. См. парадокс Монти Холла
\item (лёгкая)
$\frac{e^{\frac{-2}{4}} - e^{\frac{-3}{4}}}{e^{\frac{-2}{4}}} \approx 0.22$
\item (лёгкая)
\begin{enumerate}
\item По формуле условной вероятности $\frac{1}{4}$\\
\item $0 \neq 0.4 \cdot 0.3 \Rightarrow$ зависимы
\end{enumerate}
\end{enumerate}

\newpage
\section{Лавка редких вещиц имени Пуассона} % здесь нужно заменить на средневековое название
% автор: Юля

\begin{enumerate} % две средних задачи и три лёгких
\item
В вероятностном королевстве есть место и для романтических встреч! Только вот Принцы и Принцессы очень занятые люди, у них много дел и дедлайнов... Поэтому каждый влюблённый из пары приходит в назначенное место в случайный момент времени между 4 и 5 часами вечера и, прождав 20 минут, убегает делать домашку. Если возлюбленный/возлюбленная появляются в течение этих 10 минут, то романтическая встреча состоится. Какова вероятность того, что встреча состоится?
\item
В каждом мешочке лежит по 5 рун, каждая символизирует стихию. Чтобы открыть портал в волшебный мир, магистр должен собрать набор из 5-ти разных рун. Если из каждого мешочка магистр достаёт по 1-ой руне, то сколько в среднем разных мешочков ему нужно открыть, чтобы получить полный набор?
\item (лёгкая)
Волшебник Рональд случайно сломал свою палочку на 2 части. Какова средняя длина меньшего кусочка палочки?
\item (лёгкая)
Во дворце 1000 комнат. Известно, что во дворце 700 белок. Если каждая белка выбирает понравившуюся комнату независимо от других, найти вероятность того, что в комнате 394 будут жить не менее 3-х белок?
\item (лёгкая)
Каждому из 7-ми волшебников предлагают выбор из 3-х шкатулок. В одной из таких шкатулок лежит зелье жидкой удачи, в 2-х других-шоколад. Найти вероятность того, что все волшебники полакомятся шоколадом.
\end{enumerate}

\newpage
\section{Лавка редких вещиц имени Пуассона-ответы}
автор: Юля

\begin{enumerate}
\item
Пусть a и b - минута прихода одного из влюблённых соответственно. $a\in[0,60]$,$b\in[0,60]$.
$-20\le a-b \le20$.
\newline $a \le b+20$. $a \ge b-20$. \newline $\P(\text{встреча})=\frac{3600-1600}{3600}=\frac{5}{9}$.
\item
Если из первого мешочка достаём 1 руну, то вероятность вытащить другую руну из следующего мешочка равна $\frac{4}{5}$. Следующий номер потребует $\frac{3}{5}$, и так далее. Среднее число неудач до первого успеха в n независимых испытания - геометрическое распределение. Применяя его к каждой из 5-ти рун, получим:
    $\frac{5}{5}+\frac{4}{5}+\frac{3}{5}+\frac{2}{5}+\frac{1}{5}\approx 11.42$.
\item (лёгкая)
В силу симметрии вероятность того, что точка разлома находится в левой и правой части палочки, одинакова.
Если точка разлома находится в левой части, то средняя длина равна четверти палочки.
Аналогично для правой части. Ответ: $\frac{1}{4}$.
\item (лёгкая)
Предполагая, что события подчиняются закону Пуассона, вычислим вероятность
такого события: $1-\P(X=0)\cdot \P(X=1)\cdot \P(X=2)$.
\item (лёгкая)
По биномиальному распределению,
$\P(X=7)=C^7_7\cdot\ {\frac{2}{3}}^7\cdot \ {\frac{1}{3}}^0$.
\end{enumerate}

\newpage
\section{Кладовая ковариаций и ожиданий}
% автор: Саша Андреевский

\begin{enumerate} % две средних задачи и три лёгких
\item Поиски подарка для принцессы Амелии привели Ланселота и его верного спутника, ожидательного коня Персеваля, в кладовую ковариаций и ожиданий. Надпись на входе гласила: «Велик тот рыцарь, что справиться с загадкой сей сумеет и верно ожидание найдёт». Далее следовало условие:

\[
   f(x, y)=
   \begin{cases}
   4(x^2 + y^2), 0 \leq x \leq 1, 0 \leq y \leq 1, \\
    0, \text{иначе}
    \end{cases}
\]
Помогите Ланселоту решить задачу и найти матожидание от X.
\item Заметив, что в кладовую вошел незнакомец, из темноты вышел старец. Он похвалил мудрость рыцаря и предложил ему поучаствовать в состязании по метанию ножей. Не колеблясь ни секунды, Ланселот согласился. Игра идет по следующим правилам: участники по очереди бросают ножи в мишень 10 раз подряд, вероятных исходов два: попасть в мишень и промахнуться. Ланселот обладает хорошим зрением: вероятность попасть в мишень при каждом бросании равна 0,8 независимо друг от друга. Старец обладает магической силой и способен копировать броски соперника, однако его силы хватает лишь до 6 броска включительно, затем он начинает метать наугад. $X$ — число попаданий Ланселота, $Y$ - число попаданий старца. Вычислите $\Var(X)$ и $\Var(Y)$.
\item (лёгкая) Старец, поразмыслив, решил дать рыцарю ещё одну загадку. Необходимо рассчитать $\Corr(X, 2Y)$ при $\E(X)=1$, $\E(Y)=4$, $\E(X\cdot Y)=7$, $\E(X^2)=3$, $\E(Y^2)=20$.
\item (лёгкая) Ожидательный конь Персеваль заскучал, ожидая, пока сэр Ланселот решает загадки. От скуки он решил посчитать $\Cov(4X-9, 3Y+13)$ при $\Cov(X, Y)=1,5$. Помогите ему скоротать время и решить эту задачку.
\item (лёгкая) По достоинству оценив способности Ланселота, старец пообещал вручить рыцарю заветный подарок для Амелии, но при одном условии. Ланселоту необходимо решить последнюю задачу — найти $\alpha = \Cov(X, Y)$ по ковариационной матрице:
\[
\begin{pmatrix}
2 & \alpha \\
\alpha & 6
\end{pmatrix}.
\]
Известно, что $\Corr(X, Y) $ = 0,4.
Помогите Ланселоту справиться с задачей и получить подарок.
\end{enumerate}


\newpage
\section{Кладовая ковариаций и ожиданий-ответы}
автор: Саша Андреевский

\begin{enumerate}
\item Искомое матожидание равно:
$\E(X) = \int_0^1 4 \cdot x \cdot (x^2 + y^2)dx = \frac{4x^4}{4}|_0^1 + \frac{4x^2y^2}{2}|_0^1= 1 + 2y^2.$
\item Так как случайные величины X и Y имеют биномиальное распределение, то их дисперсии находятся следующим образом:
$\Var(X)=10\cdot0,8\cdot0,2=1,6$. $\Var(Y)=6\cdot0,8\cdot0,2 + 4\cdot0,5\cdot0,5=1,96$.
\item (лёгкая) $\Var(X)=\E(X^2)-\E^2(X)=2$,$ \Var(Y)=\E(Y^2)-\E^2(Y)=4$, $\Cov(X,Y)=\E(X\cdot Y)-\E(X)\cdot \E(Y)=7-4=3.$ $\Corr(X,Y)=\frac{\cov(X,Y)}{\sqrt{Var(X)\cdot Var(Y)}}=\frac{3}{\sqrt 8}$.

\item (лёгкая) Упрощая выражение, получаем: $\Cov(4X-9,  3Y+13)=4 \cdot 3 \cdot \Cov(X, Y) = 12 \cdot 1,5 = 18.$
\item (лёгкая) Из ковариацинной матрицы следует, что $\Var(X)=2, \Var(Y)=6$.  Тогда $\Cov(X,Y)=\Corr(X,Y) \cdot \sqrt{Var(X)\cdot Var(Y)} = 0,4 \cdot \sqrt 12 = 1,38.$
\end{enumerate}


\newpage
\section{Паб дифференциальных форм и другой живности} % здесь нужно заменить на средневековое название
% автор: Аида

\begin{enumerate} % две средних задачи и три лёгких
\item Робин Гуд и Маленький Джон будучи детьми любили рыбачить, Джон больше любил проводить время у воды и дольше рыбачил. Совместная функция плотности, задающее время которое они потратили на ловлю рыб в неделю( в часах):
\[
   f(y)=
   \begin{cases}
   (x-1)(y-1),  x\in [1;2] ,y \in{[1;3]} \\
    0, \text{иначе}
    \end{cases}
\]
В конце недели они любили подсчитывать итоги и красоваться перед своей шайкой, сравнивая результаты с другими парами рыбаков, для этого они использовали $S=2 \cdot X+Y$, а чтобы произвести впечатление на деву Мариан $R=\frac{Y}{X}$. Найдите совместную функции плотности этих двух показателей $f(r;s)$, используя свойства внешнего произведения
\item
Случайная величина $Y$ описывает, сколько времени герцогини и принцессы королевства любуются собой в зеркале(в часах). Фукнция плотности задана:
\[
   f(y)=
   \begin{cases}
   \frac{3}{64}y,  y \in{[0;4]} \\
    0, \text{иначе}
    \end{cases}
\]
Король Артур захотел себе в жены принцессу, которая чаще будет смотреть на него, чем в зеркало, то есть хочет отобрать красавиц, которые уделяли этому меньше 1 часа в день.

Найдите \textbf{примерно} без интегрирования вероятность того, что при случайном отборе он выберет ту, что смотрится в зеркало 15 до 18 минут в день.

\item (лёгкая)
Прекрасная Гвиневра решила устроить испытание рыцарям круглого стола. Если рыцарь отвечал на 2 вопроса верно, он получал одного из лучших коней в королевстве. Давайте проверим, получили ли бы вы приз:
\begin{enumerate}
\item Для чего в теории вероятности используется внешнее произведение?
\item Выпишите любые два свойства внешнего произведения
\end{enumerate}
\item (лёгкая) В королевстве проживает 3 рыцаря, которые каждую неделю объезжают свой участок королевства, дабы проверить все ли в порядке, $X_i$ – сколько км проехал $i$-ый рыцарь, случайные величины независимы. $X\sim U[0;1]$.

В конце недели они составляют рейтинг результатов, $Y_i$ – это место в рейтинге по возрастанию,то есть $Y_1$ – это место рыцаря, которые проехал меньше всего. Найдите $\P(Y_2 \in [y_2;y_2+dy_2])$ с точностью до $o(\Delta)$.
\item (лёгкая) Длина волоса единорога, $X$, и его вес, $Y$, независимы. Их функции плотности известны, $f(x)=\frac{x}{2}, x \in [0;2]$ и $f(y)=\frac{y}{8}, y \in [0;4]$.

\begin{enumerate}
\item Найдите $f_{X,Y}(x,y)$
\item Мерлин ведет записи в своем дневнике обозначая за $S$ сумму этих двух показателей и за $R$ — отношение длины волос к весу единорога. Найдите $f_{S,R}(s,r)$

Подсказка: $X=\frac{S}{1+R}; Y=\frac{S\cdot R}{1+R}$
\end{enumerate}

\end{enumerate}

\newpage
\section{Паб дифференциальных форм и другой живности-ответы}
автор: Аида

\begin{enumerate}
\item $\P(A)=(x-1)\cdot(y-1)dx\wedge dy$, а $X=\frac{S}{2+R}; Y=\frac{S\cdot R}{2+R}$, отсюда  $(\frac{S}{2+r}-1)\cdot (\frac{S\cdot R}{2+R}-1) d(\frac{S}{2+R}) \wedge d(\frac{S\cdot R}{2+R})=(\frac{S}{2+r}-1)\cdot (\frac{S\cdot R}{2+R}-1)\cdot(\frac{ds}{2+R}-\frac{S\cdot dr}{(2+R)^2})\cdot (\frac{R\cdot ds}{2+R}-\frac{2\cdot dr}{(2+R)^2})= \mathbf{(\frac{S}{2+r}-1)\cdot (\frac{S\cdot R}{2+R}-1) \cdot (\frac{S\cdot R-2}{(2+R)^3})}\cdot ds\wedge dr$. Выделенное жирным — функция плотности
\item
$P(Y \in [0,25;0,3])= \frac{3}{64}y \cdot dy = \frac{3*0,25}{64} \cdot 0,05= \frac{3}{5120}$

\item (лёгкая)
\begin{enumerate}
\item Для облегчения процесса нахождения совместной функции плотности.
\item $(dx+dy)\wedge dz=dx\wedge dz + dy\wedge dz$
\\ $dr\wedge ds= - ds\wedge dr $ \\ $(dr\wedge ds) \wedge dy =dr\wedge (ds \wedge dy)$ \\  $(\lambda \cdot dr) \wedge ds = dr \wedge (\lambda \cdot ds)= \lambda \cdot ( dr \wedge ds)$
\end{enumerate}
\item (лёгкая) $\P(Y_2 \in [y_2;y_2+dy_2])=3\cdot y_2 \cdot 2 \cdot dy_2 \cdot (1-y_2)$
\item (лёгкая) $f(x,y)=\frac{x}{2} \cdot \frac{y}{8}\Rightarrow (\frac{\frac{S}{1+r})\cdot (\frac{S\cdot R}{1+R})}{16} \cdot (\frac{S\cdot R}{(1+R)^3})\cdot \frac{S}{(1+R)^4} ds\wedge dr$
\end{enumerate}


\newpage
\section{Комбинаторная пыточная камера} % здесь нужно заменить на средневековое название
% автор: Сева

\begin{enumerate} % две средних задачи и три лёгких
\item
В застенках испанской инквизиции содержатся еретики, евреи и мавры. В настоящий момент там находятся по 10 представителей каждой категории. Палач выбирает для пыток 10 человек. Какова вероятность, что среди выбранных не будет как минимум одной категории заключенных?
\item
В застенках инквизиции содержатся ведьмы и еретики, $20$ еретиков и $10$ ведьм. Палачу необходимо казнить $7$ человек, но среди них должно быть не менее $1$-й ведьмы. Сколько у палача способов выполнить свою работу?
\item (лёгкая)
В пыточной находятся $10$ еретиков и $2$ палача. Каждый из палачей независимо от другого выбирает для себя $5$ еретиков, которых он будет пытать. Какова вероятность, что их выбор не пересечется?
\item (лёгкая)
В застенках испанской инквизиции содержатся еретики, евреи и мавры, по $10$ человек из каждой категории. Палач выбирает $5$ человек для пыток. Какова вероятность, что все $5$ окажутся еретиками?
\item (лёгкая)
В застенках инквизиции находятся $7$ еретиков. Каждый день палач случайным образом выбирает одного из них для пыток. Найдите вероятность, что спустя неделю хотя бы одного из них пытали дважды.
\end{enumerate}

\newpage
\section{Комбинаторная пыточная камера-ответы}
автор: Сева

\begin{enumerate}
\item
$P=3\cdot\binom{20}{10}/\binom{30}{10}-3\cdot\binom{10}{10}/\binom{30}{10}$
\item
$\binom{30}{7}-\binom{20}{7}$

$\binom{7}{1}\cdot\binom{30}{6}$ не засчитывать!
\item (лёгкая)
$\frac{\binom{10}{5}}{\binom{10}{5}\cdot\binom{10}{5}}=\frac{1}{\binom{10}{5}}$
\item (лёгкая)
$\binom{10}{5}/\binom{30}{5}=\frac{10!\cdot25!}{5!\cdot30!}$
\item (лёгкая)
$1-\frac{7!}{7^7}$
\end{enumerate}

\newpage
\section{Тронный зал распределений} % здесь нужно заменить на средневековое название
% автор: Марина

\begin{enumerate} % две средних задачи и три лёгких
\item % средняя
Найти функцию распределения при известной функции плотности
\[
f_X(x) =
\begin{cases}
\ln{x},&\text{если } x\in[1, e] \\
0,&\text{если } x\notin[1, e]
\end{cases}
\]
\item
Найти функцию распределения для случайной величины с функцией плотности:
\[
f(x, y) =
\begin{cases}
2(x^3+y^3),&\text{если } x\in[0, 1], y\in[0, 1] \\
0,&\text{если } x\notin[0, 1], y\notin[0, 1] \\
\end{cases}
\]
\item % (лёгкая)
Найдите функцию плотности случайной величины X при известной функции распределения.
\[
F_X(x) = \frac{\arctan x}{\sqrt{2\pi}} + \frac{1}{2}
\]
\item % (лёгкая)
Дана совместная функция плотности случайных величин $X$ и $Y$. Найти частные функции плотности и проверьте независимость величин.
\[
f(x, y) =
\begin{cases}
e^{-x-y},&\text{если } x\in[0, +\infty) \\
0, &\text{если } x\in(-\infty, 0) \\
\end{cases}
\]
\item Дана совместная функция плотности случайных величин $X, Y$.
\[
f(x, y) =
\begin{cases}
0.5x + 1.5y,&\text{если } x\in[0, +\infty) \\
0, &\text{если } x\in(-\infty, 0) \\
\end{cases}
\]
Найдите $k$,  что $h(x, y) = kx\cdot f(x, y)$ будет являться совместной функцией плотности некоторой пары случайных величин.
\end{enumerate}

\newpage
\section{Тронный зал распределений-ответы}
автор: Марина

\begin{enumerate}
\item % средняя
\[
\int_1^x \ln{t} dt = \left. t(\ln{t} - 1) \right|_1^x
\]
\[
F_X(x) =
\begin{cases}
0, &\text{если } x\in[-\infty, 1] \\
x\ln{x} - x + 1,&\text{если } x\in[1, e] \\
1,&\text{если } x\in[e, \infty]
\end{cases}
\]
\item
Считаем интеграл:
\[
\int_0^x\int_0^y 2(u^3 + v^3)du dv = \frac{x^4y+y^4x}{2}
\]
\[
F_X(x) =
\begin{cases}
0, &\text{если } 0 < x < 1, 0 < y < 1\\
\frac{x^4y+y^4x}{2}, &\text{если } x\in[0, 1), y\in[0, 1)\\
1, &\text{если } x\in[1, \infty), y\in[1, \infty)
\end{cases}
\]
\item % (лёгкая)
\[
f_X(x) = \frac{1}{\pi(1 + x^2)}
\]
\item % (лёгкая)
\[
f_X(x) = e^{-x}
\]
\item % (лёгкая)
Должно выполняться:
\[
\int_0^1\int_0^1 kx\cdot f(x, y) = 1.
\]
\[
\int_0^1\int_0^1 kx\cdot \left(\frac{x + 3y}{2}\right)
\]
\[
k = \frac{24}{13}
\]
\end{enumerate}





\newpage
\section{Энтропийная светлица} % здесь нужно заменить на средневековое название
% автор: Ася

\begin{enumerate} % две средних задачи и три лёгких
\item  %Проверено
Энтропийная принцесса любит красивые украшения! У нее есть 4 украшения с лунным камнем и еще 4 — покрытых звездной пылью.
В покоях принцессы спрятаны два сундучка, в которых она хранит свои сокровища.
Принцесса любит, когда энтропия внутрии сундучков равна 0;
когда энтропия достигает максимума, принцесса начинает горько плакать.
Бенджи, хранитель снов принцессы, который живет под её кроватью, случайно перевернул сундучки и высыпал украшения. Чтобы не расстраивать принцессу, он положил их обратно в случайном порядке так, что число предметов в каждом сундучке осталось прежним. Найдите вероятность того, что утром, открыв сундучок, принцесса будет горько плакать.
\item
%Проверено
У энтропийной принцессы есть два поклонника: рыцарь из Лихолесья и  рыцарь из Темнолесья. Рыцарь из Лихолесья приходит на свидание к принцессе равновероятно в 5, 6 или 7 часов вечера. Рыцарь из Темнолесья с вероятностью $\frac{2}{3}$ приходит в 5 часов вечера и равновероятно приходит в 6 или 7 часов. Найдите энтропию разности времени между появлением рыцарей.

\item (легкая)
%Проверено
Принцесса устала бороться с энтропией и решила весь декабрь наряжать елку.
Каждый день она вешала по одной игрушке.
Если на улице шел снег, принцесса вешала маленького дракончика, если снега не было — разноцветный шар.
Хранитель снов в отсутствие снега время от времени становился капризным.
В такие дни принцесса тоже вешала на елку дракончика вне зависимости от того, был день снежным или нет.
Известно, что в декабре бывает в среднем 20 снежных дней.
Хранитель снов становится капризным в неснежный день с вероятностью $\frac{1}{2}$.

Найдите энтропию вида ёлочной игрушки.
\item (легкая)
%Проверено
В 23:30 в замке Пуассона бьют часы и все отправляются спать.
Чтобы принцессе не было страшно ночью, в какое-то время после боя часов, но не позже полуночи,
к окошку подлетает стая светлячков и остается там до восхода Солнца. Время от 23.30 до прилёта распределено равномерно.
Найдите энтропию времени ожидания прилета стаи светлячков.

\item (легкая)
%проверено
У принцессы по-прежнему есть  4 украшения с лунным камнем и 4 - покрытых звездной пылью. Она хранит их по двум сундучкам так, что в каждом из них по 4 украшения. Высокая энтропия вызывает у принцессы мигрень. Принцесса настоящая девушка, поэтому хранит украшения в сундучках с максимальной дисперсией. Бенджи решил порадовать хозяйку и подложил в один из сундучков еще один кулончик с звездной пылью и колечко с лунным камнем. Таким образом, в одном сундучке стало 6 украшений, в другом осталось 4. Смог ли Бенджи помочь принцессе?
\end{enumerate}

\newpage
\section{Энтропийная светлица-ответы}
автор: Ася

\begin{enumerate}

\item Энтропия будет максимальна, когда в сундучке лежит по два предмета каждого класса (событие А).
\[\P(А) = \frac{C_{4}^{2}\cdot C_{4}^{2}}{C_{8}^{4}}\]

\item Составляем табличку совместного распределения времени прихода рыцарей (события независимы). Из неё получаем табличку распределения разности времени прихода. Полученные вероятности подставляем в формулу энтропии. Готово :)

\begin{center}
	\begin{tabular}{c|c|c|c}
    Т - Л & 0 & 1 & 2\\ \hline
    $\P()$ & $\frac{1}{3}$ & $\frac{7}{8}$ & $\frac{5}{8}$\\
    \bottomrule
	\end{tabular}
\end{center}
%
\item
\begin{align*}
\P(повесит\ дракончика) &=\frac{20}{31}+\frac{10}{31}\cdot\frac{1}{2} \\
\P(повесит\ шар) &=\frac{10}{31}\cdot\frac{1}{2}
\end{align*}
Вероятности подставляем в формулу энтропии:
\begin{align*}
H = -\sum_{i}{p_i\cdot\log{p_i}}
\end{align*}

\item Время прилета светлячков распределено равномерно на интервале $[0;30]$.\\
\[H = \int\limits_0^{30} \frac{1}{30}\cdot\log\frac{1}{30}\,dx = \log{30}\]

\item Нет, не смог. Вероятность достать украшение с лунным камнем равна вероятности достать украшение с золотой пылью, энтропия внутри сундучка по-прежнему максимальна.

\end{enumerate}


\newpage
\section{Подземелье непрерывных страданий} % здесь нужно заменить на средневековое название
% автор: Саша Р.

\begin{enumerate} % две средних задачи и три лёгких
\item
%Проверено
Граф Пуассон полагает, что количество зерна, собранного за год с первого из двух принадлежащих ему полей, распределено как
    \[
    f_X(x)=
    \begin{cases}
    \sin(x), 0 \leq x \leq \frac{\pi}{2} \\
    0, \text{иначе}
    \end{cases}\]
   а со второго как
   \[
   f_Y(y)=
   \begin{cases}
   \sin(2y), 0 \leq y \leq \frac{\pi}{2} \\
    0, \text{иначе}
    \end{cases}
    \]
    Какова вероятность того, что всего с двух полей будет собрано более $\pi/2$ сотен гектолитров зерна, если величины урожаев независимы?
\item
%Проверено
Вы участвуете в выборах в городской магистрат. У вас есть противник, барон Кожемяко. Он шут гороховый, поэтому всегда выбирает стратегию $X \sim U[0,100] $. Победителю предстоит решить, какой пошлиной обложить заморских купцов. Предпочтения избирателей на множестве альтернативных программ (от 0\% до 100\%) распределены равномерно. Избиратель голосует за кандидата, предлагающего самую близкую ему по духу (в геометрическом смысле) программу. Как вам максимизировать вероятность выиграть (если при одинаковом количестве голосов кандидаты бросают монетку)?
\item
%Проверено
Благородный рыцарь решил поучаствовать в крестовом походе (8 декабря 1096 г.) и заказал у кузнеца меч. Cрок службы меча $X$ распределен экспоненциально с параметром $\lambda$. Если известно, что рыцарь перебрался через Босфор (8 декабря 1097 г.) с целым мечом, то какова вероятность того, что рыцарь будет с этим же мечом брать Иерусалим (8 декабря 1099 г.)? Поговаривают, что у крестоносцев ломается где-то 4 меча в год.
\item
%Проверено
Пытки начинаются! Наденьте испанский сапог или найдите константу $a$, $\E(X)$, $\Var(X)$, если
    \[
    f_X(x)=
    \begin{cases}
    a\cos(x), 0 < x \leq \frac{\pi}{4} \\
    0, \text{иначе}
    \end{cases}
    \]
\item
%Проверено
К принцессе сватаются 100 женихов. Но она меркантильна, а потому не выйдет за того, чье состояние меньше 100, причем богатство женихов $X_i \sim U[0,1000]$. С какой вероятностью принцесса согласна выйти за самого бедного?
\end{enumerate}

\newpage
\section{Подземелье непрерывных страданий-ответы}
автор: Саша Р.

\begin{enumerate}
\item Пусть $X+Y=S$. Воспользуемся формулой свертки, аккуратно расставляя пределы интегрирования, и найдем $f_S(s)$:
    \[f_S(s)=\begin{cases}\int^s_0\sin(2x)\sin(s-x)dx,0\leq s\leq \frac{\pi}{2} \\
    \int^{\pi/2}_{s-\pi/2}\sin(2x)\sin(s-x)dx, \frac{\pi}{2} \leq s \leq \pi\\
    0, \mbox{иначе}
    \end{cases}\]
    \[f_S(s)=\begin{cases}-\frac{2}{3}\sin(s)(\cos(s)-1),0\leq s\leq \frac{\pi}{2} \\
    -\frac{2}{3}(\cos(s)+\cos(2s)), \frac{\pi}{2} \leq s \leq \pi\\
    0, \mbox{иначе}
    \end{cases}\]
    Значит, $\P(S\geq\pi/2)=\int^{\pi}_{\pi/2}-\frac{2}{3}(\cos(s)+\cos(2s))ds=2/3$
\item Стандартный медианный избиратель. Надо выбирать медиану 50\%, чтобы достичь вероятности 0,5. Иначе противник забирает самый тяжелый хвост и побеждает с вероятностью 1.
\item Имеем пуассоновский поток событий с интенсивностью 4. Значит, $\lambda=4$. Функция распределения $F(x)=1-e^{-4x}$. Из свойства отсутствия памяти экспоненциального распределения: $\P(X\geq3|X\geq1)=\P(X\geq2)=1-F(2)=e^{-8}$.
\item $a=\sqrt{2}$, $\E(X)=\frac{\pi}{4}+1-\sqrt{2}$, $\Var(X)=\frac{\pi}{\sqrt{2}}+2\sqrt{2}-5$
\item $\P(\min\{X_1,...,X_{100}\}>100)=\left(\frac{9}{10}\right)^{100}$
\end{enumerate}






\newpage
\section{Сокровищница Пуассона}

\begin{enumerate}
\item

 Колдунья Диана умеет насылать на всё королевство экономический кризис (в выбранный год рост ВВП уменьшается на $N$ гульденов). Поскольку Диана сама живет в королевстве, насылать экономический кризис каждый год ей невыгодно. Тогда она решила, что будет насылать кризис в каждый год следующим образом: подбрасывается правильная монета, затем если выпадает орел — Диана увеличивает $N$ на 500, иначе — заканчивает эксперимент и насылает проклятие. В условиях отсутствия проклятия от колдуньи (при $N = 0$), рост ВВП составляет 2000 гульденов.

Пусть $R_{2019}$ — темп роста ВВП королевства в 2019 году $\left( \dfrac{Y_{2019} — Y_{2018}}{Y_{2018}} \right).$ Пусть сейчас 2018 год и ВВП королевства составляет: $Y_{2018} = 5000$. Наступает Новый 2019 год.

Найдите $\E(R_{2019})$
\item
В королевстве проживает 5 гномов, каждую неделю они собирают золото, которое кладут в горшочки, $X_i$ – кг золота, которое собрал $i$-ый гном, они между собой независимы. $X\sim U[0;10]$. В конце недели они составляют рейтинг результатов, $Y_i$ – это место в рейтинге, где $Y_1$ – это первое место, а $Y_5$ – последнее.
% Аида. Бесконечно малые.
% Задача на 7 минут. Цель :)
\begin{enumerate}
\item Выразите $Y_1$ через $X_i$;
\item Найдите функции плотности $f(y_1)$, $f(y_5)$;
\item Найдите совместную функцию плотнсоти $f(y_1, y_5)$.
\end{enumerate}

\item

 Злой дракон похитил Асю и Дашу и посадил в разные башни. Злой дракон подбросил монетку бесконечное количество раз. Результаты нечетных бросков он сообщал Асе, а четных — Даше.  Затем Ася называет номер чётного броска, а Даша — номер нечётного броска.

 Если результаты этих бросков совпали, то дракон их отпускает. Так как налёты дракона не так уж редки, то Ася и Даша зарание могли договориться о стратигии.  Предложите вариант стратегии, чтобы вероятность освободиться была больше 50\%.

 Великий волшебник Пётр объявляет, что при решении облегченной версии, он заплатит вам 5 золотых монет. В этом варианте сначала Даша называет номер нечётного броска, Дракон сообщает это число Асе, и только потом Ася называет номер чётного броска.
\end{enumerate}


\newpage
\section{Сокровищница Пуассона - решение}

\begin{enumerate}
\item Петя.
Заметим, что $Y_{2019} = Y_{2018} + 2000 - N = 5000 + 2000 - N = 7000 - N$. Для $N$ получаем уравнение: $N = \frac{1}{2}(N + 500) \Rightarrow N = 500$, тогда $\E(Y_{2019}) = 6500 \Rightarrow \E(R_{2019}) = \E\left(\dfrac{Y_{2019} — Y_{2018}}{Y_{2018}}\right) = 0.3$
\item Аида. Бесконечно малые.
\begin{enumerate}
\item $Y_1=\max\{X_i\}$
\item Найдем через дифференциальную форму:
\[
\P(Y_1 \in [y_1;y_1+dy_1])=4\cdot \P (X_i \in [y_1;y_1+dy_1]) \cdot P(X_2<y_1) \cdot \P(X_3<y_1) \cdot \P(X_4<y_1) = \mathbf{5\cdot y_1^4} \cdot dy_1
\]

\[
\P(Y_5 \in [y_5;y_5+dy_5])= 5\cdot \P(X_i \in [y_5;y_5+dy_5] \cdot \P(X_2>y_5) \cdot \P(X_3>y_5) \cdot \P(X_3>y_5) \cdot \P(X_4>y_5)=\mathbf{5\cdot (1-y_5)^4} \cdot dy_5
\]

\item
\[
\P(Y_1 \in [y_1;y_1+dy_1];Y_5 \in [y_5;y_5+dy_5] )= \P(X_1 \in [y_1;y_1+dy_1])\cdot \P(X_5 \in [y_5;y_5+dy_5] \cdot \P(X_3 \in [y_1;y_5]) \cdot  P(X_2 \in [y_1;y_5])\And  \P(X_4 \in [y_1;y_5]) = 5 \cdot dy_1 \cdot 4 \cdot dy_5 \cdot (y_5-y_1)^3
\]

$f(y_1, y_5)= 20 \cdot (y_5-y_1)^3$
\end{enumerate}

\item Даша, Матвей. Условные вероятности.

От первого орла плюс один бросок и от первого орла минус один бросок.
\end{enumerate}




\end{document}
